%% scribe appears to not distinguish ru and rū —— confirm
\documentclass[12pt]{article}
\usepackage{microtype}
\usepackage[pdfusetitle,hidelinks]{hyperref}
\usepackage{setspace}
\usepackage{embrac}

\usepackage[series={A,B},noend,noeledsec,noledgroup]{reledmac}
\renewcommand*{\thefootnoteA}{\roman{footnoteA}}
\usepackage{marginnote}
\usepackage{longtable}

\usepackage{fontspec}
\usepackage{polyglossia} 		
	\setmainlanguage[script=latin]{sanskrit}
	\setotherlanguage{english}	
	\setmainfont[Ligatures=TeX]{Libertinus Serif}
	\newfontfamily\sanskritfont[Ligatures=TeX]{Libertinus Serif}
	\newfontfamily\mantrafont[Ligatures=TeX]{STIX Two Text}[Scale=MatchLowercase]
	\newfontfamily\devfont[Script=Devanagari]{Noto Sans Devanagari}

\newcommand{\crux} {\hspace{0em}\textsuperscript{†}\hspace{0em}}
\newcommand{\emdash} {\hspace{0em}—\hspace{0em}}

\title{Tattvaratnāvaloka and Vivaraṇa}
\author{Vāgīśvarakīrti}

\setcounter{secnumdepth}{2}
\renewcommand{\thesection}{}
\renewcommand{\thesubsection}{\arabic{subsection}}

% Increase the depth of sectioning in the table of contents
\setcounter{secnumdepth}{4} % Numbering depth
\setcounter{tocdepth}{4}    % Table of contents depth

% Define \subsubsubsection
\makeatletter
\newcounter{subsubsubsection}[subsubsection] % Create a counter for subsubsubsections
\renewcommand\thesubsubsubsection{\thesubsubsection.\arabic{subsubsubsection}} % Define numbering

\newcommand\subsubsubsection{\@startsection{subsubsubsection}{4}{\z@}%
  {-3.25ex \@plus -1ex \@minus -.2ex}%
  {1.5ex \@plus .2ex}%
  {\normalfont\normalsize\bfseries}}
\newcommand\subsubsubsectionmark[1]{}
\makeatother

% Adjust formatting (optional)
\usepackage{titlesec}
\titleformat{\subsubsubsection}[runin]
  {\normalfont\normalsize\bfseries}{\thesubsubsubsection}{1em}{}
\begin{document}
\maketitle

% Define a new intercharacter class for Sanskrit question marks
\makeatletter
\newXeTeXintercharclass\noextraclass
\XeTeXcharclass `\? = \noextraclass
\XeTeXcharclass `\! = \noextraclass
\XeTeXcharclass `\; = \noextraclass
\XeTeXcharclass `\: = \noextraclass

% Do not italicise / and |
\AddOpEmph{|}
\AddOpEmph{/}

% Remove the extra space between Sanskrit text and ? or !
\XeTeXinterchartoks 0 \noextraclass = {\nobreak}
\XeTeXinterchartoks \noextraclass 0 = {\nobreak}
\makeatother

%sigla
\newcommand{\PCreading}{$^{pc}$}
\newcommand{\ACreading}{$^{ac}$}
\newcommand{\MS}{K}
\newcommand{\EDD}{E\textsubscript{DH}}
\newcommand{\TM}{TM\textsubscript{D}}
\newcommand{\TVA}{TVA\textsubscript{D}}
\newcommand{\TVB}{TVB\textsubscript{N}}
\newcommand{\TIB}{TIB}
\newcommand{\sigmareading}[1]{$\Sigma$\textsubscript{#1}}

% App shortcuts
\newcommand{\emd} {\emph{em.}}
\newcommand{\conj} {\emph{conj.}}
\newcommand{\possibleconj} {\emph{possible conj.}}
\newcommand{\corr} {\emph{corr.}}
\newcommand{\diag} {\emph{diag.\ conj.}}
\newcommand{\possibleemd} {\emph{possible em.}}

\section*{Sigla and Abbreviations}
\noindent\begin{longtable}{ l p{12cm} }
	\noindent TaRaA & Tattvaratnāvaloka\\

	\noindent TaRaA-Vi & Tattvaratnāvalokavivaraṇa\\

	\noindent \EDD\ & Dhīḥ vol. 21, pp.\ 129–149.\\

	\noindent \MS\ & NAK 5–252 = NGMPP A 915/4\\

	\noindent \TM & \emph{De kho na nyid rin po che snang ba}. Tōhoku no.\ 1889. sDe dge bstan 'gyur, vol.\ Pi, fols.\ 203r3–204r5. Tr.\ by 'Gos Lhas btsas\\

	\noindent \TVA & \emph{De kho na nyid rin po che snang ba'i rnam par bshad pa}.  Tōh.\ 1890. sDe dge bsTan 'gyur, vol.\ 44 (rGyud 'grel, Pi), fols.\ 204r5–214v4. Tr.\ by 'Gos Lhas btsas.\\

	\noindent \TVB & \emph{De kho na nyid rin po che snang ba'i rnam par bshad pa}. Ōtani no.\ 4793. bsTan 'gyur gSer bris ma, vol.\ 84 (83 in BDRC outline(?)), (rGyud 'brel, Zhu), fols. 70v–85v. translator given.\\

	\noindent \TIB & Both Tibetan translations (differences, if any, indicated in a mini-aparatus)\\

\bigskip
$ac$ & \emph{ante correctionem} \\
\emph{deest} & omitted in \\
\diag & diagnostic conjecture [e.g.\ `reconstructed' from Tibetan]\\
\conj & conjecture\\
\emd & emendation [an emendation is made with a high degree of confidence, whereas a conjecture proposes a correction while acknowledging a greater possibility for alternatives]\\
fol./fols. & folio/folios \\
$pc$ & \emph{post correctionem} \\
$r$ & recto \\
$v$ & verso \\
$\Sigma$\textsubscript{X} & Reading shared in all witnesses but X \\
((kiṃcit)) & Reading uncertain—either illegible or otherwise in doubt \\
<kiṃcit> & Reading cancelled \\
\crux kiṃcit\crux & Reading does not make sense to the editor and an adequate conjecture was not able to be chosen. \\
{[}kiṃcit{]} & Indication of a diagnostic conjecture  \\
.. & Damaged \emph{akṣara} (one . per half \emph{akṣara}) \\
... & Lacunae of an unknown quanity of \emph{akṣara}s \\
° & Mark of abbreviation \\
\end{longtable}

\section*{Text}
\subsection{Maṅgalācaraṇa}
\begin{quote}
	[\MS\ fol.\ 1r] [siddhaṃ]\footnoteB{
		[siddhaṃ]] \MS ; oṁ \EDD
	} namaḥ śrīsadgurupādebhyaḥ |\footnoteA{
		Scribal homage
	}
	
%	[\TM\ fol.\ 203r]  || rgya gar skad du | tattv'a ratna a'a lo ka | bod skad du | de kho na nyid rin po che snang ba | bcom ldan 'das 'jam pa'i rdo rje la phyag 'tshal lo || 

	anupamasukharūpī śrīnivāso 'nivāso \\
	nirupamadaśadevīrūpavidyaḥ\footnoteB{
		nirupama°] \EDD\ ; nirūpama° \MS
	} savidyaḥ |\\
	tribhuvanahitasaukhyaprāptikāro 'vikāro \\
	jayati kamalapāṇir yāvad āśāvikāśāḥ\footnoteB{
		āśāvikāśāḥ] \corr ; āśāvikāsāḥ \MS\ \EDD
	} || 1 ||
	% LLLLLLGGGLGGLGG Mālinī
	
% 	So long as there are opportunities for hopes/spaces, the Lotus Hand reigns supreme\emdash he whose nature incomparable bliss, who is an abode of riches, who is without any abode, whose consorts are the incomparable ten goddesses, who is accompanied by his consort, who brings about the attainment of bliss that has benefit for the three worlds, and who is without degeneration.
\end{quote}

\medskip\noindent [\MS\ fol.\ 2r3] namaḥ samantakāyavākcittavajrāya.\footnoteA{
	Scribal homage
}\\

\noindent anupametyādi.
kamalaṃ padmaṃ pāṇau yasya sa kamalapāṇir avalokiteśvaro bhagavāñ jayatīti sambandhaḥ.
kiṃviśiṣṭaḥ?
anupamam ity atipraṇītatvamahattvāsaṃsārasthāyitvalakṣaṇair\footnoteB{
	°saṃsārasthāyitva°] \MS; °saṃsārasthāyisva° \EDD\ (\emph{note the two akṣaras}, tva \emph{and} sva, \emph{are very similar})
} dharmair yuktasyānyasyābhāvād\footnoteA{
	cf.\ Tib.: dpe med ces bya ba la sogs pa smos te/ dpe med pa ni (ni] \TVA; dang \TVB) shin tu gya nom pa nyid dang/ rgya (rgya] \TVA; deest in \TVB) che ba nyid dang/ 'khor ba'i mtha'i bar du gnas pa'i mtshan nyid kyi chos dang ldan pa ste/ gzhan dag la de med pa'i phyir ro/ / (āha—anumapetyādi. anupamam iti atipraṇītatvamahattvāsaṃsārasthāyitvalakṣaṇair yuktam, anyasya tadabhāvād.)\\
} upamārahitaṃ sukham eva rūpaṃ svabhāvo yasya sa tathoktaḥ.
punar api kiṃviśiṣṭaḥ?
śrīḥ puṇyajñānasambhāralakṣaṇā, tasyā nivāsa āśrayo yaḥ sa tathā.
dharmakāyarūpitvena\footnoteB{
	dharmakāyarūpitvena] \MS\ \EDD; dharmakāyarūpatvena \possibleemd\ (\emph{cf.} \TVA\ \TVB: chos kyi sku'i ngo bo nyid kyis)
} sarvagatatvāt [\EDD\ p.\ 132] pratiniyatanivāsābhāvād anivāsaḥ.

punaḥ kīdṛśaḥ?
nirupamāḥ paramarūpayauvanaśṛṅgārādirasamahākaruṇādiyuktatvenopamātikrāntā\footnoteB{
	°opamātikrāntā] \MS\ \EDD\ \TVB\ (dpe las ’das pa’o) ; dpe med pa ste/ dpe las ’das pa’i \TVA\ (nirupamā upamātikrāntā)
} rūpavajrāditārāparyantadaśadevīrūpā vidyāḥ parivārakatvena\footnoteB{
	parivārakatvena] \emd ; saparivārakatvena \MS ; saparivārakatvena \EDD
} yasya sa tathā.
saha svābhārūpayā vidyayā\footnoteB{
	vidyayā] \MS\ \EDD ; rig pa ste/ shes rab \TVA\ \TVB\ (vidyayā prajñayā)
} vartata iti savidyaḥ.
tribhuvanasya tribhuvanavartino janasya yaddhitam āyatipathyaṃ\footnoteB{
	āyatipathyaṃ] \emph{variant word division in} \EDD : āyati pathyaṃ; \emph{and in} \MS : āyati | pathyaṃ
} buddhatvādikam, saukhyaṃ tad āpātapathyaṃ\footnoteB{
	tad āpātapathyaṃ] \conj\ (\TVA : 'phral gyi phan pa); tad dāpayati pathyaṃ \MS\ \EDD ; de la bde ba ni bde ba ste \TVB
} cakravartitvādikam, tasya yā prāptiḥ\footnoteB{
	prāptiḥ] \MS\ \EDD ; thob pa ni rnyed pa ste \TVA\ \TVB
} [\MS\ fol.\ 2v] sākṣāt kriyā, tasyāḥ karaṇaṃ kāro yasya sa tathā.
aparinirvāṇadharmakatvenāpratiṣṭhitanirvāṇarūpatvenā\footnoteB{
	°rūpatvenā°] \MS\ \EDD ; ngo bo rnyed pas \TVA ; ngo bo brnyed pas \TVB\ (°rūpaprāptyā°)
}nyathātvalakṣaṇasya vikārasyābhāvād avikāraḥ.
evaṃviśiṣṭo bhagavāñ jayati.
 
kiyantaṃ kālam ity āha\emdash yāvad āśāvikāśāḥ.\footnoteB{
	āśāvikāsāḥ] \corr ; āśāvikāśāḥ \EDD\ \MS
} āśā daśa diśo gaganasvarūpāḥ. yadvā āśāḥ sarvasattvānāṃ bhavabhogatṛṣṇāḥ.\footnoteB{
	°tṛṣṇāḥ] \EDD\ (°tṛṣṇās); tṛṣṇā \MS
} tāsāṃ vikāśā\footnoteB{
	vikāśā] \corr; vikāsā \MS\ \EDD
} avakāśāḥ pravartanāni, prādurbhāvā iti yāvat.
te yāvat\footnoteB{
	te yāvat] \emd ; tā yāvat \MS\ \EDD ; deest \emph{in \TIB}
} tāvad bhagavāñ jayati, sarvahariharahiraṇyagarbhādibhyaḥ prakṛṣṭo bhavatīty arthaḥ.

atrānupamasukharūpīty anena svārthasaṃpattiḥ kathitā.
śrīnivāsa ity anena tadupāyaḥ, puṇyajñānasambhārayoḥ śrīśabenābhihitatvāt.
tribhuvanahitasaukhyaprāptikāra ity anena parārthasaṃpattir uktā.
nirupamadaśadevīrūpavidyaḥ savidya ity anena tadupāyaḥ, \footnoteB{
	tathābhūta°] \MS\ \EDD ; \emph{no reflect in } \TIB
}\hspace{0em}tathābhūtadaśadevīdvātriṃśallakṣaṇāśītyanuvyañjanakāyākāraśūnyena\footnoteB{
	°kāyā°] \MS\ \EDD ; dam pa'i sku \TIB\ (satkāya)
} sarvākāraparārthasaṃpatteḥ kartum aśakyatvād iti.

\subsection{prayojanādyabhidhānam}
\begin{quote}
	śrīmantranītigatacārucaturthaseka-\\
	rūpaṃ vidanti na hi ye sphuṭaśabdaśūnyam |\\
	nānopadeśagaṇasaṃkulasaptabhedais\\
	teṣāṃ sphuṭāvagataye kriyate prayatnaḥ || 2 ||
	% GGLGLLLGLLGLGL Vasantatilaka

% 	Employing seven divisions thronging with the teachings of various [teachers], I undertake the present effort to bring about clear understanding to those who indeed do not know the nature, devoid of [expression through] clear words, of the beautiful fourth consecration belonging to the glorious Way of Mantra.
\end{quote}

\noindent śrīmantranītiśabdena sāmānyayogatantravācakenāpi śrīsamājaḥ parigṛhyate, caturthārthakasyānyatrāsambhavāt.
	% svārthe kaḥ?
śeṣaṃ subodham.
nānācāryopadeśagaṇasaṃkulai\hspace{0em}[\EDD\ p.\ 133]\hspace{0em}r vyākulaiḥ saptabhir bhedaiḥ prakārair atītānāgatavartamānācārya\footnoteB{
	°vartamānā°] \EDD ; °pravartamānā° \MS
}gatopadeśarāśisaṃgrāhakaiḥ.
sphuṭāvagataye sukhena sphuṭapratītyartham iti.

\subsection{tīrthikānāṃ tattvasādhyayor prastāvaḥ}
\begin{quote}
	sambhrāntabodhā nikhilā hi tīrthyās \\% final s in tīrthyās is unclear, but probably there; there is a slight crease in the image
	tattvasya sādhyasya ca rūpavittau |\\
	tebhyaḥ prakṛṣṭaḥ kila tattvavettā\\
	vedāntavādīti janapravādaḥ || 3 ||
	% GGLGGLLGLGG, Indravajra

% 	All non-Buddhist philosophers (\emph{tīrthya}) indeed (\emph{hi}) have completely mistaken ideas regarding the realisation of the nature of reality and the goal.
% 	People say that say that the best among them are the Vedāntins, supposedly knowers of reality.
\end{quote}

\noindent sambhrāntetyādi.
sambhrānto vibhrānto bodhaḥ prajñāviśeṣo yeṣāṃ tīrthikānāṃ te tatho[\MS\ fol.\ 3r]ktāḥ.\footnoteB{
	te tathoktāḥ]; \MS\PCreading ; te thoktāḥ \MS\ACreading ; tathoktāḥ \EDD
}
sarva eva tīrthyā ātmātmīyagrahatimiropahatabuddhinayanāḥ.
tattvam idam iti sādhyam idam\footnoteB{
	sādhyam idam] \emd ; sādhyaṃ cedam \MS\ \EDD
} iti ca tattvasya sādhyasya yat\footnoteB{
	yat] \EDD\ (\emd); tat \MS
} svarūpaṃ tasya yā vittiḥ pratītiḥ. tasyāṃ bhrāntāḥ.
śeṣaṃ subodham.

nanu tattvasādhyayor upādeyatvenaikarūpatvāt tattvasya sādhyasya ceti kathaṃ\footnoteB{
	tattvasya sādhyasya ceti kathaṃ] \EDD\ (\emd); tat kathaṃ tatvasya sādhyasya ceti \MS
} bhedena nirdeśa iti cet.
asad etat.
tattvaṃ hy upādeyatve 'pi\footnoteB{
	upādeyatve 'pi] \conj\ (\TIB : blang bar bya ba nyid yin); upādeyatvenāpi \MS\ \EDD
} sukhaduḥkhopekṣādisakalapratibhāsasaṃdohavyāpakam.
sādhyaṃ cānabhimataparihāreṇecchālakṣaṇaṃ phalam upādeyatve 'pi sakalaprāṇibhir avaśyam evāsādhyavyāvṛttyā sādhayitavyatvenābhimatam ity adoṣaḥ.

\subsection{vedāntavādināṃ śrāvakapratyekabuddhānāṃ ca sādhyāni}
tatra tāvad\footnoteB{
	tāvad] \MS\ \EDD\ \TVA\ (re zhig); \emph{no reflex in} \TVB
} vedāntavādyabhimataṃ sādhyam āha\emdash ānandarūpam ityādi.

\begin{quote}
	ānandarūpaṃ svavid aprakampyaṃ \\
	vedāntinaḥ sādhyam uṣanti śāntam\footnoteB{
		śāntam] \corr ; sāntam \MS\ \EDD ; \emph{no reflex in} \TIB
	} |\\
	saśrāvakāḥ\footnoteB{
		saśrāvakāḥ] \emd ; saśrāvakā \MS\ \EDD
	} khaḍgajināś ca sādhyam\\
	icchanti rūpādyupadher virāmam || 4 ||

% 	The Vedāntin long for (√\emph{vaś}) a goal that is self awareness (\emph{svavit})\emdash bliss by nature, unmovable, and at peace. And the rhinoceros-jinas, as well as \emph{śrāvaka}s, desire as a goal the cessation (\emph{virāma}) of the substrates (\emph{upadhi}) of material form (\emph{rūpa}) and so forth.
\end{quote}

\noindent ānandarūpam iti sadāsukhamayatvāt.
svavid iti jyotīrūpatvena\footnoteB{
	jyotīrūpatvena] \MS ; jyotirūpatvena \EDD
} svayaṃ prakāśamānatvāt.\footnoteB{
	prakāśamānatvāt] \EDD\ (\emd); prakāśamānāt \MS
}
aprakampyam iti nityatayā\footnoteB{
	nityatayā] \EDD ; anityatayā \MS\ \TIB\ (mi rtag pa nyid kyis)
} kampayitum aśakyatvāt.
śāntam\footnoteB{
	śāntam] \corr ; sāntam \MS\ \EDD
} iti kleśopakleśaśūnyatvena parikalpitatvāt.
evaṃvidhaṃ sādhyam uṣanti kāmayante.

saha śrāvakair vartante ye khaḍgajināḥ khaḍgaviṣāṇakalpā ekacāriṇo vargacāriṇaś\footnoteB{
	vargacāriṇaś] \MS\ (\emph{cf.\ Abhidharmakośabhāṣya}); vanacāriṇaś \EDD 
} ca pratyekabuddhās te sādhyam icchanti.
kīdṛśam?
rūpādyupadher virāmaṃ rūpavedanāsaṃjñāsaṃskāravijñānalakṣaṇānām upadhīnāṃ skandhānāṃ virāmaṃ vicchedam, nirodham iti yāvat.
[\EDD\ p.\ 134] etad uktaṃ bhavati\emdash sarvaśrāvakapratyekabuddhāḥ sopadhiśeṣanirupadhiśeṣabhedena bhinne 'pi nirvāṇe\footnoteB{
	nirvāṇe] \EDD ; nirvāṇa° \MS
} nirupadhiśeṣam eva nirvāṇaṃ sā[\MS\ fol.\ 3v]kṣātkartavyatvena sādhyaṃ pratipannāḥ.

\subsection{pāramitānayavādināṃ caturvidhaṃ sādhyam}
idānīṃ pāramitānayavādinām abhimataṃ\footnoteB{
	abhimataṃ] \EDD; abhimata \MS
} caturvidhaṃ sādhyam āha\emdash ākāraśūnyam ityādi.
	% citrarūpa -- bahuvrīhi or not?

\begin{quote}
	ākāraśūnyaṃ gaganendurūpaṃ \\
	pratyātmavedyaṃ karuṇārasaṃ ca |\\
	sallakṣaṇair bhūṣitam\footnoteB{
		bhūṣitam] \EDD ; bhuṣitam \MS
	} arthakāri \\
	dānādiniṣyandam apetasaukhyam || 5 ||
	% Indravajra: GGLGGLLGLGG
 
	sānandasallakṣaṇamaṇḍitāṅgaṃ \\
	sambhujyamānaṃ daśabhūmisaṃsthaiḥ |\\
	sattvārthakāri pravadanti sādhyaṃ \\
	dānādiṣaṭpāramitānayasthāḥ || 6 ||
	% GGLGGLLGLGG, Indravajrā

% 	Followers of the Way of the Six Perfections—namely, giving and the others—teach that the goal is the body adorned with the blissful excellent qualities, enjoyed by the lords of the tenth stage and accomplishing the aims of sentient beings.
% 	It is empty of forms, has the nature of the sky and the moon, is to be realised by oneself, and has compassion as its flavour.
% 	Adorned with all excellent characteristics, it accomplishes aims, it is the outflow of giving and so forth, and is free of bliss.
\end{quote}

\subsubsection{pāramitānaye prathaṃ sādhyam}
\noindent ākārair nīlapītasukhaduḥkhādibhiś citrarūpaiḥ śūnyaṃ nirākāram.
ata eva gaganasyeva nirākāratvenendor iva prabhāsvaratvena rūpaṃ svabhāvo yasya tat tathā.
pratyātmavedyam iti svasaṃvedanaikavedyam.\footnoteB{
	svasaṃvedanaikavedyam] \EDD\ (\emd) (°vedyaṃ); svasaṃvedyanaikavedyaṃ \MS
}
karuṇā duḥkhād\footnoteB{
	karuṇā duḥkhād] \MS; karuṇāduḥkhā° \EDD
} duḥkhahetor vā sakalajagadabhyuddharaṇakāmatā.\footnoteB{
	abhyuddharaṇakāmatā] \emd ; °atyuddharaṇakāmatā \MS\ \EDD
}\footnoteA{
	This definition can be found in various older sources, such as the \emph{Pramāṇavārttikavṛtti}.
	Possibly in the Sāramañjarī?
}
saiva rasaḥ svabhāvo yasya tat tathoktam.
etad uktaṃ bhavati\emdash nīlapītādicitrākāraśūnyaṃ nirābhāsaṃ\footnoteB{
	nirābhāsaṃ] \emd ; nirābhāsa \MS\ \EDD
} nirañjanaṃ\footnoteA{
	See also in \emph{Amṛtakaṇika} and \emph{Kāllotara mahātantra} for instances of the pair \emph{nirābhāsaṃ nirañjanaṃ}. One word is probably acceptable as a \emph{viśeṣaṇasamāsa}.
} gaganopamaṃ svacchaṃ sakalajagadarthakāri\footnoteA{
	sakalajagadarthakāri can also be read in compound with mahākaruṇā°. This is reflected in both Tibetan translations: \emph{'gro ba ma lus pa'i don byed pa'i snying rje chen po}
} mahākaruṇāyuktaṃ pratyātmavedyaṃ pāramitopadeśaśabdābhidheyaṃ sādhyam iti pāramitānaye prathamaṃ sādhyam.

\subsubsection{pāramitānaye dvitīyaṃ sādhyam}
\noindent śobhanāni ca tāni lakṣaṇāni ca dvātriṃśallakṣaṇasaṃjñakānīti.\footnoteB{
	°saṃjñakānīti] \conj\ (\textsc{Isaacson}); °saṃjñakāni ceti \MS\ \EDD ; mdzes pa'i mtshan sum cu rtsa gnyis zhes bya ste \TIB
}
tair bhūṣitam.
arthaṃ janānāṃ prayojanaṃ kartuṃ śīlaṃ svabhāvo yasya tad arthakāri.\footnoteB{
	tad arthakāri] \MS\ \EDD ; de ni de'i don mdzad pa'o \TIB\ (tad tadarthakāri)
}
dānādīnāṃ daśapāramitānāṃ niṣyandaṃ tatprakarṣaprabhavatvena sadṛśaṃ phalam.
duḥkhasya pūrvam eva prahīṇatvāt sākṣātkaraṇāvasthāyāṃ\footnoteB{
	sākṣātkaraṇāvasthāyāṃ] \conj\ (\textsc{Isaacson}); sākṣātkṛtāvasthāyāṃ \EDD ; sākṣātkṛtāvatāsthāyāṃ \MS
}\footnoteA{
	\textsc{Isaacson} (personal communication) proposes \emph{sākṣātkaraṇāvasthāyāṃ} or \emph{sākṣātkṛtyāvasthāyāṃ} as potentially supperior readings.

	\hspace*{1em} In support of the former, see \emph{Saṃkṣipābhiṣekavidhi}: \emph{tadanantaram ekatathatāmatena tayaiva bhinnamate tv ānayā svasaṃviditajñānasākṣātkaraṇāvasthāyāṃ pūrvoktagāthayā adhyeṣitavate śiṣyāya tatpāṇau tasyāḥ pāṇiṃ pratisthāpya |}
} saukhyasyāpy abhāvād\footnoteB{
	abhāvāt] \emd\ (\textsc{Isaacson}); abhāvatvāt \MS\ \EDD
} upekṣārūpatvenāpetasaukhyam apagatasaukhyam.
etad uktaṃ bhavati\emdash dvātriṃśallakṣaṇadharāśītyanuvyañjanavirājitaśarīraṃ sakalajagadarthakāri dānādipāramitābhyāsa\crux balenātmānaṃ\footnoteB{
	°balenātmānaṃ] \MS\ \EDD; stobs kyis bdag nyid \TVA; stobs kyis byung ba \TVB
}\crux samyaksaṃbuddharūpaṃ sukhaduḥkharahitatvenopekṣārūpaṃ dvitīyaṃ sādhyam.

\subsubsection{pāramitānaye tṛtīyaṃ sādhyam}
[\EDD\ p.\ 135] sānandetyādi.
sahānandena vartata iti sā[\MS\ fol.\ 4r]nandam.
sānandaṃ ca tat sallakṣaṇamaṇḍitāṅgaṃ ca\footnoteB{
	sallakṣaṇamaṇḍitāṅgaṃ ca] \emph{em.} (\textsc{Isaacson}); sallakṣaṇamaṇḍitāṅgaṃ \MS\ \EDD
} sambhujyamānaṃ dharmadeśanādvāreṇopajīvyamānam.\footnoteB{
	°opajīvyamānam] \MS\ \EDD; nye bar longs spyod par gyur pa'o \TIB\ (°opabhujyamānam)
}
kaiḥ?
daśabhūmīśvaraiḥ, pariśiṣṭabhūmisthitānām\footnoteB{
	pariṣiṣṭabhūmi°] \corr; pariṣiṣṭa bhumi° \EDD
} agocaratvāt.
daśabhūmiprāptair avalokiteśvaramañjuśrīprabhṛtibhir upabhujyamānam iti yāvat.
etad uktaṃ bhavati\emdash śuddhāvāsopari ghanavyūhasaṃjñake\footnoteB{
	°saṃjñake] \emd; °saṃjñako \MS; °saṃjñakaḥ \EDD\ (\emd)
} samyaksaṃbuddhabhuvane yathā bhagavān ānandarūpaḥ sambhogakāyātmā nirmāṇadvāreṇa\footnoteB{
	nirmāṇadvāreṇa] \MS\ \EDD ; sprul pa'i sku'i sgo nas \TIB\ (nirmāṇakāyadvārena)
} sakalajagadarthasampādakaḥ śrāvakapratyekabuddhanavabhūmīśvarair apy adṛśyaśarīro daśabhūmīśvarair eva paraṃ bodhisattvair\footnoteB{
	paraṃ bodhisatvair] \MS\ \EDD\ (°sattvair); mchog tu gyur pa'i byang chub sems dpa' \TIB\ (paramabodhisattvair)
} dharmaśravaṇadvāreṇopabhujyamāna\footnoteB{
	bhujyamāna] \emd ; bhujyamānam \MS\ \EDD
} āsaṃsāraṃ cakāsti, tathaiva tat sādhyam iti tṛtīyam.

\subsubsection{pāramitānaye caturthaṃ sādhyam}

\begin{quote}
	saṃpūrya dānādiguṇān aśeṣān \\
	saṃbuddhakṛtyaṃ\footnoteB{
		saṃbuddhakṛtyaṃ] \emd\ (\emph{cf.} TaRaA-V: saṃbuddhānāṃ \ldots\ avaśyakartavyaṃ kṛtsnaṃ); saṃbuddhya kṛtyaṃ \MS\ \EDD
	} sakalaṃ ca kṛtvā |\\
	yad bhūtakoṭeḥ karaṇaṃ ca sākṣāt \\
	sādhyaṃ tad apy asti nirodharūpam || 7 ||
	% GGLGGLLGLGG, Indravajra
% 
% 	After making replete all the qualities of giving and so forth, and after completing all the tasks of a perfect buddha, there is also a goal, taking the form of cessation, that is the direct perception of the limit of reality.
\end{quote}

\noindent saṃpūryetyādi.
dānādipāramitā eva guṇā, guṇyante 'bhyasyanta iti kṛtvā.
tān saṃpūrya paripūrṇān\footnoteB{
	paripūrṇān] \emd ; paripūrṇaṃ \MS\ \EDD
} kṛtvā, yat saṃbuddhānāṃ kṛtyaṃ sakalam\footnoteB{
	kṛtyaṃ sakalam] \conj ; sakalam \MS\ \EDD	
}\footnoteA{
	The manuscript reading of simply \emph{sakalaṃ} instead of \emph{kṛtyaṃ sakalam} is asymmetrical given the following gloss, \emph{avaśyakartavyaṃ kṛtsnaṃ}. Here Tib.\ reads simply \emph{nges par mdzad par bya ba ma lus pa}, reflecting only the gloss and neither \emph{sakalam} of the Sanskrit nor the conjecture \emph{kṛtyaṃ sakalam}. It is also possible that \emph{sakalam} is a mistaken scribal addition, but it's also possible that even if the Tibetan translators saw \emph{kṛtyaṃ sakalam}, they chose not to render this because of the superfluous sounding result in Tibetan.
} avaśyakartavyaṃ kṛtsnaṃ tad api kṛtvā, bhūtakoṭeḥ śūnyatālakṣaṇāyāś cittacaittanirodhātmikāyā\footnoteB{
	cittacaitta°] \EDD\ (\emd); cittacaitya° \MS
} yat sākṣāt karaṇaṃ tad api sādhyam astīti pāramitānayasthā evaṃ bruvate caturthaṃ sādhyam iti.

\subsection{mantranaye saptavidhaṃ sādhyam}
\subsubsection{mantranaye prathamaṃ sādhyam}
idānīṃ mantranayopadiṣṭaṃ saptavidhaṃ\footnoteB{
	saptavidhaṃ] \EDD\ (Tib: rnam pa bdun); caturthaṃ \MS
} sādhyaṃ kathayitum āha\emdash svābhāṅganetyādi.
\begin{quote}
	svābhāṅganāśleṣi\footnoteB{
		svābhāṅganāśleṣi \EDD\ (\corr); svābhāṅgaṇāśleṣi \MS
	} janārthakāri\footnoteB{
		janārthakāri] \conj\ (Tib: 'gro ba yi don mdzad; TaRaA-V: jagadarthakāri); ta..rthakāri \MS\ (\emph{akṣara uncertain, perhaps} gna \emph{or} mva); tadarthakāri \EDD
	} \\
	duḥkhaiḥ sukhaiś caiva vimuktirūpam |\\
	aśītyanuvyañjanabhūṣitāṅgam \\
	apetakalpaṃ pravadanti sādhyam || 8 ||
	% GGLGGLLGLGL X 2
	% LGLGGLLGLGL X 2
	
% 	They declare that the goal is the body adorned with the eighty minor marks that, while embracing the woman who is one's personal consort and acting for the sake of people, is by nature free of pleasure and pain and devoid of conceptualisation.
\end{quote}

\noindent svābhāṅganām\footnoteB{
	svābhāṅganām] \EDD\ (\corr); svābhāṅganām \MS
} āśleṣituṃ śīlaṃ svabhāvo yasya tat svābhāṅganāśleṣi.\footnoteB{
	svābhāṅganāśleṣi] \corr ; svābhāṅgaṇāśleṣi \MS\ \EDD
}
[\EDD\ p.\ 136] apetakalpaṃ vyapagatakalpam, kalpanārahitam iti yāvat.
anyat subodham.
ayam arthaḥ\emdash samāliṅgitasvābhāṅganāśleṣi jagadarthakāri\footnoteB{
	°svābhāṅganāśleṣi jagadarthakāri] \conj\ (\TVB : nyid dang mtshungs pa'i lha mos 'khyud pa can 'gro ba'i don mdzad pa); °svābhāṅganāśleṣajagadarthakāri \MS\ \EDD; nyid dang mtshungs pa'i lha mos 'khyud pa can | 'gro ba ma lus pa'i don mdzad pa \TVA\ (°svābhāṅganāśleṣy aśeṣajagadarthakāri)
}\footnoteA{
	The compound \emph{°svābhāṅganāśleṣajagadarthakāri} is strinckly speaking not impossible, and could be read as a kind of instrumental \emph{tatpuruṣa}, for example; however, given that this is a prose explanation of the verse, there is no need for the author to use such a compound and it seems mostly likely that the scribe left off the \emph{ikāra}.
} dvātriṃśallakṣaṇavibhūṣitaśarīram\footnoteB{
	śarīram] \EDD ; śarīra \MS
} upekṣārūpaṃ\footnoteB{
	upekṣārūpaṃ] \MS\ \EDD ; btang snyoms kyi ngo bo du 'khor ba ji srid du bzhugs pa (ji srid bzhugs pa] \TVA ; ju bzhugs pa \TVB) mngon du bya ba yin no zhe bya ba TIB (upekṣārūpaṃ āsaṃsārasthāyi sākṣātkriyata iti)
} prathamaṃ sādhyam.

\subsubsection{mantranaye dvitīyaṃ sādhyam}
\begin{quote}
	svadevatākāraviśeṣaśūnyaṃ \\
	prāg eva sambhāvya sukhaṃ sphuṭaṃ sat |\\
	mahāsukhākhyaṃ jagadarthakāri \\
	cintāmaṇiprakhyam uvāca kaścit || 9 ||
	% Metre is viparītākhyānikī:
	% LGLGGLLGLGG
	% GGLGGLLGLGG

% 	A certain [authority] says that [the goal is] bliss devoid of the qualifier of the form of a personal deity. It is meditated on from the very beginning, and when it becomes vivid, it is called Great Bliss, which accomplishes the aim of beings like a wish-fulfilling jewel.
\end{quote}

\noindent svadevatetyādi.
svadevatākāraviśeṣeṇa\footnoteB{
	svadevatā°] \sigmareading{\TVA}; lha \TVA\ (devatā°)
} sveṣṭadevatākāreṇa śūnyam, nirākāram iti yāvat.
prāg eva prathamataram\footnoteB{
	prathamataram] \MS ; prathamataro° \EDD
} upadeśānantaram eva\footnoteB{
	upadeśānantaram eva] \EDD\ (\emd); upadeśāntaram eva \MS ; bshad ma thag pa'i \TIB\ (anantarokta°)
} devatākāranirapekṣaṃ sukhaṃ sambhāvya, bhāvanayā sākṣāt kṛtvā, sphuṭaṃ\footnoteB{
	sphuṭaṃ] \MS ; \emph{deest in} \EDD ; ma gsal ba TIB 
}\footnoteA{
	The understanding reflected in \TIB , namely \emph{asphuṭaṃ} instead of \emph{sphuṭaṃ}, is an alternative word division and also yields sense.
	It seems more likely, however, that the author is glossing \emph{sphuṭaṃ}.
} sphu[\MS\ fol.\ 4v]\hspace{0em}ṭīkṛtaṃ san mahāsukhasaṃjñakaṃ bhavati.
tac ca jagadarthakāri cintāmaṇisamānarūpam.
etad uktaṃ bhavati\emdash upadeśānantaram eva mantramudrādevatākārarahitaṃ\footnoteB{
	°rahitaṃ] \sigmareading{\TVA}; spangs te | bde ba 'ba' zhig tsam \TVA\ (°rahitaṃ sukhamātraṃ)
} bhāvanayā sphuṭīkṛtaṃ mahāsukhasaṃjñakaṃ cintāmaṇivaj jagadarthakāri māyopamam āsaṃsārasthāyi dvitīyaṃ sādhyam.

\subsubsection{mantranaye tṛtīyaṃ sādhyam}
\begin{quote}
	kṛtvā sākṣāt svādhipaṃ [\MS\ fol.\ 1v] sātarūpaṃ \\
	paścāt tyaktvā sātamātraṃ phalaṃ syāt |\\
	śuddhaṃ sākṣāc chakyate naiva kartuṃ \\
	tenākāro bhāvitaḥ svādhipasya || 10 ||
	% Śālinī
	% GGGGGLGGLGG
	% GGGGGLGGLGG
	% GGGGGLGGLGG
	% GGGGGLGGLGL

% 	After directly perceiving one's lord with the nature of bliss, one then abandons it for a fruit that is mere bliss.
% 	The pure cannot be directly perceived; therefore, the form of one's lord is to be meditated on.
\end{quote}

\noindent kṛtvetyādi.
svādhipaṃ sveṣṭadaivataṃ sākṣāt kṛtvāmukhīkṛtya sātarūpaṃ sukhaikasvabhāvam, paścād devatākāraṃ parityajya, sukhamātraṃ\footnoteB{
	sukhamātraṃ] \emd ; sukhamātra° \MS\ \EDD
} phalaṃ sādhyaṃ vyavasthitaṃ syāt.

nanu yadi\footnoteB{
	nanu yadi] \conj ; nanu \MS\ \EDD ; gal te \TVA\ ([nanu] yadi); \TVB : \emph{not clearly rendered}
} sākṣāt kṛtvāpi devatākāras tyaktavyaḥ, tarhi prathamam eva kasmād [\EDD\ p.\ 137] vibhāvitaḥ?
sukhamātram eva dvitīyasādhyavat kiṃ na vibhāvitam?\footnoteB{
	vibhāvitam] \emd ; vibhāvitaḥ \EDD\ (\emd); vibhāgato \MS
}
kiṃ vṛthāprayāsenety\footnoteB{
	vṛthāprayāsenety] \EDD ; vyathāprayāsenety \MS
} āha\emdash śuddham ityādi.
śuddhaṃ kevalaṃ devatākāravirahitaṃ sukhamātraṃ naiva sākṣāt kartuṃ śakyate, ākārarahitasya sukhasyānupalambhāt.
tasmāt tena kāraṇenākāro bhāvitaḥ svādhipasyeti tṛtīyam.\footnoteB{
	tṛtīyam] \emd\ \TVB\ (gsum pa yin no); tṛtīyaḥ \MS\ \EDD ; bsgrub par bya ba gsum pa yin no \TVA\ (tṛtīyaṃ sādhyam)
}
ayam arthaḥ\footnoteB{
	arthaḥ] \EDD ; artha \MS
}\emdash devatākārasaṃvalitam eva sukhaṃ vibhāvya, sākṣādbhūte devatākāraṃ tyaktvā, sukhamātram eva sādhyam uktaguṇam.

\subsubsection{mantranaye caturthaṃ sādhyam}
\begin{quote}
	gagaṇasamaśarīraṃ lakṣaṇair bhūṣitāṅgaṃ \\
	nirupamasukhapūrṇaṃ\footnoteB{
		nirupama°] \EDD ; nirupama° \MS
	} svābhayā saṃgataṃ ca |\\
	sphuradamitamunīndraiḥ\footnoteB{
		munīndraiḥ] \emd ; munīndraḥ \MS\ \EDD
	} sarvasattvārthakāri \\
	pravadati punar anyaḥ sādhyam ucchedaśūnyam || 11 ||
	% Mālinī, LLLLLLGGGLGGLGG
% 
% 	Another says that the goal has a body equal to space, a body adorned with the marks, filled with incomparable bliss, joined with his person consort, accomplishing the aims of all beings with limitless Buddha-sages who shoot forth, free of eradication.
\end{quote}

\noindent gagaṇetyādi.
gagaṇasamaṃ māyopamaṃ vicārāsahaṃ\footnoteB{
	māyopamaṃ vicārāsahaṃ] \MS\ (\emph{reading slightly unclrear}); māyopamavicārasaha \EDD
} śarīraṃ yasya.
lakṣaṇair dvātriṃśadbhir aśītibhiś cānuvyañjanair maṇḍitāny aṅgāni yasya.
nirupamaiḥ sthaulya\footnoteB{
	sthaulya°] \MS\ \EDD ; rgya nam pa nyid dang | rgya che ba nyid dang \TVA\ (praṇītatvasthaulya°); lhun che ba nyid dang | \TVB\ (sthaulya ?)
}nairantaryā\footnoteB{
	°nairantaryā°] \EDD\ (\emd); °nairuttaryā° \MS
}saṃsāra\footnoteB{
	°āsaṃsāra°] \emd ; °āsaṃsāraṃ \EDD\ \MS
}pravāhitvanirāsravatvādibhir upamābhāvād upamātikrāntaiḥ sukhaiḥ pūrṇaṃ romāgraparyantaṃ\footnoteB{
	\conj\ (\TIB : gang ba ni | ba spu rtse mo'i mthar thug pa); pūrṇṇaṃ masimāgrapayantaṃ \MS ; pūrṇatāṃ samāśrayantaṃ \EDD ; \TVA\ (pūrṇaṃ romāgraparyantaṃ)
} saṃpūrṇam.
svābhayā ca tathābhūtayā saṃgataṃ samāliṅgitam.
sphuradbhir anantanirmitair munīndrais tathābhūtair eva sarvasattvārthakāri.\footnoteB{
	sarvasattvārtha°] \MS\ \EDD\ (\TVB : sems can thams cad kyi don); sems can gyi don \TVA\ (sattvārtha°)
}
ucchedeneti nirodhena śūnyam tucchaṃ riktam.\footnoteB{
	tucchaṃ riktaṃ \MS ; bhūsthaṃ riktam \EDD ; spangs pa’o \TIB\ (tucchaṃ  | riktaṃ)
}

% For sthaulya°, see commentary on verse one. These are to be taken as qualities (dharmas) qualifying the sukha and thus making it unique.

etad uktaṃ bhavati\emdash gaganamāyāmarīci\footnoteB{
	māyāmarīci] \MS\ \EDD\ (\TVB : sgyu ma dang | smig rgyu dang |) ; sgyu ma dang | smig rgyu dang | smig rgyu dang | \TVA\ (māyāmarīcīndrajāla  | māyendrajālamarīci)
}\hspace{0em}gandharvanagarodakacandrapratibimbasvapnopamam\footnoteB{
	°svapnopayam] \EDD ; svapnāpayaṃ \MS
} [\MS\ fol.\ 5r] ekānekabhāvābhāvagrāhyagrāhakasvabhāvarahitam anādyantam aśeṣavastusaṃdohasvabhāvam\footnoteB{
	anādyantam aśeṣavastusaṃdohasvabhāvam] \MS\ \EDD; thog ma dang tha ma med pa’i dngos po ma lus pa’i rang bzhin \TVA\ \TVB\ (anādyantāśeṣavastusvabhāvam)
}\footnoteA{
	See parallels in \emph{Samantabhadrasādhana} for \emph{mtshan ma med pa'i dga' ba}
} anābhāsaṃ nirañjanaṃ sarvopamātikrāntaṃ paramasūkṣmātigambhīraprajñārūpatayā dharmakāyasvabhāvam, dvātriṃśallakṣaṇavibhūṣitaśarīram aśītyanuvyañjanavirājitagātraṃ\footnoteB{
	°gātraṃ] \MS\ \EDD ; \emph{deest} in \TVA\ and \TVB
} paramaśṛṅgārayauvanādyupetaṃ svābhāṅganāliṅgitāṅgaṃ rūpavajrāditārāparyantadevīgaṇair anantaprabhedānimittarati\footnoteB{
	°ānimittarati°] \conj\ (\TVA : mtshan ma med pa'i dga' ba'i); °ānimittārati° \MS \EDD ; mtshan ma med pa'i \TVB
}svarūpaparamānandopabhogadvāreṇa pratibimbavat [\EDD\ p.\ 138] sambhujyamānaṃ karuṇāsaṃvalitodārarūpatayā sambhogakāyarūpam, nānādhimuktivineyajanaparipācanārtham % EDD misreports MS as reading paripāvanārtha
anekavidhaprātihāryadvāreṇa\footnoteB{
	anekavidhaprātihārya°] \MS\ \EDD ; rdzu 'phrul dang cho 'phrul rnam pa du ma \TVA\ \TVB\ (anekaṛddhiprātihārya°)
} nirmitānantakulāntarbhūtasaṃbuddhabodhisattvaspharaṇasaṃhārakāritvena\footnoteB{
	°bodhisattva°] \conj\ (\TVB byang chub sems dpa'i); °bodhi° \MS\ \EDD ; byang chub sems dpa' la sogs pa'i \TVA\ (°bodhisattvādi°) 
} nirmāṇakāyātmakam, śūnyatākaruṇābhinnabodhicitta\footnoteB{
	°bodhicitta°] \EDD; °bodhicittā° \MS
}\hspace{0em}svabhāvāmalaprajñopāyasamādhisambhūtasatsukhāpūrṇam\footnoteA{
	See Sahajavilāsa, \emph{Svādhiṣṭhānakurukullāsādhana} (SāMā no.\ 183, p.\ 383): \emph{tataḥ prajñopāyāmalasamādhisambhūtasatsukhāpūrṇam iva svadehaṃ trailokya ca paśyet}.
} āsaṃsārasthitidharmaṃ\footnoteB{
	\conj\ (cf.\ Tib: chos can) ; dharmāṇāṃ \MS\ \EDD	
} apratiṣṭhitanirvāṇarūpaṃ nirmalanivātaniścalapradīpaśikhāprabandhanityatayā nirodhaśūnyaṃ caturthaṃ\footnoteB{
	caturthaṃ] \EDD ; caturtha \MS
} sādhyam.

\subsubsection{mantranaye pañcamaṃ sādhyam}
\begin{quote}
	kṛtvā sākṣāt svādhipaṃ sātarūpaṃ \\
	tyaktvopekṣājñānamātraṃ\footnoteB{
		tyaktvopekṣā°] \MS\ (\emph{\EDD\ reports as \emph{tyajyo°}, but it cannot be; see commentary}); bhāvopekṣā° \EDD\ (\emd); not reflected in \TM
	} phalaṃ syāt |\\
	āsaṃsārasthāyi sattvārthakāri \\
	cintā\footnoteB{
		cintā°] \MS\PCreading\ \EDD ; cittā° \MS\ACreading
	}ratnaprakhyam\footnoteB{
		°prakhyam] \EDD ; °prakhyaṃm \MS
	} ekāntaśāntam || 12 ||
	% Śālinī, GGGGGLGGLGG X 4
	
% 	After one directly perceives and then relinquishes one's lord with his blissful nature, there comes as fruit mere indifferent awanreness.
% 	It remains as long as there is \emph{saṃsāra} and accomplishes the aims of beings.
% 	Known as the wish-fulfilling jewel, it is entirely at peace.
\end{quote}

\noindent kṛtvetyādi.
sākṣāt svādhipaṃ kṛtvā, paścāt\footnoteB{
	paścāt] \EDD ; paścāta \MS
} tyaktvā, upekṣārūpaṃ yaj jñānaṃ tanmātraṃ sādhyaṃ syāt.
anyat sugamam.\footnoteB{
	sugamaṃ] \EDD ; sūgamaṃ \MS
}
etad uktaṃ bhavati\emdash maṇḍalacakrarūpaṃ sākṣāt kṛtvā, paścāt tan nirodhya, upekṣājñānamātraṃ sādhyaṃ syāt pañcamam.

\subsubsection{mantranaye ṣaṣṭhamaṃ sādhyam}
\begin{quote}
	kṛtvā sākṣān maṇḍalaṃ sātarūpaṃ \\
	paścāt tasya svecchayā nirvṛtiś\footnoteB{
		nirvṛtiś] \MS ; nirvṛtiṃ] \EDD 
	} ca|\\
	sattvārthasyāpy asty abhāvo na vāsmin \\
	prādurbhāvo nirvṛtād\footnoteB{
		nirvṛtād] \EDD ; nivṛtād \MS
	} asti yasmāt || 13 ||
	% Śālinī, GGGGGLGGLGG

% 	After directly perceiving the \emph{maṇḍala} whose nature is pleasure, there is later its cessation based on one's desire.
% 	Neither is there here an absence of [accomplishing] the aims of beings, since from cessation there is manifestation.
\end{quote}

\noindent kṛtvetyādi.
kṛtvā sākṣān maṇḍalaṃ sātasaṃvalitam.\footnoteB{
	sātasaṃvalitam] \emd\ (\TIB : bde ba'i rang bzhin can); sātaṃ saṃvalitaṃ \MS\ \EDD
}
tasya svecchayā nirvṛtir nirodhaḥ.

nanu yadi sākṣāt kṛtvāpi paścāt svecchayā nirodhayita[\MS\ fol.\ 5v]vyam,\footnoteB{
	nirodhayitavyam] \emd ; nirodhayitavyaḥ \MS\ \EDD
} tadā karuṇāyā anekakālābhyastāyā abhāvaḥ syāt.
tasyāś cābhāvāt sattvārthābhāvaḥ [\EDD\ p.\ 139] syād ity āśaṅkyāha\emdash sattvārthasyāpy asty abhāvo na vetyādi.
asmin pakṣe sattvārthābhāvo nāsti, yasmān nirvṛtāc cakrāt karuṇāsaṃvalitāt sattvārthasya prādurbhāvo 'sti.\footnoteA{
	\TIB\ suggests reading \emph{karuṇāsaṃvalitasya}: ’gags pa’i ’khor lo las snying rje’i rang bzhin can sems can gyi don (’gags pa’i] \TVB ; ’gog pa’i \TVA)
}

etenaitad evāha\emdash sātasaṃpūrṇacakraṃ sākṣāt kṛtvā, yāvad iṣṭaṃ kālaṃ vyavasthāpya, paścāt tasya sarvathaiva pradīpavan nirodhaṃ kṛtvā sthātavyam.
yadā punaḥ sattvārthābhilāṣo bhavati, tadā niruddhād eva cakrāntaram utpādya sattvārthaḥ kartavyaḥ.
cakrāntarotpāde\footnoteB{
	cakrāntarotpāde] \EDD ; cakrāntaropāde \MS
} 'pi ciraniruddhād\footnoteB{
	ciraniruddhād] \emd (\TIB : rin du 'gags pa'i); citaniruddhād \MS ; cittaniruddhād \EDD
} eva cakrād yathābhavyatayā\footnoteB{
	yathābhavyatayā] \emph{variant word division in} \EDD : yathā bhavyatayā
} vineyānāṃ yathābhilaṣitaprāptir bhavatīti ṣaṣṭham.

\subsubsection{mantranaye saptamaṃ sādhyam}
\begin{quote}
	kṛtvā sphuṭaṃ rūpam abhīṣṭam eṣāṃ \\
	paścān nirodhaḥ\footnoteB{
		nirodhaḥ] \emd ; nirodha(ṃ) \MS\ (\emph{this may be corrected to ḥ}); nirodhaṃ \EDD
	}\footnoteA{
		It is possible to take \emph{phala} as the direct object of √\emph{ah} and then read \emph{nirodhaṃ}, construing it as an accusative form; however, the agent of \emph{√kṛ} and \emph{√ah} would have to be the same.
		Rather, with the reading \emph{nirodhaḥ phalaṃ}, we can avoid this problem and simply supply an \emph{iti}.
	} phalam āha kaścit |\\
	abhinnarūpaś ca yato nirodho \\
	na pakṣabhede 'pi tato 'sti bhedaḥ || 14 ||
	% Rāmā, GGLGGLLGLGG X2, LGLGGLLGLGG X2
 
% 	Some say that after realising directly the desire form of these [other six positions], [one later realises] the fruit which is cessation.
% 	And because cessation is not different in nature, even though there are different position, it has no distinction.
\end{quote}

\noindent kṛtvetyādi. ṣaṇṇāṃ pakṣāṇām anyatamasya phalasya\footnoteB{
	anyatamasya phalasya] \conj ; arthaphalaysa \MS\ \EDD ; nang nas 'bras bu \TIB
} sādhyatvād yad yad evābhiṣṭaṃ\footnoteB{
	phalasya sādhyatvād yad yad evābhiṣṭaṃ] \MS\ \EDD ; 'bras bu bsgrub bya gang kho na \TVA\ (phalaṃ yad eva); bsgrub bya gang kho na mngon par 'dod pa \TVB\ (phalaṃ yad evābhiṣṭaṃ);
} tad\footnoteB{
	tad] \EDD ; sa \MS
} eva sākṣāt kṛtvā, paścāt sarvathaiva pradīpavan nirodha uttarakālaṃ sattvārthādiśūnyaḥ sākṣāt kartavyaḥ.

nanu ṣaṭpakṣabhedena ṣaḍ eva\footnoteB{
	ṣaḍ eva] \EDD ; ṣatreva \MS
} nirodhāḥ syuḥ. tat katham eka eva nirodha ity āśaṅkyāha\emdash abhinnetyādi. abhinnaṃ\footnoteB{
	abhinnaṃ] \EDD ; abhinna \MS
} rūpaṃ yasya sa tathā.\footnoteB{
	sa tathā] \emd ; tat tathā \MS\ \EDD
} na hi nirodhānāṃ ṣaṭpakṣalakṣaṇabhede 'pi bhedo 'sti, abhāvaikarūpatayā nirodhasya samānatvāt. ayam arthaḥ\emdash anyatamapakṣaṃ sākṣāt kṛtvā paścāt tasya santānocchedarūpo nirodha iti saptamaṃ sādhyam.

\subsection{caturthe 'bhiṣekase vipratipattiḥ}
\subsubsection{caturthaseke vipratipattiḥ prathamā}
\begin{quote}
	prajñājñānād uttaraṃ bodhicittā-\\
	svādas turyaṃ sekam\footnoteB{
		sekam] \EDD ; seṣam \MS
	} āhāvaraṃ tat |\\
	yasmāt\footnoteB{
		yasmāt] \EDD ; paścāt \MS
	} sarvo bhāvanāsu prayāso \\
	vyarthaḥ prāptas tatphalasya prasiddheḥ || 15 ||

% 	[Some] say that the fourth consecration is tasting \emph{bodhicitta} subsequent to the wisdom-knowledge [consecration]. This is bad. For, given that its fruit is already established. all efforts in the meditations, having already obtained, would pointless. 
\end{quote}

\noindent [\EDD\ p.\ 140] prajñājñānetyādi. prajñājñānopadeśād uttarakālaṃ\footnoteB{
	prajñājñānopadeśād uttarakālaṃ] \MS\ \EDD ; shes rab dang ye shes ni shes rab ye shes te | dbang bskur ba'i bye brag go || phyis ni 'das pa'i 'og tu'o || gang zhe na | \TVA\ (prajñājñānetyādi. prajñā ca jñānaṃ prajñājñānaṃ sekaviśeṣaḥ. uttaram paścāt. kim?); \TVB
} yat bodhicittasyāmṛtarūpasya\footnoteB{
	bodhicittasyāmṛtarūpasya] \emd\ (\TVA : byang chub kyi sems te); saṃ bodhicittasyāmṛtarūpasya \MS\ \EDD ; sems te \TVB\ (cittasya)
} rasanayā grahaṇam, tat turyaṃ caturthaṃ [\MS\ fol.\ 6r] sekam āha kaścit.
tac cāvaraṃ hīnam, vinikṛṣṭam iti yāvat.
kasmād avaram?
yasmāt sarvaprayāso mantramudrādevatādyākārabhāvanāsu punaḥ punar anuṣṭhānalakṣaṇas tathāgatokto\footnoteB{
	tathāgatokto] \MS ; tathāgatoktau \EDD
} vyarthaḥ prāptaḥ.
kutaḥ?
tatphalasya bhāvanāsādhyasya phalasya bodhicittāsvādakāla eva prasiddhatvāt prāptatvāt, anyasya viśiṣṭasya phalasyābhāvād iti yāvat.

\subsubsection{caturthaseke vipratipattir dvitīyā}
\begin{quote}
	prajñājñānād uttaraṃ prāptarāmā-\\
	svādas turyaṃ sekam āhādhamaṃ tat |\\
	yasmāt sarvo bhāvanādau prayatno \\
	buddhoddiṣṭo niṣphalaḥ saṃprasaktaḥ || 16 ||

% 	[Some] say that the fourth initiation is the enjoyment of women who are met with after the wisdow-knowledge [initiation].
% 	This is most reprehensible.
% 	For all of the effort taught by the Buddha in meditation and the like would be useless, it follows(?).
% GGGGGLGGLGG
\end{quote}

\noindent prajñetyādi.
prajñājñānād uttarakālaṃ yāḥ prāptā yathāmilitā rāmāḥ striyas tāsāṃ samāpattidvāreṇa\footnoteB{
	samāpattidvāreṇa] \EDD ; rig pa'i sgo nas \TVA ; reg pa'i sgo nas \TVB\ (sparṣadvāreṇa)
} ya āsvādaḥ, tat turyaṃ sekam.
tad apy adhamam.
śeṣaṃ gatārtham.

\subsubsection{āgamasya arthavyākhyānam}
atha caturthaṃ tat punas tatheti\footnoteB{
	punas tatheti] \EDD\ (\emd); punar iti \MS
}\footnoteA{
	\emph{Samājottara} 112c
} vyākhyāyate. caturtham iti\footnoteB{
	caturtham iti] \MS\ \EDD\ \TVA\ (bzhi pa ni); deest \emph{in} \TVB
} prajñājñānaṃ tṛtīyam apekṣya caturtham ity ucyate.
tad iti tacchabdena tad eva prajñājñānaṃ tadrūpaṃ parāmṛśyate. punar iti punaḥśabdena tasmād viśeṣaḥ. viśeṣaś cātra nirāsravaniruttarātyantasphītāvicchinnaprabandha\footnoteB{
	°niruttarātyantasphītāvicchinnaprabandha°] \MS\ \EDD ; shin tu rgyas pa nyid rgyun mi chad par \TVB\ (°ātyantasphītāvichinnaprabandha°); nirantarātyantasphītāvicchinnaprabandha° \EDD\ (\emd); shin tu rgyas pa nyid dang | bar chad med pa nyid dang | rgyun mi 'chad par (°ātyantasphītāvicchinnaprabandhanirantara)
}pravāhitvalakṣaṇaḥ.\footnoteB{
	°lakṣaṇaḥ] \EDD ; °lakṣaṇaṃ \MS
} tatheti tathāśabdena tādṛśatvam abhidhīyate. tādṛśatvaṃ ca yādṛśyā prajñādiyuktyā\footnoteB{
	°yuktayā] \conj\ (\TIB : dang ldan pa'i); °yuktyā \MS\ \EDD
} sāmagryā yādṛśaṃ prajñājñānam utpannam, paścād api tādṛśyaiva sāmagryā tathaiva cotpadyate nānyatheti tathāśabdārthaḥ.

atra ca lakṣyalakṣaṇabhāvenārtho boddhavyaḥ. lakṣyate 'neneti lakṣaṇam anubhūyamānaṃ prajñājñānam, apratīyamānasya lakṣaṇatvāyogāt, nāgṛhītaviśeṣaṇā\hspace{0em}[\EDD\ p.\ 141]viśeṣyabuddhir iti nyāyāt. lakṣyate jñāyate pratipādyate 'neneti lakṣyaṃ sākṣāt kariṣyamāṇaṃ caturtham.

\subsubsection{caruthaseke vipratipattis tṛtīyā}
atra caturthaṃ nāstīty eke.\footnoteA{
	\TVA\ adds near the beginning of this sentence \emph{Samājottara} 112ab \emph{abhiṣekaṃ tridhā bhedam asmin tantre prakalpitam |}: \emph{'dir 'ga' zhig | dbang ni rnam pa gsum dag tu | rgyud 'di las ni rab tu grags || zhes gsungs pas na | bzhi pa ni yang dag pa ma yin no zhe na |}
} nanu caturtham ity etad asti tatpadam.\footnoteB{
	nanu caturtham ity etad asti tatpadam] \MS\ (nanu caturtham ity etad asti | tat padan) \EDD ; de ltar de bzhin bzhi pa yang || zhes bya ba'i tshig bcom ldan 'das kyis gsungs pa yod pa ma yin nam | \TVA\ (caturthaṃ tat punas tatheti padaṃ bhagavatā notkaṃ vā); de lta na de ma yin pa gzhan de ltar de bzhin bzhi pa yang zhes bya ba der bzhi pa zhes bya ba'i tshig bcom ldan 'das kyis gsungs pa yod pa ma yin nam | \TVB\ (nanu yadi evaṃ na syāt, tadā caruthaṃ tat punas tatheti padaṃ bhagavatā noktaṃ vā)
} tat kathaṃ nāstīty ucyate? satyam, upadeśasaṃrakṣārthaṃ sattvavyāmohanāya ca tṛtīyam eva caturthaśabde[\MS\ fol.\ 6v]noktaṃ bhagavatā. anyathā tat punar iti noktaṃ syāt.\footnoteA{
	A portion seems to have dropped out from \TVA . 
}

tad atyantāsaṃgatam, caturthasya pramāṇasiddhasya pratipāditatvāt pratipādayiṣyamāṇatvāc ceti.\footnoteA{
	Tib.\ discusses two further \emph{pakṣa}s here: that the fourth referred to in the \emph{Samājottara} is the four \emph{aṅga} of \emph{sevā} and so forth; and what appears to be the idea that the four initiation consists in the third accompanied by its fruits (\emph{'bras bu dang bcas pa}).
}

\subsubsection{lakṣyasya vicāraṇam}
atra lakṣaṇaṃ prajñājñānaṃ pratītam eva sarvaiḥ. lakṣye\footnoteB{
	lakṣye] \EDD\ (\emd); lakṣyā \MS
} paraṃ vyāmohaḥ. tad vicāryate. lakṣyaṃ hi bhaved\footnoteB{
	lakṣyaṃ hi bhavet] \conj (\TIB : mtshon par bya ba yang srid na); lakṣyaṃ hi bhagavat \MS\ \EDD\ (°gavad)
} artharūpaṃ vā syāt jñānarūpaṃ vā. na tāvad artharūpam, arthasyaikasyābhāvāt, ekānekaviyogitvena pramāṇena tasya nirākṛtatvāt. mantranaye ca vijñānavādamadhyamakamatayor\footnoteB{
	matayor] \EDD ; tamayor \MS
} eva pradhānatvād jñānarūpaṃ vā syāt. jñānaṃ ca sākāraṃ vā nirākāraṃ vā. sākāram api citrādvaitarūpaṃ vā syād anekarūpaṃ vā syād iti vikalpāḥ.

\subsubsubsection{sākārasya vijñānasya nirākaraṇam}
tatra sākāravijñānaṃ sarvathaiva gagaṇakamalavan nāstīti nirākāravādino bruvate. nanu nīlapītaśuklādighaṭapaṭaśakaṭādi\footnoteB{
	°śakaṭādi°] \EDD\ (\emd); °prakaṭādi° \MS
}rūpeṇākārāḥ\footnoteB{
	°ākārāḥ] \conj; ((cā))kārāḥ] \MS ; vākārāḥ \EDD
} pratibhāsante\footnoteB{
	pratibhāsante] \EDD ; pratibhāṣante \MS
} pratyakṣataḥ.\footnoteA{
	\TIB\ phrases this sentence as a rhetorical question, as if the Sanskrit started \emph{kiṃ na} \ldots .
} te cārthasyābhāvād jñānarūpā eva. tat kathaṃ sākāraṃ nāstīti?\footnoteA{
	\TVA 's expression of the argument runs differently: \emph{don (rnam pa) de dag kyang med pa'i phyir shes pa'i ngo bo nyid kyang med yin na | de ji ltar rnam pa dang bcas pa ma yin zhe na |} `Because those objects [i.e., \emph{ākāra}s] also do not exist, the nature of cognition too cannot exist. So how can cognition not have \emph{ākāra}s?'
} satyam. pratibhāsanta evākārāḥ, param alīkarūpeṇa. alīkarūpatā caikānekaviyogitvena\footnoteB{
	°viyogitvena] \conj\ °viyogitva° \MS\ \EDD
} pramāṇalakṣaṇena\footnoteB{
	°pramāṇalakṣaṇena] \MS\ \EDD\ (\TVB : tshad ma'i mtshan nyid kyis); mtshan nyid kyis \TVB\ (°lakṣaṇena)
} prasiddhā. tasya ca pramāṇasvarūpasyānyatra\footnoteB{
	pramāṇasvarūpasyā°] \EDD ; pramāṇa(((pe)))rūpasyā° \MS
} kathitatvān neha\footnoteB{
	neha] \EDD ; eha \MS
} pratanyate. alīkatvaprasiddhā ca māyāmayā ivākārā bhrāntirūpāḥ prakāśante.\footnoteB{
	prakāśante] \MS\ (prakāsante) ; prakāśyante \EDD
} bhrāntinivṛttau ca nirākāram eva\footnoteB{
	nirākāram eva] \MS\ \EDD\ \TVB\ (rnam pa med pa kho na); rnam pa med pa de kho na \TVB\ (nirākāram eva tad)
} śuddhasphaṭikasaṃkāśaṃ pāramārthikaṃ\footnoteB{
	pāramārthikaṃ] \EDD\ (\emd); pārarthikaṃ \MS
} siddhaṃ bhavati.\footnoteB{
	bhavati] \MS ; bhavatīti \EDD
} ataś citrādvaitarūpam anekarūpaṃ ca sākāraṃ vijñānam astīti vikalpadvayaṃ nirastaṃ bhavatīti.

\subsubsubsection{nirākārasya vijñānasya samarthanam}
nanu nirākāram api vijñānam upalabdhilakṣaṇaprāptaṃ svapne 'pi nopalabhyate. tat kathaṃ tad asti paramārthata\footnoteB{
	paramārthata] \emd ; paramārtham \MS\ \EDD
} i[\MS\ fol.\ 7r]ty ucyate? ucyate. sukhākāraṃ vijñānam antaḥparisphuradrūpaṃ nirākāraṃ saṃvedyata eva. nīlādyākārāḥ punar alīkāḥ pratibhāsante. % annotated ms. circles kā in nīlādyākārāḥ, but other 
anyathā teṣāṃ satyatve sarva evākārāḥ satyāḥ syuḥ. tathā hi grāhyagrāhakabhāvādikam api satyaṃ [\EDD\ p.\ 142] syāt. tataś ca sarveṣām eva satyapratibhāsatvena muktiprasaṅgaḥ,\footnoteB{
	muktiprasaṅgaḥ] \conj ; yuktiprasaṅgāt \MS ; muktiprasaṅgāt \EDD\ (\emd)
} keṣāñcid api mithyāpratibhāsasya bhrāntirūpasyāpratibhāsanāt. tathā coktam—

\begin{quote}
	draṣṭavyaṃ\footnoteB{
		draṣṭavyaṃ] \EDD ; draṣṭavya \MS
	} bhūtato bhūtaṃ bhūtadarśī vimucyate |\footnoteA{
		\emph{Abhisamayālaṅkāra} 5.21; \emph{Ratnagotravighāba} 154; \emph{Pratītyasamputpādahṛdayakārikā} 7; etc.
	}
\end{quote}

\noindent tasmād akāmakenāpi nīlādyākārāṇām alīkatvam evaiṣṭavyam. sukhādikaṃ nirākāraṃ\footnoteB{
	nirākāraṃ] \MS\ \EDD ; rnam pa brdzun pa \TIB\ (alīkākāraṃ)
} satyam upalabhyate. tat kathaṃ nopalabhyata iti.

nanu sukhādyākāram sākāraṃ eva vijñānam\footnoteB{
	sākāraṃ eva vijñānam] \conj (\TIB : rnam pa dang bcas pa'i kho na shes pa); eva vijñānam \MS\ \EDD
} upalabhyate, sukhāder ākārasvabhāvatvāt. na ca sukhādyākāraśūnyaṃ jñānaṃ svapne 'pi saṃvedyate. sakalabhrāntivigamād aṣṭamyāṃ bhūmāv upalabdhilakṣaṇaprāptir bhavatīty atrāpi kośapānaṃ\footnoteB{
	kośapānaṃ] \MS\ (kosapānaṃ); śapathollaṅghanaṃ \EDD\ (\emd)
} vinā anyan na\footnoteB{
	anyan na] \EDD ; anyatra \MS
} pramāṇam asti prasādhakam iti. tad asat,\footnoteB{
	tad asat] \conj\ (\TIB : de ni bden pa ma yin te); tad \MS\ \EDD ; asad etat \possibleconj
} abhiprāyāparijñānāt, sukhādyākārasyaiva nīlādyākārarahitasya vijñānasya nirākāratveneṣṭatvāt. tac cedānīm eva svasaṃvedanapramāṇasiddhaṃ sakalaprāṇabhṛtam\footnoteB{
	°bhṛtam] \emd ; °bhṛtām \MS\ \EDD
} astīti kathaṃ nopalabdhiḥ?

\subsubsubsection{Establishing the Madhyamaka position}
nanu tad\footnoteB{
	nanu tad \MS\ \EDD ; tat \possibleconj
} apy ekānekasvabhāvaviyogād alīkam eva bhrāntimātram, ekānekasvabhāvarahitasya sākāranirākāravijñānavyāpitvāt.

nanv anena nyāyena sakalasākāranirākāravijñānasyālīkatvaprasādhanān na kiñcid api pāramārthikaṃ vastutattvam asti.\footnoteB{
	asti] \conj ; astīti \MS\ \EDD\ (astīti?) (\emph{iti} has no reflex in \TIB)
} tat kathaṃ lakṣyasya svarūpaṃ pramāṇata upalakṣayitavyam? naiṣa doṣaḥ, madhyamakamate pramāṇato 'līkatāsiddhāv api māyopamapratibhāsamātrasyaikānekasvabhāvarahitasya dharmirūpasyāpratiṣedhāt. tatraiva cālīke pratibhāsamātre lakṣyalakṣaṇasaṃsāranirvāṇa[\MS\ fol.\ 7v]maṇḍalacakrādibhāvanāsakalajagadarthakriyādīnām\footnoteB{
	°bhāvanā°] \MS ; °bhāvanā \EDD\ (variant word division); bsgoms pas \TIB\ (bhāvanayā)
} avyāhatā vyavasthā\footnoteB{
	vyavasthā] \MS ; vyavasthā ca \EDD\ (\emd)
} sidhyati.\footnoteB{
	sidhyati] \conj ; sidhyatīti \MS\ \EDD\ (\emph{no reflext of} iti in \TIB)
}\footnoteA{
	\EDD\ appears to understand the text as saying that both \emph{bhāvanā} and \emph{jagadarthakriyādīnāṃ vyavasthā} are established.
	\TIB\ suggests that it is \emph{bhāvanā} which is the instrument by which the \emph{vyāvasthā} is established.
	The manuscript reading suggests taking \emph{°bhāvanā} in compound with the following word—i.e., in the Madhyamaka system, although mere appearance is false, the framework of everything starting with \emph{lakṣyalakṣaṇa} is established.
} tathā coktam\emdash 

\begin{quote}
	buddhatvaṃ vajrasattvatvaṃ saṃvṛtyaiva prasādhayet |\footnoteA{
		\emph{Kurukullākalpa} 3.16cd
	}
% 	\textbf{\TVA}\\
% 	sangs rgyas rdo rje sems dpa' nyid ||\\
% 	kun rdzob nyid du rab tu grub ||
% 
% 	\textbf{\TVB}\\
% 	sangs rgyas dang rdo rje sems dpa' nyid kyang\\
% 	kun rdzob nyid du rab tu bsgrub
\end{quote}

\noindent iti.\footnoteB{
	iti] \EDD ; deest \emph{in} \MS
}

nanu sarvam eva vastujātam alīkarūpatayā niḥsāram. tadā kimarthaṃ maṇḍalacakrādibhāvanāprayāsaḥ\footnoteB{
	maṇḍala°] \EDD ; bri ba'i 'dkyil 'khor (lekhyamaṇḍala°)
} kriyate? asad etat,

\begin{quote}
	mithyādhyāropahānārthaṃ\footnoteB{
		mithyādhyāropahānārthaṃ] \emd ; mithyādhyāropaṇārthaṃ \MS\ \EDD
	} yatno 'saty api\footnoteB{
		'saty api] \MS ; 'styopi \EDD
	} [\EDD\ p.\ 143] bhoktari~|\footnoteB{
		bhoktari] \MS\ (bhoktarī°) (\emph{the letter} no \emph{is added abhove} bho); muktaye \EDD\ (\emd)
	}\footnoteA{
		\emph{Pramāṇavārttika}, Pramāṇasiddhi 193cd.
	}
\end{quote}

\noindent iti vacanāt. yady api vicāryamāṇaṃ pāramārthikaṃ vasturūpaṃ nāsti, tathāpy ahaṃ sukhī bhaveyaṃ mā\footnoteB{
	mā] \EDD\ (\emd); deest \emph{in} \MS
} duḥkhy abhūvam iti tṛṣṇā sakalaprāṇabhṛtām asti. yathā tulye 'pi mithyātve śubhāśubhasvapnayoḥ śubhasvapnadarśanāt saumanasyam aśubhasvapnadarśanāc ca daurmanasyam, tadapanayanāya ca saddharmapāṭhamantrajāpādau pravṛttir bhavati, tathā mithyātvāviśeṣe 'pi duḥkhādiprākṛtavikalpahānāya\footnoteA{
	cf.\ \emph{Samantabhadrasādhana} (as quoted in Kamalanātha's \emph{Ratnāvalī} ad HeTa 2.2.45, fol.\ 16r6): prākṛtavikalpavṛttair aparaṃ na hi kiñcad asti bhavaduḥkham | tasya viruddhaṃ caitat sākṣādavagamyate cetaḥ ||
} samyaksaṃbodhilakṣaṇaprāptaye\footnoteB{
	lakṣaṇaprāptaye] \MS\ \EDD ; mtshan nyid kyi 'bras bu thob par bya ba'i phyir \TVA\ (lakṣaṇaphalaprāptaye); mtshan nyid 'bras bu thob par bya ba'i phyir \TVB\ (lakṣaṇaphalaprāptaye)
} ca prekṣāvatām arthināṃ pravṛttir bhaviṣyatīti.

\subsection{saptavidheṣu sādhyeṣu sārāsāravicāraṇam}
nanu yadarthas tavāyam\footnoteB{
	yadarthas tavāyam] \conj ; yadarthasvā'yaṃ \MS ; yadarthatvād ayaṃ \EDD 
}\footnoteA{
	The manuscript's reading \emph{yadarthasvā'yaṃ} seems like a plausible corruption of \emph{yadarthas tavā'yaṃ}, but Tibetan shows no reflex of \emph{tava}. \TVA\ reads: \emph{rtsom pa 'di'i don gang yin pa}. \TVB\ reads: \emph{gal te gang gi don du (bzhi pa bshad pa'i bshad pa'i dus) 'di brtsams pa'i}.
} ārambhaḥ so 'rthaḥ pralayaṃ gataḥ. tathā hi lakṣyalakṣaṇacintātra prastutā. sā ca vismṛtā, kva gateti na jñāyate.

na tu\footnoteB{
	na tu] \conj ; nanu \MS\ \EDD
} kṛtaiva sā saptabhir bhedaiḥ?

satyam, kintu guḍagorasanyāyena. tathā hi na jñāyate, kiṃ tat sāram asāraṃ veti.

ucyate.

\subsubsection{prathamasya asāratvam}
mantranayavihitakramābhāvāt samāpattibhāvanāvaiyarthyād\footnoteB{
	samāpatti°] \MS\ \EDD\ \TVB\ (snyom par 'jug pa); lha'i rnal 'byor gyi snyoms par 'jug pa'i \TVA\ (devatāyogasamāpatti°)
} yuktyabhāvāc\footnoteB{
	yuktyabhāvāc] \EDD ; yuktābhāvāc \MS
} ca prathamasya niḥsāratā. tathā hi samagrasāmagrīkaṃ yat phalaṃ\footnoteB{
	yat phalaṃ] \conj\ (\TIB : 'bras bu gang yin pa); yat \MS \EDD
} tad avaśyam eva bhavati. anyathā samagrasāmagrīkam eva tan na bhavet. sākṣātkaraṇāvasthāyāṃ samagrasāmagrīkaṃ tad vartate. tad avaśyaṃ tena\footnoteB{
	tena \MS\ \EDD\ \TVB\ (de); de'i 'bras bu \TVA\ (tena phalena)
} bhavitavyam. sati ca bhavati\footnoteB{
	\conj ; bhavane na \MS\ \EDD ; de ltar gyur pas dang po nyams pa yin no \TVA ; de ltar gyur pa dang po nyams pa yin no \TVB\ (evaṃsati )
}\footnoteA{
	\TIB\ could be rendered as something like \emph{evaṃsati ca prathamasya hānir iti}. Indeed this seems to be the sense, but the manuscript reading of \emph{bhavane na} or \emph{bhavanena} is hard to account for.
} prathamasya hānir iti.

\subsubsection{dvitīyasya asāratvam}
śarīrādyākāraśūnyasya kevalasātarūpasyānupalabdher\footnoteB{
	°labdher] \EDD ; °bdher \MS
} na dvitīyasya sāratā. tathā hi pramāṇaniścitaṃ prekṣāvatā bhāvanīyam, na yathākathañcit. pramā[\MS\ fol.\ 8r]ṇena saṃvalitarūpam eva sarvadopalabhyate.\footnoteA{
	\TVA\ lacks a reflex of \emph{sarvadā}, whereas \TVB\ lacks a reflect of \emph{eva}.
} tad eva sarvajanānāṃ kamanīyatayā pratibhāsate. tasmāt kevalasya rucyabhāvāc cakrākārasaṃvalitasyopalabdheḥ sākṣāt kartum aśakyatvāc\footnoteB{
	aśakyatvāc] \EDD\ (\emd); aśakyatāc \MS
}\footnoteA{
	\TIB\ suggests reading: \emph{kevalasyānupalabdheḥ rucyabhāvāc cakrākārasaṃvalitasyānupalabdheḥ sākṣātkartum aśakyatvāc ca}.
	The addition of \emph{anupalabdheḥ} after \emph{kevalasya} renders the flow of argument's logic less smooth. The addition of the same word after \emph{cakrākārasaṃvalitasya} does not change the argument in its overall sense.
	Note that the Tibetan \emph{'khor lo'i rang bzhin} could instead imply the reading \emph{cakrasvabhāva} or something similar, but it may simply be a free rendering of \emph{cakrākārasaṃvalita}.

	\TIB\ also adds the reason \emph{'bad pa mtshung pa'i phyir} (`becaue the effort is equal'). The purport of this is unclear.
} ca dvitīyasya kalpanāmātrateti.\footnoteB{
	kalpanāmātrateti] \EDD\ (\emd); kalpanātrateti \MS
}

\subsubsection{tṛtīyasya asāratvam}
nirupadravabhūtārthasvabhāvatvena sātmībhūtasya tyaktum aśakyatvāt, saṃvalitarūpasya [\EDD\ p.\ 144] bhedābhāvāt, prayojanābhāvāc ca na tṛtīyasya\footnoteB{
	tṛtīyasya] \conj ; tṛtīya \MS ; tṛtīyaḥ \EDD
} kalyāṇabhāvaḥ.\footnoteB{
	] \conj\ (Tib: dge ba [ma] yin); kalyanībhāvaḥ \MS\PCreading ; kalyānībhāvaḥ \MS\ACreading ; dge ba ma yin \emph{[na] kalyāṇabhāvaḥ}
} tathā hi sahopalambhena\footnoteB{
	sahopalambhena] \EDD ; saholaṃbhena \MS
} tādātmyasiddhāv ekasya parityāge 'parasyāvaśyaṃ parityāgo na vā kasyacid iti. 

\subsubsection{caturthasya sārāsāratvavicāraṇam}
prapañcatvena bahuprayāsatvād vicārāsahatvena bhrāntirūpatayāparamārtharūpatayā ca na tṛtīyāntapakṣasya\footnoteB{
	tṛtīyāntapakṣasya] \emd\ (\TVA : gsum pa'i tha' ma'i phyogs \TVA ; \TVB : gsum pa'i mtha' ma'i phyogs); tṛtīyāntaḥ | pakṣasya \MS ; tṛtīyapakṣasya \EDD 
} kalyāṇateti.\footnoteB{
	kalyāṇateti] \EDD ; kalyānateti \MS
}
atra kecid yuktiṃ varṇayanti.\footnoteA{
	\TVA\ renders this sentence differently: \emph{de la 'ga' zhig las rigs pa cung zhig cig brjod par mi bya ste |}
} prapañcarūpatvābhāve\footnoteB{
	prapañcarūpatvābhāve] \MS\ \EDD ; spros pa'i ngo bo nyid du gyur \TIB
} 'pi sūkṣmasya bindvādeḥ punaḥ punar bhāvanayā sākṣātkaraṇaṃ yāvat prayāsas tāvat sarvatraiva bhāvyavastuni sambhavati. tad atra yadi prayāsabhayam, na kiñcid api bhāvanīyam.

prapañcarūpatvād iti cet, prapañcāprapañcayor bhāvanāvasthāyāṃ ko viśeṣaḥ\footnoteB{
	viśeṣaḥ] \conj ; viśeṣa iti cet \MS\ \EDD
}? nanu\footnoteB{
	nanu] \conj\ (\TIB : 'on te); deest \emph{in} \MS\ \emph{and} \EDD
} aprapañcaṃ śīghram eva sthirībhavatīty ayaṃ viśeṣaḥ.
yatraivālambane\footnoteB{
	yatraivālambane] \conj\ (\emph{no reflect of nanu} \emph{in} \TIB); nanu yatraivālambane
} cittaṃ punaḥ punaḥ preryate nirantaraṃ\footnoteB{
	nirantaraṃ] \EDD\ (\emd) \TIB\ (rgyun mi 'chad par); niruttaraṃ \MS
} dīrghakālaṃ ca tatraiva sthirībhavatīty āgamaḥ. yuktiś cātrāsti. tathā coktam—

\begin{quote}
	tasmād bhūtam abhūtaṃ vā yad yad evābhibhāvyate | \\
	bhāvanābalaniṣpattau\footnoteA{
		The reading \emph{bhāvanābalaniṣpattau} is supported by the Tibetan translation and occurs in other sources (\emph{bsgom pa'i stobs ni rdzogs pa na}). Another more mainstream reading for this \emph{pāda} is \emph{bhāvanāpariniṣpattau}.
	} tat sphuṭākalpadhīphalam\footnoteB{
		kalpadhīphalam] \emd ; kalpadhīḥ phalam \MS\ \EDD
	} ||\footnoteA{
		\emph{Pramāṇavārttika}, Pratyakṣapramāṇa 285
	}

\end{quote}

punaś coktam—

\begin{quote}
	aho kusīdatvam aho vimūḍhatā\\
	aho janasyāsya sadarthavakratā |\\
	svacittamātrapratibaddhabuddhatā\footnoteB{
		°pratibaddha°] \conj\ (\TIB; 'brel pa); °pratibuddha° \MS\ \EDD
	}\\
	adūravartiny api yan na sevyate ||\footnoteB{
		Untraced. Also cited in *\emph{Saptāṅga} fol.\ 202r7.
	}
	% Vaṃśastha, LGLGGLLGLGLG X 4
% 
% 	Oh the laziness! Oh the foolishness! Oh these people's distortion of true meaning! For, that Buddhahood, which is tied to their own mere minds, is not served/pursued (na sevyate), though is resides so near.
\end{quote}

\noindent iti. tasmān nāyaṃ viśeṣaḥ.

bhrāntirūpatvenāparamārthatvam api sarvatraiva bhāvanāviśeṣe vastuni sambhavatīti na kiñcid api bhāvanīyaṃ syāt. [\MS\ fol.\ 8v] tataś ca sarvatraiva mokṣamārge bhāvanāyā vaiyarthyaṃ syāt. māyopamākārānupraveśena bhrāntirūpam apy aprapañcād [\EDD\ p.\ 145] bhāvyamānam\footnoteB{
	aprapañcād bhāvyamānam] \EDD ; aprapañcā bhāvyamāṇam
} aduṣṭaṃ bhavatīti cet, na tv ayaṃ māyākārānupraveśaḥ prapañce 'pi samāna iti. tatrāpi ko doṣasyāvakāśaḥ? tasmāt prapañcam aprapañcaṃ vā yad eva rocate pramāṇasaṃgatam itarad vā, tad evālasyaṃ vihāya mahāpuruṣārthibhir bhāvayitavyam\footnoteB{
	bhāvayitavyam] \EDD ; bhaviyitavyam \MS
} ity alam atiprasaṅgeneti.

atra ca sāretaravibhāgaḥ paryupāsitagurubhir eva jñātavyaḥ.

\subsubsection{pañcamasya asāratvam}
\noindent tṛtīyapakṣoktadoṣatvān\footnoteB{
	tṛtīyapakṣoktaṣatvān \conj\ (\TVB : gsum pa'i phyogs la bshad pa'i nyes pa yod pa dang); tṛtīyapakṣe ktato \MS ; tṛtīyapakṣe kuto \EDD ; \emph{no reflex} in \TVA
} nīrasatvena\footnoteB{
	nīrasatvena] \conj ; nīrasatvena te \MS\ \EDD
} prayojanābhāvān mantranayakramābhāvāc ca na pañcamaḥ parikṣīṇadoṣaḥ.

% \textbf{\TVA}\\
% dgos pa la sogs pa gsum pa'i phyogs la bshad pa'i nyes pa dang | gsang sngags kyi tshul gyi rim pa med pa'i phyir | lnga pa skyon dang bral ba ma yin no || \\
% 
% \textbf{\TVB}\\
% gsum pa'i phyogs la bshad pa'i nyes pa yod pa dang | dgos pa med pa'i phyir dang | gsang sngags kyi tshul gyi rim pa med pa'i phyir lnga pa skyon dang bral ba ma yin no || \\

nanu sākṣātkaraṇāt pūrvaṃ mantranayaprayogo 'sti. tat kathaṃ tasyābhāvaḥ? satyam, sākṣātphalāvasthā sādhyā. tasyāṃ ca nāsty asau kramaḥ. \crux sākṣātparityāge\footnoteA{
	Segment instead: kramaḥ sākṣāt. parityāge ?
}\crux\ ca na prayojanam utpaśyāma iti.

\subsubsection{ṣaṣṭhamasya asāratvam}
\noindent svecchayā nirvāyayitum\footnoteB{
	nirvāyayitum] \MS ; nirvāpayitum \EDD
} aśakyatvāt, prayojanābhāvāt, sattvārthābhāvāc ca na pañcāntaraprabhedakalpanā\footnoteB{
	pañcāntara°] \emd\ \TIB\ (lnga pa'i mtha'i rab tu dbye ba); prapañcāntara° \MS\ \EDD
} kalaṅkāśūnyā. tathā hi kasyacin nivṛttiḥ kāraṇanivṛttyā vyāpakanivṛttyā\footnoteB{
	vyāpakanivṛttyā] \EDD ; vyāpakānivṛttyā \MS
} vā bhavati. na cātra sākṣātkṛtamaṇḍalacakrasya nivartakaṃ kāraṇaṃ vyāpakaṃ vā icchākāle dṛśyate.\footnoteA{
	\TIB\ lacks a reflex of \emph{icchākāle dṛṣyate}. Both translations add an extra sentence to this paragraph: \emph{rang gi 'dod pas ('dos pas \TVB ; 'gog par \TVB)'gog pa yang mi nus te mi mthun pa med pa'i phyir | sdug bsngal la sogs pa 'gog pa 'dod kyang sdug bsngal la sogs pa la 'jug pa mthong ba'i phyir ro ||} 
}

nanu śūnyataiva nivartikāsti. yathā dārusaṅghātaprajvalito\footnoteB{
	dārusaṅghātaprajvalito] \conj ; dārusaṃghāte prajvalito \EDD ; dārusaṃghāt pravjalito \MS
} vahnir niḥśeṣam indhanaṃ bhasmīkṛtya paścāt svarasata eva nivartate, tathā maṇḍalacakraprajvalitaḥ śūnyatājñānāgniḥ sākṣāt kṛtvā\footnoteB{
	sākṣāt kṛtvā] \conj ; sākṣān \MS\ \EDD
} maṇḍalacakraṃ nivartayiṣyatīti cet.\footnoteB{
	\TIB\ a fuller sentence here. \TVB\ reads: \emph{de ltar dkyil 'khor gyi 'khor lo stong pa nyid kyi ye shes kyi me rab tu 'bar bas mngon sum du byas nas kyang | dkyil 'khor gyi 'khor lo ma lus par ldog par byed la | bdag nyid kyang rang gi ngang gis ldog par 'gyur ro zhe na |} \TVA\ appears to be slightly more corrupt, but suggests that same readings: \emph{de dkyil 'khor gyi 'khor lo stong pa nyid kyi ye shes kyi me rab tu 'bar bas mngon sum du byas nas kyang | dkyil 'khor gyi 'khor lo ma lus par ldog par byed la | de yang rang gi ldog par 'gyur ro zhe na |} 
} tad asat, viṣamatvād dṛṣṭāntasya. tathā hi tatrendhanaṃ kāraṇaṃ\footnoteB{
	kāraṇaṃ] \conj ; na kāraṇaṃ \MS\ \EDD
} vahneḥ. kāraṇasya indhanalakṣaṇasya nivṛttau\footnoteB{
	kāryasya indhanalakṣaṇasya nivṛttau] \conj ; kāryam indhanalakṣaṇanivṛttau
} yuktaiva vahnilakṣaṇasya kāryasya nivṛttiḥ. iha tu na śūnyatā kāraṇaṃ maṇḍalacakrasya. tat ka[\MS\ fol.\ 9r]thaṃ tannivṛttau nivṛttiḥ? na\footnoteB{
	na] \conj ; athavā na] \MS\ \EDD
} ca śūnyatāyā nivṛttir asti.\footnoteA{
	The response the objection is considerably different in Tibetan. It states that while fire is regarded by mundane consensus as having a causal effect on fuel insofar as it transforms it, emptiness has no such effect on the \emph{maṇḍalacakra}. It is also therefore not something that causes it to cease, nor is it known to itself cease of its own accord. Although the Sanskrit MS is very corruprt in this paragraph, it is difficult to see how the text it transmits corresponds to the Tibetan translation.
}

nanu sā na\footnoteB{
	na] \EDD\ (\emd); deest \emph{in} \MS
} bhavatu kāraṇaṃ. śūnyatā vyāpakaṃ tu bhaviṣyati. vyāpakasya vṛkṣasya nivṛttau śiṃśapātvasya vyāpyasya nivṛttivan nivṛttir bhaviṣyatīti cet. etad apy asāram. tathā hi śūnyatā sarvadā sarvajñeyamaṇḍalavyāpikā tattvarūpā.\footnoteB{
	tattvarūpā] \EDD ; tatvarūpāḥ \MS
} na ca tasyā nivṛttiḥ kadācid apy asti. yadi syāt samyaksaṃbodhisākṣātkaraṇāt [\EDD\ p.\ 146] pūrvam anantaram eva vā nivṛttiḥ syāt. na ca bhavati, samyaksaṃbuddhībhūyāpi katipayakālāvasthānasya svayam eva svīkṛtatvāt.

kintu śūnyatāpi jñānarūpā, cakram api jñānarūpam. śūnyatājñānotpattyā cakrajñānasyānivṛttau\footnoteB{
	°ānivṛttau] \MS\ \EDD ; log na \TIB (nivṛttau)
} śūnyatājñānaṃ kena nivartanīyam. tena nivṛttiś ca virodhino 'bhāvāt kāraṇavyāpakayoś cābhāvān nāsti. tasmāc chūnyatājñānasya na nivṛttiḥ,\footnoteB{
	na nivṛttiḥ] \conj\ (\TIB : ldog pa med do); nivṛttiḥ \MS\ \EDD
} nāpi maṇḍalacakrasya śūnyatāto nivṛttir iti śūnyatā na nivartikā.

ko brūte śūnyatā nivartikā? kiṃ tarhi yan nivartakaṃ\footnoteB{
	nivartakaṃ] \emd ; nivartikās \MS\ \EDD
} tad gurūpadeśato jñeyam ity apy asāram. gurūpadeśato 'pi na śūnyatāvyatiriktaṃ\footnoteB{
	śūnyatāvyatiriktaṃ] \conj\ vyatiri((ktiḥ)) \MS\ (i \emph{in} kti \emph{lacks a} pṛṣṭhamātrā); vyatiriktaḥ \EDD
} pramāṇato 'stīti yatkiñcid etat.\footnoteA{
	\TVB : bla ma'i man ngag las kyang stong pa nyid kyis ldog par byed pa ma yin ldog pa'i tshad ma cung zhig kyang yod pa ma yin pas. \TVA : bla ma'i man ngag las kyang stong pa nyid dang | de ldog pa las ma gtogs pa'i ldog par byed pa'i tshad ma gzhan cung zad yod pa ma yin no ||
} pratikṣaṇanivṛttiś ca kṣaṇabhaṅgarūpā sarvapadārthavyāpinī. na sā santānanivartikā. tasmān na svecchayā nivṛttiḥ.\footnoteB{
	nivṛttiḥ] \MS\ACreading ; nivṛrttiḥ \MS\PCreading
} na ca nivṛttyā\footnoteB{
	nivṛttyā] \EDD\ (\emd); nivartyā \MS
} nīrasarūpayā prayojanam asti prekṣāvatām. tathā coktam\emdash 

\begin{quote}
	mucyamāneṣu sattveṣu ye te prāmodyasāgarāḥ | \\
	tair eva nanu paryāptaṃ mokṣeṇārasikena kim ||\footnoteA{
		\emph{Bodhicaryāvatāra} 8.108
	}

% 	Inexpressible\footnoteA{
% 		The pronound combination \emph{ye te} can have the sense of "whatever there may be" (\cite[§287]{speijer1886}). In this case, Prajñākaramati in his \emph{Bodhicaryāvatārapañjikā} interprets them as conveying the sense of inexpressibility: \emph{ye te iti teṣām eva anubhavasiddhatvād idaṃtayā kathayitumaśakyāḥ} (p.\ 341).
% 	} oceans of happiness [arise for \emph{bodhisattva}s] when sentient beings are liberated.
% 	One may ask if those [oceans of happiness] alone are enough, but what would be the use of [any other] liberation that would be without relish?
\end{quote}

\noindent iti.

sattvārtho 'pi nivṛttau nāsti. na hi gagane\footnoteB{
	gagane] \MS\ \EDD\ \TVB ; \emph{no reflext in} \TVA
	} gaganakamale vā kācid arthakriyā sambhavati. ciraniruddhād apy atītād avasturūpāc\footnoteB{
		avasturūpāc] \MS\ \EDD\ \TVB\ (dngos po med pa'i ngo bo); dngos po'i ngo bo \TVA\ (vasturūpāc)
} cakrāt sattvārtho bhaviṣyatīty apy asāram, ciranīrutasyāpi\footnoteB{
	ciranīrutasyāpi] \conj ; cirutasyāpi \MS ; virutasyāpi \EDD ; yun rin por khyim bya shi ba \TVA ; yun ring por long pa'i khyim bya shi ba \TVB\ (ciramṛtasyāpi)
} kukku[\MS\ fol.\ 9v]ṭasya kaṇṭhadhvaniprasaṅgāt.

nanu yogyadhiṣṭhānād gaganād apy arthakriyāḥ sambhavantīti cet.\footnoteB{
	sambhavantīti cet] \conj ; saṃbhavanti \MS\ \EDD
} na sambhavanti, yogyadhiṣṭhānād eva cittarūpād arthakriyā, na gaganāt, nīrūpatvāt tasya.\footnoteA{
	\TVA\ varies significantly for this paragraph.
}

nanu nirodhya maṇḍalacakraṃ sattvārthakāle punar utpādyate. tato 'rthakriyā bhavati. tataḥ punar eva nirodhyate, punar evotpadyata iti cet. asad etat. yathā sattvārthakriyāyās tattvato\footnoteB{
	tattvato] \MS\ (tatvato) \EDD ; de las \TIB\ (tato)
} nāsti prādurbhāvaḥ, tathā cakrasyāpi. tato nārthakriyāyāḥ sambhavaḥ. na ca nirodhya\footnoteB{
	nirodhya] \EDD ; niro((dhya)) \MS\ (\emph{some kind of correction is made, but uncertain from what to what}); 'gogas pa las (\emph{possibly} nirodhāt)
} punar utpāde kiñcit prayojanam astīty alam atiprapañceneti.

\subsubsection{saptamasya asāratvam}
\noindent ṣaṣṭhapakṣoktadoṣasandohasya saptame\footnoteB{
	ṣaṣṭhapakṣoktadoṣasandohasya saptame] \conj\ (\TIB : drug pa'i phyogs la bshad pa'i skyon gyi (gyi] \TVA ; gyis \TVB) tshogs bdun pa la); ṣaṣṭhapakṣoktaṃ saṃdāhasyāṣṭame \MS ; ṣaṣṭhapakṣoktasaṃdohasyāṣṭame \EDD
} 'pi bhāvān na piṣṭapeṣaṇaṃ\footnoteB{
	piṣṭapeṣaṇaṃ] \MS\ACreading\ \EDD ; piṣṭapre | ṣaṇaṃ \MS\ACreading
} kriyate. nanu ṣaṣṭhena saptamasya samānatvāt kathaṃ saptamasya tato viśeṣaḥ? asti viśeṣaḥ. pūrvāvasthāyāṃ niyatacakrākāratā, punaḥ svecchayā nirvṛtiḥ svecchotpādanaṃ\footnoteB{
	nirvṛtiḥ svecchotpādanaṃ] \conj\ (\TVB : yang rang gi 'dod pas 'gog cing rang gi 'dod pas skyed par byed pa); svecchetpādanaṃ \MS ; svecchotpādanaṃ \EDD ; yang dang yang du rang gi 'dod pas skyed par byed pa nyid \TVA
} ceti. saptame punar etan nāsti. tato na samānatā. bhinnaś ca nirdiṣṭa iti.\footnoteB{
	\MS\ \EDD ; tha mi dad pa ma yin par bstan to \TVA ; tha mi dad pa ma yin par bstan to \TVB
}

\subsection{caturthasya sekasya svarūpam}
\begin{quote}
	dambholibījasrutidhautaśuddha-\footnoteB{
		°sruti°] \corr ; śruti \MS\ \EDD
	}\\
	pāthojabhūtāṅkurabhūtapuṣṭi\footnoteB{
		pāthoja°] \EDD\ (\emph{\EDD reports the ms.\ as reading \emph{pāthojña}, but this seems to be incorrect}); pāthauja° \MS
	}|\\
	turīyaśasyaṃ\footnoteB{
		turīyaśasyaṃ] \EDD; tutīyaśasyaṃ \MS
	} paripākam eti\footnoteB{
		eti] \EDD\ (\emd); eta \MS
	} \\
	sphuṭaṃ caturthaṃ viduṣo 'pi gūḍham || 17 ||

% 	The fourth, whose flourishing (\emph{puṣṭi}) is like (\emph{bhūta}) a sprout that cleased by the flow of the vajra's seed and which arises (\emph{bhūta}) from a pure lotus, reaches fruition. Although the fourth is manifest, it is hidden from even the wise.
\end{quote}

\noindent [\EDD\ p.\ 147] dambholītyādi. etat sadgurūpadeśato jñeyam.

\subsection{aparaṃ mithyāsādhyaṃ mithyātattvaṃ ca}
\begin{quote}
	pañcapradīpāmṛtabinducandra-\\
	bhrūmadhyabindūdbhavamaṇḍalāni |\\
	vāyoḥ svarūpaṃ galaśuṇḍikādyam \\
	atattvarūpaṃ svayam ūhanīyam || 18 ||
\end{quote}

\noindent pañcapradīpetyādi. pañcapradīpaśabdena gokudahanalakṣaṇasya, amṛtaśabdena vimumāraśulakṣaṇasya satatānuṣṭhānam eva sādhyaṃ manyante. bindur iti hṛccandrasthaṃ binduṃ dedīpyamānaṃ tattvaṃ sādhyaṃ ceti kṛtvā kecid bhāvayanti. candra iti hṛdisthaṃ kalārūpam ardhacandraṃ vā hṛtkamalasthaṃ kecid bhāvayanti.

bhrūmadhyabindūdbhavamaṇḍalānīti bhruvor madhye ūrṇāyāṃ binduṃ vibhāvya tadbindūdbhavāni maṇḍalāni vāyuvāruṇamāhendrāgneyalakṣaṇāni. etad uktaṃ bhavati—mukhaśravaṇanāsikācakṣurghrāṇarasanāni\footnoteB{
	mukhaśravaṇanāsikācakṣurghrāṇarasanāni] \MS\ \EDD ; kha dang | rna ba dang | sna dang | mig \TVA\ \TVB
} hastāṅgulībhiḥ pidhāya bhrūmadhyabindur draṣṭavyaḥ. tasya sphuṭāvasthāyāṃ śubhāśubhani[\MS\ fol.\ 10r]mittasaṃsūcakāni māhendrādimaṇḍalāny upajāyante. taṃ ca binduṃ tattvam iti manyante.

vāyoḥ svarūpam iti pūrakakumbhakarecakapraśāntakalakṣaṇam\footnoteB{
	°recaka°] \EDD ; recakaṃ \MS
} ānāpānādilakṣaṇaṃ\footnoteB{
	ānāpānādilakṣaṇaṃ] \EDD ; anāpānā° \MS
} ceti. etad\footnoteB{
	etad] \EDD\ (\emd); tad \MS
} uktaṃ bhavati—śaivasāṃkhyādinirdiṣṭaṃ\footnoteB{
	śaivasāṃkhyādi°] \EDD\ (\emd) \TVB\ (shi ba dang grangs can la sogs pas); saivasaṃkhyādi° \MS ; grangs can la sogs pas \TVA\ (sṃākhyādi°)
} vāyusvarūpaṃ jñātvā taṃ vāyuṃ nirodhabhāvanayā sthirīkṛtyākāśenotplutya gamanaṃ parapurapraveśaṃ yāvan muktiṃ ca sākṣātkurvanti vāyuvādinaḥ. 

galaśuṇḍiketi. galapradeśe jihvāmūlopari hastiśuṇḍikākārā adhaḥpralambamānā upajihvāsaṃjñikā galaśuṇḍikāsti. sā ca śaktirūpā. tadadhaḥ śivarūpam\footnoteB{
	tadadhaḥ śivarūpam] \MS\ \EDD\ \TVB\ (de'i 'og na zhi ba'i ngo bo) ; sdig pa'i rang bzhin du yong pa \TVA
} asti tattvam. sā ca [\EDD\ p.\ 148] jihvāgreṇa spṛśyamānā nirantarāmṛtaṃ sravati. tena ca ghargharāmṛtavarṣaṇena santarpyamānam ātmānaṃ dhyāyād iti galaśuṇḍikātattvam. ādiśabdena hṛnmadhyaṣoḍaśanāḍikācakramadhyasthajñānasvarūpaṃ\footnoteB{
	hṛnmadhyaṣoḍaśanāḍikācakramadhyasthajñānasvarūpaṃ] \MS\ \EDD\ \TVB\ (snying ka'i dbus kyi 'khor lo rtsibs bcu drug pa'i dbus na gnas pa ye shes kyi rang bzhin); snying ga'i dbus kyi dkyil 'khor rtsibs bcu drug pa'i dbus na hūm gnas pa ye shes kyi rang bzhin (hṛnmadhyaṣoḍaśanāḍikāmaṇḍalamadhyahūṁsthajñānasvarūpaṃ)
} śivarūpaṃ tattvaṃ bhāvayitavyam ityādīnāṃ parigrahaḥ.\footnoteA{
	\TIB\ continues to describe this practice. \TVA\ reads: \emph{yang smras pa | bcu las drug lhag rtsa dang ldan pa'i 'khor lo yi || dkyil na gnas pa'i snying gar rnam par gnas pa'i bdag | des ni de yi khyad par lta bu'i grub pa ster || de ni mngon par mi g-yo ba yi yid dag gis || rnal 'byor pa yi sems de de ltar mngon par bsam || nub par gyur pa'i mgon po rgyal bar gyur de ni || nus pa dag gis de ni yongs su bskor dang bcas ||} \TVB\ reads: \emph{de yang smras pa | bcu las drug lhag rtsa dang ldan pa'i 'khor lo'i dkyil na gnas pa snying kar rnam par gnas pa'i bdag  |des ni de'i khyad par lta bu yi grub pa ster | de ni mngon par mi g.yo ba'i yid dag gis || rnal 'byor pa yis de ltar mngon par bsam par bya || nus par gyur pa'i mgon po rgyal bar gyur || de ni nus pa dag gis de ni yongs su bskyor dang bcas ||}
}

tatsarvaṃ tīrthikādibhis tattvarūpeṇābhimatam. atattvam iti svayam evohanīyaṃ vicāraṇīyam iti yāvat.


\subsection{upasaṃhāra}
\begin{quote}
	svapnendrajālapratibimbamāyā-\\
	marīcigandharvapurāmbu{[}\MS\ fol.\ 2r{]}candraiḥ |\\
	anyaiś ca śabdair\footnoteB{
		śabdair \emd\ (cf.\ comm.); sarvair \MS\ \EDD
	} upamābhidheyair \\
	naivāsti sādhyaṃ kathitād ihānyat || 19 ||
	% Vāṇi: GGLGGLLGLGG, LGLGGLLGLGG, GGLGGLLGLGG, GGLGGLLGLGG
\end{quote}

\noindent svapnendrajāletyādi. svapnendrajālopamaṃ pratibimbamāyāmarīcigandharvanagarodakacandropamam iti śabdair anyaiś ca gagaṇapratiśrutkaphenopamam ityādiśabdair upamābhidheyair upamāvācakair naivāsti sādhyaṃ kathitāt sādhyād anyat. paraṃ kathita eva sādhye, ete śabdāḥ pravartanta iti svayaṃ boddhavyam.

\begin{quote}
	gambhīraśūnyapratibhāsamātra-\footnoteB{
		°mātra°] \EDD ; mātraṃ \MS
	}\\
	śāntāti\footnoteB{
		śāntāti] \EDD ; sāntādi \MS
	}sūkṣmānabhilāpyaśabdaiḥ |\\
	nirlepanīrūpa\footnoteB{
		nirlepanīrūpa°] \EDD\ (\emd) ; nirlepanīpa \MS
	}nirañjanādyair \\
	bhrāntir na kāryāparasādhyasattve || 20 ||
	% Indravajrā, GGLGGLLGLGL
\end{quote}

\noindent [\EDD\ p.\ 149] gambhīraśūnyaṃ pratibhāsamātraṃ śāntātisūkṣmam anabhilāpyaṃ nirlepaṃ nīrūpam\footnoteB{
	nīrūpam] \EDD\ (\emd); nirupamaṃ \MS
} nirañjanādi.\footnoteB{
	nirañjanādi] \MS ; nirañjanaṃ \EDD
} ādiśabdāt śivaṃ nirākāraṃ niṣprapañcam anādyantanidhanam i[\MS\ fol.\ 10v]tyādiśabdair bhrāntir na kartavyā. aparasādhyasattve, aparasya sādhyasya sattve sattāyām.\footnoteB{
	sattāyām] \MS ; sattvāyā \EDD
} ebhiḥ sarvair eva param api kiñcit sādhyaṃ kathitād astīti bhrāntir na kartavyā. atha nātikathitam eva sādhyam ebhiḥ sarvair abhidhīyata iti niścayaḥ.

\subsection{pariṇāmanā}
\begin{quote}
	akhilagagaṇagarbhavyāpisaptaprakāra-\footnoteB{
		°saptaprakāra°] \EDD ; °sarvaprakāra° \MS
	}\\
	grathitavacanarūpād yan mayāsādi puṇyam |\\
	anupamasukhavidyāsaktasaddehanirmij-\\
	jinajanitajanārthas tena loko 'yam astu ||
	% ā√-sad in the sense of to obtain, passive aorist = āsādi
	% Mālinī, LLLLLLGGGLGGLGG

	tattvaratnāvalokaḥ samāptaḥ. kṛtir iyaṃ paṇḍitavāgīśvarakīrtipādānām.\\
\end{quote}

\setlength\parindent{0pt}
śrīsamāje parā yasya bhaktir niṣṭhā ca nirmalā\\
tasya vāgīśvarasyeyaṃ kṛtir vimatināśinī\footnoteB{
	vimatināśinī] \EDD ; vimatināsanī \MS
} ||\\

% This composition, which destroys wrong views, is by Vāgīśvara, whose devotion intent on glorious \emph{Guhyasamāja} is supreme and stainless.
 
vikacakumudatārākṣīrakundānukāri\footnoteB{
	vikacakumudatārākṣīrakundānukāri \emd ; vikacakumudakṣīratārakundānukāri] \EDD ; vikarektāmudakṣīratārākundānukāri \MS
}\\
pracitam api ca puṇyaṃ yan mayā granthito 'smāt |\\
anupamasukhapūrṇaḥ svābhavidyopagūḍho\\
bhavatu nikhilalokas tena vāgīśvaraśrīḥ ||\\

% Mālinī, LLLLLLGGGLGGLGG X 4

tattvaratnāvalokavivaraṇaṃ samāptam. kṛtir iyaṃ paṇḍitācāryavāgīśvarakīrtipādānām.\\
\end{document}
