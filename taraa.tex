\documentclass[12pt]{article}
\usepackage{microtype}
\usepackage[pdfusetitle,hidelinks]{hyperref}
\usepackage{setspace}

\setlength\parindent{0pt}

\usepackage[series={A,B},noend,noeledsec,noledgroup]{reledmac}
\renewcommand*{\thefootnoteA}{\roman{footnoteA}}
\usepackage{marginnote}

\usepackage{fontspec}
\usepackage{polyglossia} 		
	\setmainlanguage[script=latin]{sanskrit}
	\setotherlanguage{english}	
	\setmainfont[Ligatures=TeX]{Libertinus Serif}
	\newfontfamily\sanskritfont[Ligatures=TeX]{Libertinus Serif}
	\newfontfamily\mantrafont[Ligatures=TeX]{STIX Two Text}[Scale=MatchLowercase]
	\newfontfamily\devfont[Script=Devanagari]{Noto Sans Devanagari}

\newcommand{\crux} {\hspace{0em}\textsuperscript{†}\hspace{0em}}
\newcommand{\emdash} {\hspace{0em}—\hspace{0em}}

\title{Tattvaratnāvaloka and Vivaraṇa}
\author{Vāgīśvarakīrti}

\begin{document}
\maketitle

% Define a new intercharacter class for Sanskrit question marks
\makeatletter
\newXeTeXintercharclass\noextraclass
\XeTeXcharclass `\? = \noextraclass
\XeTeXcharclass `\! = \noextraclass
\XeTeXcharclass `\; = \noextraclass
\XeTeXcharclass `\: = \noextraclass

% Remove the extra space between Sanskrit text and ? or !
\XeTeXinterchartoks 0 \noextraclass = {\nobreak}
\XeTeXinterchartoks \noextraclass 0 = {\nobreak}
\makeatother

%sigla
\newcommand{\PCreading}{$^{pc}$}
\newcommand{\ACreading}{$^{ac}$}
\newcommand{\MS}{K}
\newcommand{\EDD}{ED\textsubscript{DH}}
\newcommand{\TM}{TM\textsubscript{D}}
\newcommand{\TVA}{TVA\textsubscript{D}}
\newcommand{\TVB}{TVB\textsubscript{N}}
\newcommand{\TIB}{TIB}
\newcommand{\sigmareading}[1]{$\Sigma$\textsubscript{#1}}

% App shortcuts
\newcommand{\emd} {\emph{em.}}
\newcommand{\conj} {\emph{conj.}}
\newcommand{\corr} {\emph{corr.}}
\newcommand{\diag} {\emph{diag.\ conj.}}
\newcommand{\possibleemd} {\emph{possible em.}}
\newcommand{\possibleconj} {\emph{possible conj.}}

\section{Sigla and Abbreviations}
\noindent TaRaA = Tattvaratnāvaloka\\

\noindent TaRaA-V = Tattvaratnāvalokavivaraṇa\\

\noindent \EDD\ = Dhīḥ vol. 21, pp.\ 129–149.\\

\noindent \MS\ = NAK 5–252 = NGMPP A 915/4\\

\noindent \TM : \emph{De kho na nyid rin po che snang ba}. Tōhoku no.\ 1889. sDe dge bstan 'gyur, vol.\ Pi, fols.\ 203r3–204r5. Tr.\ by 'Gos Lhas btsas\\

\noindent \TVA : \emph{De kho na nyid rin po che snang ba'i rnam par bshad pa}.  Tōhoku\ no.\ 1890.  sDe dge bstan 'gyur, vol.\ Pi, fols.\ 204r5–214v4. Tr.\ by 'Gos Lhas btsas.\\

\noindent \TVB : \emph{De kho na nyid rin po che snang ba'i rnam par bshad pa}. Ōtani no.\ 4793. sNar thang bstan 'gyur. No translator given.

\noindent \TIB : Both Tibetan translations (differences, if any, indicated in a mini-aparatus)

\section{Text}
\subsection{Verse 1}
\subsubsection{Root text}
\begin{quote}
	[\MS\ fol.\ 1r] [siddhaṃ]\footnoteB{
		[siddhaṃ]] \MS ; oṁ \EDD
	} namaḥ śrīsadgurupādebhyaḥ |\footnoteA{
		Scribal homage
	}
	
	[\TM\ fol.\ 203r] / /rgya gar skad du/ tattv'a ratna a'a lo ka/ bod skad du/ de kho na nyid rin po che snang ba/ bcom ldan 'das 'jam pa'i rdo rje la phyag 'tshal lo/ /

	anupamasukharūpī śrīnivāso 'nivāso \\
	nirupamadaśadevīrūpavidyaḥ\footnoteB{
		nirupama°] \EDD\ ; nirūpama° \MS
	} savidyaḥ |\\
	tribhuvanahitasaukhyaprāptikāro 'vikāro \\
	jayati kamalapāṇir yāvad āśāvikāśāḥ\footnoteB{
		āśāvikāśāḥ] \corr ; āśāvikāsāḥ \MS\ \EDD
	} || 1 ||
	% LLLLLLGGGLGGLGG Mālinī
	
	So long as there are opportunities for hopes/spaces, the Lotus Hand reigns supreme\emdash he whose nature incomparable bliss, who is an abode of riches, who is without any abode, whose consorts are the incomparable ten goddesses, who is accompanied by his consort, who brings about the attainment of bliss that has benefit for the three worlds, and who is without degeneration.
	
	dpe med bde ba'i ngo bo dpal gyi gnas gyur gnas 'dren pa/ /\\
	dpe med lha mo bcu yi ngo bo rig dang rig par bcas/ /\\
	srid pa gsum phan bde ba thob mdzad 'gyur min 'gyur ba med/ /\\
	ji srid re ba snang bar phyag na padma rgyal gyur cig /
\end{quote}

\subsubsection{Commentary}
[\MS\ fol.\ 2r3] namaḥ samantakāyavākcittavajrāya.\footnoteA{
	Scribal homage
}\\

\textbf{\TVA}\\
{[}\TVA\ fol.\ 204r{]}/ /rgya gar skad du/ tattv'a ratna a'a lo ka by'a khy'a na/ bod skad du/ de kho na nyid rin po che snang ba'i rnam par bshad pa/ bcom ldan 'das gsung gi dbang phyug la phyag 'tshal lo/ /'od dpag med pa rin chen gyis/ /'od kyis spyi gtsug mdzes pa yi/ /'jig rten dbang la phyag 'tshal lo/ /bzhi pa'i snang bar rab bshad bya/ /\\

\textbf{\TVB}\\
{[}\TVB\ fol.\ 70v{]} rgya gar skad du/ tad ta rad na ā lo ka bya khya na/ bod skad du/ de kho na nyid rin po che snang ba'i rnam par bshad pa/ bcom ldan 'das gsung gi dbang phyug la phyag 'tshal lo/\\

anupametyādi.
kamalaṃ padmaṃ pāṇau yasya sa kamalapāṇir avalokiteśvaro bhagavāñ jayatīti sambandhaḥ.
kiṃviśiṣṭaḥ?
anupamam ity atipraṇītatvamahattvāsaṃsārasthāyitvalakṣaṇair\footnoteB{
	°saṃsārasthāyitva°] \MS; °saṃsārasthāyisva° \EDD\ (\emph{note the two akṣaras}, tva \emph{and} sva, \emph{are very similar})
} dharmair yuktasyānyasyābhāvād\footnoteA{
	cf.\ Tib.: dpe med ces bya ba la sogs pa smos te/ dpe med pa ni (ni] \TVA; dang \TVB) shin tu gya nom pa nyid dang/ rgya (rgya] \TVA; deest in \TVB) che ba nyid dang/ 'khor ba'i mtha'i bar du gnas pa'i mtshan nyid kyi chos dang ldan pa ste/ gzhan dag la de med pa'i phyir ro/ / (āha—anumapetyādi. anupamam iti atipraṇītatvamahattvāsaṃsārasthāyitvalakṣaṇair yuktam, anyasya tadabhāvād.)\\
} upamārahitaṃ sukham eva rūpaṃ svabhāvo yasya sa tathoktaḥ.
punar api kiṃviśiṣṭaḥ?
śrīḥ puṇyajñānasambhāralakṣaṇā.
tasyā nivāsa āśrayo yaḥ sa tathā.
dharmakāyarūpitvena\footnoteB{
	dharmakāyarūpitvena] \MS\ \EDD; dharmakāyarūpatvena \possibleemd\ (\emph{cf.} \TVA\ \TVB: chos kyi sku'i ngo bo nyid kyis)
} sarvagatatvāt [\EDD\ p.\ 132] pratiniyatanivāsābhāvād anivāsaḥ.\\

\textbf{\TVA}\\
dpe med ces bya ba la sogs pa la gang gi phyag na padma bsnams pa de ni phyag na padma ste/ bcom ldan 'das spyan ras gzigs dbang phyug yin la/ de rgyal bar gyur cig ces bya bar sbyar ro/ /khyad par ji lta bu zhig dang ldan zhe na/ dpe med ces bya ba la sogs pa smos te/ dpe med pa ni shin tu gya nom ba nyid dang/ rgya che ba nyid dang/ 'khor ba'i mtha'i [\TVA\ fol.\ 204v] bar du gnas pa'i mtshan nyid kyi chos dang ldan pa ste/ gzhan dag la de med pa'i phyir ro/ /gang gis rang bzhin bde ba nyid kyi ngo bo dpe dang bral bar gyur pa de la de skad ces bya'o/ /gzhan yang khyad par ji lta bu zhig dang ldan zhe na/ dpal de bsod nams dang ye shes kyi tshogs kyi mtshan nyid yin la/ de'i gnas ni gzhir gyur pa gang yin pa de la de skad ces bya'o/ /chos kyi sku'i ngo bo nyid kyis kun du 'gro ba nyid yin pa'i phyir/ so sor nges pa'i gnas med pas na gnas med pa'o/ /\\

\textbf{\TVB}\\
dpe med ces bya ba la sogs pa la/ phyag na pad ma bsnams pa de ni/ phyag na pad ma ste bcom ldan 'das spyan ras gzigs dbang phyug ste/ de rgyal gyur cig ces bya bar sbyar ro/ /khyad par ci lta bu zhig dang ldan pa ce na/ dpe med ces bya ba la sogs pa smos te/ dpe med pa dang shin tu <bdag don gyi> gya nom pa nyid dang/ <'dar ma ka ya> che ba nyid dang/ <gzhan gyi don tu ba pa ka ya> 'khor ba'i mtha'i bar du gnas pa'i mtshan nyid kyi chos dang ldan pa ste/ gzhan dag la med pa'i phyir ro/ /gang gi rang bzhin bde ba nyid kyi ngo bo dpe dang bral bar gyur pa de la de skad ces bya'o/ /gzhan yang khyad par ji lta bu zhig dang ldan zhe na/ dpal ste bsod nams dang ye shes kyi tshogs kyi mtshan nyid yin la/ de'i gnas ni gang gi gzhir gyur pa de la de skad ces bya'o/ /chos kyi sku'i ngo bo nyid kyis kun tu 'gro ba nyid yin pa'i phyir so sor nges pa'i gnas med pas na gnas med pa'o/ /\\

punaḥ kīdṛśaḥ?
nirupamāḥ paramarūpayauvanaśṛṅgārādirasamahākaruṇādiyuktatvenopamātikrāntā\footnoteB{
	°opamātikrāntā] \MS\ \EDD\ \TVB\ (dpe las ’das pa’o) ; dpe med pa ste/ dpe las ’das pa’i \TVA\ (nirupamā upamātikrāntā)
} rūpavajrāditārāparyantadaśadevīrūpā vidyāḥ parivārakatvena\footnoteB{
	parivārakatvena] \emd ; saparivārakatvena \MS ; saparivārakatvena \EDD
} yasya sa tathā.
saha svābhārūpayā vidyayā\footnoteB{
	vidyayā] \MS\ \EDD ; rig pa ste/ shes rab \TVA\ \TVB\ (vidyayā prajñayā)
} vartata iti savidyaḥ.
tribhuvanasya tribhuvanavartino janasya yaddhitam āyatipathyaṃ\footnoteB{
	āyatipathyaṃ] \emph{variant word division in} \EDD : āyati pathyaṃ; \emph{and in} \MS : āyati | pathyaṃ
} buddhatvādikam, saukhyaṃ tad āpātapathyaṃ\footnoteB{
	tad āpātapathyaṃ] \conj\ (\TVA : 'phral gyi phan pa); tad dāpayati pathyaṃ \MS\ \EDD ; de la bde ba ni bde ba ste \TVB
} cakravartitvādikam, tasya yā prāptiḥ\footnoteB{
	prāptiḥ] \MS\ \EDD ; thob pa ni rnyed pa ste \TVA\ \TVB
} [\MS\ fol.\ 2v] sākṣātkriyā, tasyāḥ karaṇaṃ kāro yasya sa tathā.
aparinirvāṇadharmakatvenāpratiṣṭhitanirvāṇarūpatvenā\footnoteB{
	°rūpatvenā°] \MS\ \EDD ; ngo bo rnyed pas \TVA ; ngo bo brnyed pas \TVB\ (°rūpaprāptyā°)
}nyathātvalakṣaṇasya vikārasyābhāvād avikāraḥ.
evaṃviśiṣṭo bhagavāñ jayati.\\

\textbf{\TVA}\\
yang khyad par ci 'dra zhe na/ dpe med rnams te/ gzugs mchog tu gyur pa dang/ na tshod dang/ sgeg pa la sogs pa'i nyams dang snying rje chen po la sogs pa dang ldan pa nyid kyis na dpe med pa ste/ dpe las 'das pa'i gzugs rdo rje la sogs pa nas/ sgrol ma'i mthar thug pa'i lha mo bcu'i ngo bo rig ma'i 'khor gyis bskor ba gang la yod pa de la de skad ces bya'o/ /de nyid ni rang gi 'od kyis rig pa ste/ shes rab dang lhan cig 'jug pa'i phyir rig pa dang bcas pa'o/ /srid pa gsum ni 'jig rten gsum na gnas pa'i skye bo'o/ /gang la phan pa ni ma 'ongs pa'i phan pa ste/ sangs rgyas nyid la sogs pa'o/ /de la bde ba ni 'phral gyi phan pa ste/ 'khor los sgyur ba nyid la sogs pa'o/ /de thob pa ni rnyed pa ste mngon du byas pa'o/ /de dag sgrub par mdzad po gang yin pa de la de skad ces bya'o/ /yongs su myang ngan las mi 'da' ba'i chos nyid kyis mi gnas pa'i myang na las 'das pa'i ngo bo rnyed pas rnam pa gzhan du mi 'gyur ba'i phyir 'gyur ba med pa yin te de lta bu'i khyad par can gyi bcom ldan 'das rgyal bar gyur cig ces bya ba'o/ /\\

\textbf{\TVB}\\
yang khyad par ci lta bu zhig dang ldan ce na/ dpe med te gzugs mchog tu gyur pa dang na tshod dang/ sgeg pa la sogs pa'i rang bzhin dang/ snying rje chen po la sogs pa dang ldan pa nyid kyis na dpe las 'das [\TVB\ fol.\ 71r] pa'o/ /gzugs rdo rje la sogs pa nas sgrol ma'i mthar thug pa lha mo bcu'i ngo bo rig ma'i 'khor gyis bskor ba gang la yod pa de la de skad ces bya'o/ /rang snang ba'i ngo bo rig pa ste shes rab dang lhan cig 'jug pa'i phyir rig pa dang bcas pa'o/ /srid pa gsum ni srid pa gsum na gnas pa'i skye bo'o/ /gang la phan pa ni ma 'ongs pa'i phan pa ste/ rgyas nyid la sogs pa dang/ de la bde ba ni bde ba ste/ 'khor lo bsgyur ba nyid la sogs pa'o/ /de thob pa ni rnyed pa ste/ mngon sum du byas pa'o/ /de dag gi rgyu mdzad pa gang yin pa de la de skad ces bya'o/ /yongs su mya ngan las mi 'da' ba'i chos nyid kyi mi gnas pa'i mya ngan las 'das pa'i ngo bo nyid brnyes pas rnam pa gzhan du 'gyur ba med pa'i phyir 'gyur ba med pa yin te/ de lta bu'i khyad par can gyi bcom ldan 'das rgyal bar gyur cig ces pa'o/\\

kiyantaṃ kālam ity āha\emdash yāvad āśāvikāśāḥ.\footnoteB{
	āśāvikāsāḥ] \corr ; āśāvikāśāḥ \EDD\ \MS
} āśā daśa diśo gaganasvarūpāḥ. yadvā āśāḥ sarvasattvānāṃ bhavabhogatṛṣṇāḥ.\footnoteB{
	°tṛṣṇāḥ] \EDD\ (°tṛṣṇās); tṛṣṇā \MS
} tāsāṃ vikāśā\footnoteB{
	vikāśā] \corr; vikāsā \MS\ \EDD
} avakāśāḥ pravartanāni, prādurbhāvā iti yāvat.
te yāvat\footnoteB{
	te yāvat] \emd ; tā yāvat \MS\ \EDD\ deest \emph{in \TIB}
} tāvad bhagavāñ jayati.
sarvahariharahiraṇyagarbhādibhyaḥ prakṛṣṭo bhavatīty arthaḥ.\\

\textbf{\TVA}\\
dus ji srid du zhe na/ ji srid re ba snang bar zhes bya ba smos te/ re ba'i ngo bo'i sgras ni phyogs bcu nam mkha'i rang bzhin pa'am/ yang na re ba ste/ sems can thams cad srid pa'i longs spyod la sreg pa'o/ /re ba de snang ba ni gsal ba dang/ rab tu 'jug pa dang/ rab tu skye ba'i bar du ste/ de srid du bcom ldan 'das rgyal bar gyur pa ni khyab 'jug dang/ lha chen po dang/ tshangs pa la sogs pa thams cad las khyad par du 'phags par gyur [\TVA\ fol.\ 205r]/ /cig ces bya ba'i don to/ /\\

\textbf{\TVB}\\
dus ji srid du ce na/ ci srid re bar snang ba zhes bya ba smos te/ <ka pa> re ba'i ngo bo'i sgra ni phyogs bcu'i nam mkha'i <a ka pa> rang gi ngo bo'am/ yang na re ba ste sems can thams cad srid pa'i longs spyod la sred pa'o/ /re ba snang ba ni gsal ba ste/ rab tu 'jug pa/ rab tu skye ba'i bar du'o/ /de srid bcom ldan 'das rgyal bar gyur pa ste/ khyab <ta ri ha ra> 'jug dang lha chen po dang/ tshangs pa la sogs pa thams cad las khyad par du 'phags par [\TVB\ fol.\ 71v] gyur pa zhes bya ba'i don to/ /\\

atrānupamasukharūpīty anena svārthasaṃpattiḥ kathitā.
śrīnivāsa ity anena tadupāyaḥ, puṇyajñānasaṃbhārayoḥ śrīśabenābhihitatvāt.
tribhuvanahitasaukhyaprāptikāra ity anena parārthasaṃpattir uktā.
nirupamadaśadevīrūpavidyaḥ savidya ity anena tadupāyaḥ, tathābhūtadaśadevīdvātriṃśallakṣaṇāśītyanuvyañjanakāyākāraśūnyena sarvākāraparārthasaṃpatteḥ kartum aśakyatvād iti.\\

\textbf{\TVA}\\
de la dpe med bde ba'i ngo bo zhes bya ba 'dis ni rnag gi don phun sum tshogs pa bstan to/ /dpal gyi gnas 'gyur zhes bya ba ni de'i thabs yin te bsong nams dang ye shes kyi tshogs dag ni dpal gyi sgras mngon par brjod pa'i phyir ro/ srid pa gsum na phan dang bde ba thob par mdzad/ /ces bya ba 'dis ni gzhan gyi don phun sum tshogs pa bstan to/ /dpe med lha mo bcu yi ngo bo yi ngo bo rig dang rig mar bcas/ /zhes bya ba 'dis ni de'i thabs yin te/ lha mo bcu dang dam pa'i sku mtshan sum cu rtsa gnyis dang/ dpe byad bzang po brgyad cu'i rnam pas stong na/ gzhan gyi don phun sum tshogs pa rnam pa thams cad du mdzad pa'i mthu mam mnga' ba'i phyir ro/ /\\

\textbf{\TVB}\\
de la dpe med bde ba'i ngo bo zhes bya ba 'dis ni rang gi don phun sum tshogs pa bstan to/ /dpal gyi gnas gyur ces bya ba 'dis ni/ de'i thabs yin te/ bsod nams dang ye shes kyi tshogs dag ni dpal gyi sgras mngon par brjod pa'i phyir ro/ /srid pa gsum la phan pa dang bde ba thob mdzad ces bya ba 'dis ni gzhan gyi don phun sum tshogs pa bstan to/ /dpe med lha mo bcu'i ngo bo'i rig ma dag dang rig par bcas zhes bya ba 'dis ni de'i thabs yin te/ de lta bu'i lha mo bcu dang dam pa'i sku mtshan sum cu rtsa gnyis dang/ dpe byad bzang po brgyad cu'i rnam pas stong na/ gzhan gyi don phun sum tshogs pa rnam pa thams cad du mdzad pa'i mthu mi mnga' ba'i phyir ro/ /

\subsection{Verse 2}
\subsubsection{Root Text}
\begin{quote}
	śrīmantranītigatacārucaturthaseka-\\
	rūpaṃ vidanti na hi ye sphuṭaśabdaśūnyam |\\
	nānopadeśagaṇasaṃkulasaptabhedais\\
	teṣāṃ sphuṭāvagataye kriyate prayatnaḥ || 2 ||
	% GGLGLLLGLLGLGL Vasantatilaka

	Employing seven divisions thronging with the teachings of various [teachers], I undertake the present effort to bring about clear understanding to those who indeed do not know the nature, devoid of [expression through] clear words, of the beautiful fourth consecration belonging to the glorious Way of Mantra.

	sngags kyi gzhung lugs mdzes pa'i dbang bzhi'i ngo bo dag /\\
	rab tu gsal ba'i sgras stong gyur pas gang gis mi shes shing/ /\\
	du ma'i man ngag bdun gyi dbye ba'i tshogs kyis yongs 'khrugs pa/ /\\
	de dag rnams kyi gsal bar rtogs par bya phyir 'bad pas bya/ /
\end{quote}

\subsubsection{Commentary}
śrīmantranītiśabdena sāmānyayogatantravācakenāpi śrīsamājaḥ parigṛhyate, caturthārthakasyānyatrāsambhavāt.
	% svārthe kaḥ?
śeṣaṃ subodham.
nānācāryopadeśagaṇasaṃkulai\hspace{0em}[\EDD\ p.\ 133]\hspace{0em}r vyākulaiḥ saptabhir bhedaiḥ prakārair atītānāgatavartamānācārya\footnoteB{
	°vartamānā°] \EDD ; °pravartamānā° \MS
}gatopadeśarāśisaṃgrāhakaiḥ.
sphuṭāvagataye sukhena sphuṭapratītyartham iti.\\

\textbf{\TVA}\\
dpal ldan sngags kyi gzhung lugs shes/ /zhes bya ba la sogs pa la/ sngags kyi gzhung lugs zhes bya ba'i sgras ni/ rnal 'byor gyi rgyud spyir rjod par byed pa yin du zin kyang shugs kyis dpal gsang ba 'dus pa yongs su gzung bar bya ste/ gzhan dag la bzhi pa'i don mi srid pa'i phyir ro/ /lhag ma ni go sla'o/ /du ma'i man ngag ces bya ba la sogs pa la/ 'das pa dang ma 'ongs pa dang/ da ltar gyi slob dpon du ma'i man ngag gi tshogs yang dag par bsdus pa'i mdun gyi dbye bas yongs su dkrugs pa ni/ rnam par dkrugs pas rnam pa thams cad la rnam par khyab pa ste/ des bsgrub par bya ba dkrugs pa'o/ /gsal bar rtogs par bya ba'i phyir/ /zhes bya ba ni bde bar gnas par khong du chud par bya ba'i phyir zhes bya ba'i thi tshig go/ /\\

\textbf{\TVB}\\
dpal ldan sngags kyi gzhung lugs zhes bya ba'i sgras/ rnal 'byor rgyud shugs kyis dpal gsang ba 'dus spyir brjod par byed pa yin du zin kyang/ pa yongs su gzung bar bya ste/ gzhan dag la don bzhi pa mi srid pa'i phyir ro/ /lhag ma ni go sla'o/ /slob dpon du ma'i man ngag gis yongs su dkrugs pa'i rnam par 'khrugs pa rnam par bkab pa ste bdun gyi dbye bas zhes bya ba ni 'das pa dang/ ma 'ongs pa dang/ da ltar gyi slob dpon du gtogs pa'i man ngag gi tshogs yang dag par bsdus pas so/ /gsal bar rtogs par bya ba'i phyir/ zhes bya ba ni bde bar gsal bar khong du chud par bya'o/ /[\TVB\ fol.\ 72r] zhes bya ba'i tha tshig go/ /

\subsection{Verse 3}
\subsubsection{Root text}
\begin{quote}
	sambhrāntabodhā nikhilā hi tīrthyās \\% final s in tīrthyās is unclear, but probably there; there is a slight crease in the image
	tattvasya sādhyasya ca rūpavittau |\\
	tebhyaḥ prakṛṣṭaḥ kila tattvavettā\\
	vedāntavādīti janapravādaḥ || 3 ||
	% GGLGGLLGLGG, Indravajra

	All non-Buddhist philosophers (\emph{tīrthya}) have completely mistaken ideas regarding the realisation of the nature of reality and the goal.
	People say that say that the best among them are the Vedāntins, supposedly knowers of reality.\\

	dpal ldan de nyid bsgrub bya'i ngo bor rig pa la/ /\\
	mu stegs ma lus blo rnams rnam par 'khrul/ /\\
	de las rig byed mthar ni smra ba yi/ /\\
	skye bo rnams ni de nyid mchog ces smra/ /
\end{quote}

\subsubsection{Commentary}
sambhrāntetyādi.
sambhrānto vibhrānto bodhaḥ prajñāviśeṣo yeṣāṃ tīrthikānāṃ te tatho[\MS\ fol.\ 3r]ktāḥ.\footnoteB{
	te tathoktāḥ]; \MS\PCreading ; te thoktāḥ \MS\ACreading ; tathoktāḥ \EDD
}
sarva eva tīrthyā ātmātmīyagrahatimiropahatabuddhinayanāḥ.
tattvam idam iti sādhyaṃ cedam iti ca tattvasya sādhyasya yat\footnoteB{
	yat] \EDD\ (\emd); tat \MS
} svarūpaṃ tasya yā vittiḥ pratītis tasyāṃ bhrāntāḥ.
śeṣaṃ subodham.\\

\textbf{\TVA}\\
de nyid bsgrub bya'i zhes bya ba la sogs pa la/ mu stegs can gang la kun nas 'khrul pa ste/ rnam par 'khrul pa'i blo shes rab kyi bye brag yong pa de la de skad ces bya'o/ /mu stegs can de dag thams cad bdag dang bdag gir 'dzin pa'i ling tog gis blo'i mig khebs par gyur pas/ de kho na nyid 'di zhes sam/ bsgrub par bya ba ni 'di zhes de kho na nyid dang/ bsgrub par bya ba de'i rang gi ngo bo gang yin pa rig cing khong du chud par bya ba de dag la 'khrul pa'o/ /lhag ma ni go sla'o/ /\\

\textbf{\TVB}\\
de nyid zhes bya ba la sogs pa la/ mu stegs can gang la kun nas 'khrul pa ste/ rnam par 'khrul pa'i blo shes rab kyi bye brag yod pa de la de skad ces bya'o/ /mu stegs can de dag thams cad bdag dang bdag gir 'dzin pa'i ling tog gis blo yi dmig khebs par gyur pas/ de kho na nyid ni 'di'am/ bsgrub par bya ba ni 'di'o zhes de kho na nyid dang bsgrub par bya ba de'i rang gi ngo bo gang yin pa rig cing khong du chud pa la de dag 'khrul pa'o/ /lhag ma ni go sla'o/ /\\

nanu tattvasādhyayor upādeyatvenaikarūpatvāt tattvasya sādhyasya ceti kathaṃ\footnoteB{
	tattvasya sādhyasya ceti kathaṃ] \EDD\ (\emd); tat kathaṃ tatvasya sādhyasya ceti \MS
} bhedena nirdeśa iti cet.
asad etat.
tattvaṃ hy upādeyatvenāpi sukhaduḥkhopekṣādisakalapratibhāsasaṃdohavyāpakam.
sādhyaṃ cānabhimataparihāreṇecchālakṣaṇaṃ phalam.
upādeyatve 'pi sakalaprāṇibhir avaśyam evāsādhyavyāvṛttyā sādhayitavyatvenābhimatam ity adoṣaḥ.\\

\textbf{\TVA}\\
de kho na nyid [\TVA\ fol.\ 205v] dang bsgrub par bya ba dag ni blang bar bya ba nyid du ngo bo gcig pa ma yin nam/ ci'i phyir de kho na nyid dang/ bsgrub par bya ba zhes dbye ba bstan zhe na/ de ni bden pa ma yin te/ de kho na nyid ni blang bar bya ba nyid yin yang bde ba dang sdug bsngal dang/ btang snyoms la sogs pa so sor snang ba'i tshogs ma lus pa la shes bya tsam du khyab par byed pa yin la/ bsgrub par bya ba yang mngon par 'dod par bya ba ma yin pa las bzlog par 'dod pa'i mtshan nyid kyis blang bar bya ba nyid yin kyang/ srog chags thams cad kyis gdon mi za bar bsgrub par bya ba ma yin pa las bzlog pa/ /bsgrub par bya ba'i 'bras bu nyid du mngon par 'dod pa'i phyir nyes pa med do/ /\\

\textbf{\TVB}\\
gal te de kho na nyid dang bsgrub par bya ba dag la ni blang bar bya ba nyid du ngo bo nyid gcig pa ma yin nam/ ci'i phyir de kho na nyid dang bsgrub par bya ba dbye ba bstan ce na/ de ni bden pa ma yin te/ de kho na nyid blang bar bya ba nyid yin yang bde ba dang sdug bsngal dang btang snyoms la sogs pa so sor snang ba'i tshogs ma lus pa la khyab par byed pa yin la/ bsgrub par bya ba yang mngon par 'dod pa ma yin pa las bzlog par 'dod pa'i mtshan nyid kyis 'bras bu blang bar bya ba nyid yin yang srog chags thams cad kyis gdon mi za bar bsgrub bya ba ma yin pa las bzlog pa bsgrub par bya ba nyid du mngon par 'dod pa'i phyir nyes pa med do/ /

\subsection{Verse 4}
\subsubsection{Root text}
\begin{quote}
	ānandarūpaṃ svavid aprakampyaṃ \\
	vedāntinaḥ sādhyam uṣanti śāntam\footnoteB{
		śāntam] \corr ; sāntam \MS\ \EDD
	} |\\
	saśrāvakākhaḍgajināś ca sādhyam\\
	icchanti rūpādyupadher virāmam || 4 ||

	The Vedāntin long for (√\emph{vaś}) a goal that is self awareness (\emph{svavit})\emdash bliss by nature, unmovable, and at peace. And the rhinoceros-jinas, as well as \emph{śrāvaka}s, desire as a goal the cessation (\emph{virāma}) of the substrates (\emph{upadhi}) of material form (\emph{rūpa}) and so forth.

	dga' ba'i ngo bo rang rig mi g-yo zhing/ \\
	bsgrub byar 'dod pa rig byed mtha' can pa'o/ /\\
	nyan thos rang rgyal bcas pa gzugs la sogs/ /\\
	phung po dag dang bral ba bsgrub byar 'dod/ /
\end{quote}

\subsubsection{Commentary}
tatra tāvad vedāntavādyabhimataṃ sādhyam āha\emdash ānandarūpam ityādi.
ānandarūpam iti sadāsukhamayatvāt.
svavid iti jyotirūpatvena\footnoteB{
	jyotīrūpatvena] \MS ; jyotirūpatvena \EDD
} svayaṃ prakāśamānānatvāt.\footnoteB{
	prakāśamānānatvāt] \EDD\ (\emd); prakāśamānāt \MS
}
aprakampyam iti nityatayā\footnoteB{
	nityatayā] \EDD ; anityatayā \MS
} kampayitum aśakyatvāt.
śāntam\footnoteB{
	śāntam] \corr ; sāntam \MS\ \EDD
} iti kleśopakleśaśūnyatvena parikalpitatvāt.
evaṃvidhaṃ sādhyam uṣanti kāmayante.\\

\textbf{\TVA}\\
de la re zhig rig byed kyi mthar smra ba'i mngon par 'dod pa'i bsgrub par bya ba ni dga' ba'i ngo bo zhes bya ba la sogs pa smos te dga' ba'i ngo bo ni rtag tu bde ba'i rang bzhin nyid yin pa'i phyir ro/ / rnag rig pa ni gsal ba'i ngo bo nyid yin gyis bdag nyid rab tu snang ba nyid kyi phyir ro/ /mi g-yo ba zhes bya ba ni mi rtag pa nyid kyis bskyod par mi nus pa'i phyir ro/ /zhi ba ni nyon mongs pa dang/ nye ba'i nyon mongs pas stong pa nyid du yongs su brtags pa'i phyir ro/ /rnam pa de lta bu bsgrub par bya ba la bdun pa (sic for 'dun pa) la ni 'dod par gyur pa'o/ /\\

\textbf{\TVB}\\
de la rig byed kyi mthar smra ba'i mngon par 'dod pa'i sgrub par bya ba ni/ dga' ba'i ngo bo zhes bya ba la sogs pa smos te/ dga' ba'i ngo bo nyid ni rtag [\TVB\ fol.\ 72v] tu bde ba'i rang bzhin nyid yin pa'i phyir ro/ /rang rig pa ni gsal ba'i ngo bo nyid kyis bdag nyid rab tu snang ba nyid kyi phyir ro/ /mi g.yo ba zhes bya ba ni/ mi rtag pa nyid kyis bskyod par mi nus pa'i phyir ro/ /zhi ba ni nyon mongs pa dang nye ba'i nyon mongs pas stong pa nyid du yongs su brtags pa'i phyir ro/ /rnam pa de lta bu bsgrub par bya ba la 'dun pa ni 'dod pa'o/ /\\

saha śrāvakair vartante ye khaḍgajināḥ khaḍgaviṣāṇakalpā ekacāriṇo vargacāriṇaś\footnoteB{
	vargacāriṇaś] \MS\ (\emph{cf.\ Abhidharmakośabhāṣya}); vanacāriṇaś \EDD 
} ca pratyekabuddhās te sādhyam icchanti.
kīdṛśam?
rūpādyupadher virāmaṃ rūpavedanāsaṃjñāsaṃskāravijñānalakṣaṇānām upadhīnāṃ skandhānāṃ virāmaṃ vicchedam, nirodham iti yāvat.
[\EDD\ p.\ 134] etad uktaṃ bhavati\emdash sarvaśrāvakapratyekabuddhāḥ sopadhiśeṣanirupadhiśeṣabhedena bhinne 'pi nirvāṇe\footnoteB{
	nirvāṇe] \EDD ; nirvāṇa° \MS
} nirupadhiśeṣam eva nirvāṇaṃ sā[\MS\ fol.\ 3v]kṣātkartavyatvena sādhyaṃ pratipannāḥ.\\

\textbf{\TVA}\\
gang nyan thos dang bcas pa 'jug pa'i rang rgyal ba ni bse ru lta bu ste/ gcig pu rgyu ba dang/ tshogs kyis spyod pa'i rnag sangs rgyas dag bsgrub byar 'dod do/ /ci 'dra ba zhig ce na/ gzugs la sogs pa'i phung po dang bral ba'o/ /gzugs dang/ tshor ba dang/ 'du shes dang/ 'du byed dang/ rnam par shes pa'i mtshan nyid dag gi phung po rnams ni spungs pa ste bral ba dang rnam chad pa dang 'gags par gyur pa zhes bya ba'i bar du'o/ /'di skad bstan par 'gyur te/ nyan thos dang rang sangs rgyas thams cad ni phung po lhag ma dang bcas pa dang/ phung po lhag ma med pa'i dbye bas mya ngan las 'das pa tha dad kyang phung po lhag ma med pa'i mya ngan las 'das pa kho na mngon sum du bya ba'i phyir bsgrub par bya bar zhugs pa yin no/ /\\

\textbf{\TVB}\\
gang nyan thos dang bcas pa'i 'jug pa'i rang rgyal ba ni bse ru lta bu ste/ gcig pu rgyu ba dang/ tshogs kyi spyod pa'i rang sangs rgyas dag bsgrub byar 'dod do/ /ci 'dra ba zhig ce na/ gzugs la sogs pa'i phung po dang bral ba'o/ /gzugs dang tshor ba dang 'du shes dang/ 'du byed rnams dang/ rnam par shes pa'i mtshan nyid dag gi phung po rnams ni spung bste/ bral ba dang/ rnam par chad pa dang/ 'gag par gyur pa zhes bya ba'i bar du'o/ /'di skad du bstan par 'gyur te/ nyan thos dang rang sangs rgyas thams cad ni phung po lhag ma dang bcas pa dang/ phung po lhag ma med pa'i dbye bas mya ngan las 'das pa tha dad kyang/ phung po lhag ma med pa'i mya ngan las 'das pa kho na mngon sum du bya ba'i phyir bsgrub par byed par zhugs pa yin no/ /

\subsection{Verse 5}
\subsubsection{Root text}
\begin{quote}
	ākāraśūnyaṃ gaganendurūpaṃ \\
	pratyātmavedyaṃ karuṇārasaṃ ca |\\
	sallakṣaṇair bhūṣitam\footnoteB{
		bhūṣitam] \EDD ; bhuṣitam \MS
	} arthakāri \\
	dānādiniṣyandam apetasaukhyam || 5 ||
	% Indravajra: GGLGGLLGLGG

	It is empty of forms, has the nature of the sky and the moon, is to be realised by oneself, and has compassion as its flavour.
	Adorned with all excellent characteristics, it accomplishes aims, it is the outflow of giving and so forth, and is free of bliss.

	rnam pas stong pa nam mkha' zla ba'i ngo bo nyid/ /\\
	so sor rang rig snying rje'i rang bzhin ca/ /\\
	mdzes pa'i mtshan rnams kyis brgyan don mdzad pa/ /\\
	sbyin la sogs pa'i rgyu mthun bde bral ba/ /
\end{quote}

\subsubsection{Commentary}
idānīṃ pāramitānayavādinām abhimataṃ\footnoteB{
	abhimataṃ] \EDD; abhimata \MS
} caturvidhaṃ sādhyam āha\emdash ākāraśūnyam ityādi.
ākārair nīlapītasukhaduḥkhādibhiś citrarūpaiḥ śūnyaṃ nirākāram.
ata eva gaganasyeva nirākāratvenendor iva prabhāsvaratvena rūpaṃ svabhāvo yasya tat tathā.
pratyātmavedyam iti svasaṃvedanaikavedyam.\footnoteB{
	svasaṃvedanaikavedyam] \EDD\ (\emd) (°vedyaṃ); svasaṃvedyanaikavedyaṃ \MS
}
karuṇā duḥkhād\footnoteB{
	karuṇā duḥkhād] \MS; karuṇāduḥkhā° \EDD
} duḥkhahetor vā sakalajagadatyuddharaṇakāmatā.\footnoteA{
	This definition can be found in various older sources, such as the \emph{Pramāṇavārttikavṛtti}.
}
saiva rasaḥ svabhāvo yasya tat tathoktam.
etad uktaṃ bhavati\emdash nīlapītādicitrākāraśūnyaṃ nirābhāsanirañjanaṃ\footnoteA{
	See also in \emph{Amṛtakaṇika} and \emph{Kāllotara mahātantra} for instances of the pair \emph{nirābhāsaṃ nirañjanaṃ}. One word is probably acceptable as a \emph{viśeṣaṇasamāsa}.
} gaganopamaṃ svacchaṃ sakalajagadarthakāri\footnoteA{
	sakalajagadarthakāri can also be read in compound with mahākaruṇā°. This is reflected in both Tibetan translations: \emph{'gro ba ma lus pa'i don byed pa'i snying rje chen po}
} mahākaruṇāyuktaṃ pratyātmavedyaṃ pāramitopadeśaśabdābhidheyaṃ sādhyam iti pāramitānaye prathamaṃ sādhyam.\\

\textbf{\TVA}\\
da ni pha rol tu phyin pa'i tshul du smra ba dag la mngon par 'dod pa'i bsgrub par bya ba rnam pa bzhi bstan pa'i phyir/ rnam pa stong pa zhes bya ba [\TVA\ fol.\ 206r]/ /la sogs pa smos te/ rnam pa ni sngon po dang/ ser po dang/ bde ba dang sdug bsngal la sogs pa sna chags pa'i rnam pa'o/ /des stong pa ste rnam pa med pa'o/ /rnam pa med pa de nyid kyi phyir nam mkha' lta bu'o/ /gang gi rang bzhin zla ba dang 'dra bar 'od gsal ba'i ngo bo yin pa de la de skad ces bya'o/ /so sor rang rig pa gcig pu'i ngo bo rig pa'o/ /'dir sdug bsngal dang sdug bsngal gyi rgyu dag las 'gro ba ma lus pa gdon par 'dod pa'i snying rje'i rang bzhin gang yin pa de la de skad ces bya'o/ /'di skad du bshad par 'gyur te/ pha rol tu phyin pa nye bar ston pa'i sgras brjod par bya/ sngon po dang/ ser po la sogs pa'i rnam pa sna tshogs pas stong zhing snang ba med la/ dri ma med pa nam mkha' lta bur dag pa 'gro ba ma lus pa'i don byed pa'i snying rje chen po dang ldan pa so sor rang gis rig pa'i bsgrub par bya ba yin te/ pha rol tu phyin pa'i tshul gyis bsgrub par bya ba dang po'o/ /\\

\textbf{\TVB}\\
da ni pha rol tu phyin pa'i tshul du smra ba dag mngon par 'dod pa'i bsgrub par bya ba rnam pa bzhi bstan pa'i phyir/ zhes bya ba la sogs pa smos te/ rnam pa ni [\TVB\ fol.\ 73r] sngon po dang/ rnam pas stong pa ser po dang/ bde ba dang/ sdug bsngal la sogs sna tshogs pa'i ngo bo'o/ /des stong pa ste rnam pa rnam par brdzun pa'i gzhung brdzun pa'o/ /rnam pa med pa de nyid kyi phyir nam mkha' lta bu'o/ /gang gi rang bzhin zla ba dang 'dra bar 'od gsal ba'i ngo bo nyid yin pa de la de skad ces bya'o/ /so so rang rig ces bya ba ni rang rig pa gcig pu'i ngo bo rig pa'o/ /'dir <phung po> sdug bsngal dang sdug bsngal gyi rgyu <mi dge bcu las dang nyon mongs> dag las 'gro ba ma lus par 'don par 'dod pa'i snying rje'i rang bzhin gang yin pa de la de skad ces bya'o/ /'di skad du bshad par 'gyur te/ /pha rol tu phyin pa nye bar ston pa'i sgras brjod par bya ba sngon po dang/ ser po la sogs pa'i rnam pa sna tshogs pas stong zhing snang ba med la/ dri ma med pa nam mkha' lta bur dag pa 'gro ba ma lus pa'i don byed pa'i snying rje chen po dang ldan pa so so rang gis rig pa ni bsgrub par bya ba yin te/ pha rol tu phyin pa'i tshul gyi bsgrub par bya ba dang po'o/\\

śobhanāni ca tāni lakṣaṇāni ca dvātriṃśallakṣaṇasaṃjñakāni ceti.
tair bhūṣitam.
arthaṃ janānāṃ prayojanaṃ kartuṃ śīlaṃ svabhāvo yasya tadarthakāri.
dānādīnāṃ daśapāramitānāṃ niṣyandaṃ tatprakarṣaprabhavatvena sadṛśaṃ phalam.
duḥkhasya pūrvam eva prahīṇatvāt, sākṣātkṛtāvasthāyāṃ\footnoteB{
	sākṣātkṛtāvasthāyāṃ] \EDD; sākṣātkṛtāvatāsthāyāṃ \MS
} saukhyasyāpy abhāvatvāt, upekṣārūpatvenāpetasaukhyam apagatasaukhyam.
etad uktaṃ bhavati\emdash dvātriṃśallakṣaṇadharāśītyanuvyañjanavirājitaśarīraṃ sakalajagadarthakāri dānādipāramitābhyāsa\crux balenātmānaṃ\footnoteB{
	°balenātmānaṃ] \MS\ \EDD; stobs kyis bdag nyid yang dag par rdzogs pa'i \TVA; stobs kyis byung ba yang dag par \TVB
}\crux samyaksaṃbuddharūpaṃ sukhaduḥkharahitatvenopekṣārūpaṃ dvitīyaṃ sādhyam.\\

\textbf{\TVA}\\
mdzes pa yang de nyid yin la/ mtshan yang de nyid yin pas na mdzes pa'i mtshan sum cu rtsa gnyis zhes bya ste/ des brgyan pa'o/ /don ni skye bo rnams kyis dgongs pa ste/ de mdzad pa'i ngang tshul can gyi rang bzhin gang la yod pa de ni don mdzad pa'o/ / sbyin pa la sogs pa pha rol tu phyin pa bcu'i rgyu mthun pa de ni/ rang gi mtha'i stobs kyis 'dra ba'i 'bras bu'o/ /sdug bsngal sngar spangs pa nyid kyi phyir ro/ /bde ba yang sngon du byas pa'i gnas skabs na med pa'i phyir btang snyoms kyi ngo bo nyid kyi bde ba nyid med pa ni bde bral ba'o/ /'di skad du bstan par 'gyur te/ mtshan sum cu rtsa gnyis 'dzin cing dpe byad bzang po brgyad cus rnam par spras pa'i skus/ 'gro ba ma lus pa'i don mdzad pa/ sbyin pa la sogs pa pha rol tu phyin pa goms par mdzad pa'i stobs kyis bdag nyid yang dag par rdzogs pa'i sangs rgyas kyi ngo bo nyid bde ba dang/ sdug bsngal spangs pa'i btang snyoms kyi ngo bo nyid ni bsgrub par bya ba gnyis pa yin no/ /\\

\textbf{\TVB}\\
mdzes pa <rnam pa dang bcas pa> yang de nyid yin la mtshan yang de nyid yin pas na/ /mdzes pa <rnam pa dang bcas pa> yang de nyid yin la mtshan yang de nyid yin pas na/ mdzad pa'i mtshan sum cu rtsa gnyis zhes bya sde/ des brgyan pa'o/ /don te skye bo rnams kyis dgos pa mdzad pa'i ngang tshul can gyi rang bzhin gang la yod pa de ni de'i don mdzad pa'o/ /sbyin pa la sogs pa pha rol tu phyin pa [\TVB\ 73v] bcu'i rgyu mthun pa de ni rab kyi mtha'i stobs nyid kyis 'dra ba'i 'bras bu'o/ /sdug bsngal sngar spangs pa nyid kyi phyir dang/ bde ba yang mngon sum du byas pa'i sa nas skabs na med pa'i phyir/ btang snyoms kyi ngo bo nyid kyis bde ba nyid med pa'i ni bde bral ba'o/ /'di skad du bstan par 'gyur te/ mtshan sum cu rtsa gnyis 'dzin cing/dpe byad bzang po brgyad cus rnam par spras pa'i sku 'gro ba ma lus pa'i don mdzad pa/ sbyin pa la sogs pa'i pha rol tu phyin pa goms par mdzad pa'i stobs kyis byung ba yang dag par rdzogs pa'i sangs rgyas kyi ngo bo nyid bde ba dang/ sdug bsngal spangs pa'i btang snyoms kyi ngo bo nyid bsgrub par bya ba gnyis pa yin no/ /

\subsection{Verse 6}
\subsubsection{Root text}
\begin{quote}
	sānandasallakṣaṇamaṇḍitāṅgaṃ \\
	sambhujyamānaṃ daśabhūmisaṃsthaiḥ |\\
	sattvārthakāri pravadanti sādhyaṃ \\
	dānādiṣaṭpāramitānayasthāḥ || 6 ||
	% GGLGGLLGLGG, Indravajrā

	Followers of the Way of the Six Perfections—namely, giving and the others—teach that the goal is the body adorned with the blissful excellent qualities, enjoyed by the lords of the tenth stage and accomplishing the aims of sentient beings.

	sbyin sogs pha rol [\TM\ fol.\ 203v] phyin drug tshul gnas pa/ /\\
	dka' bcas sku ni mdzes pa'i mtshan gyis brgyan/ /\\
	sa bcur bzhugs pas yang dag longs spyod pa/ /\\
	'gro don mdzad pa bsgrub byar rab tu smra/ /
\end{quote}

\subsubsection{Commentary}
[\EDD\ p.\ 135] sānandetyādi.
sahānandena vartata iti sā[\MS\ fol.\ 4r]nandam.
sānandaṃ ca tat sallakṣaṇamaṇḍitāṅgaṃ sambhujyamānaṃ dharmadeśanādvāreṇopajīvyamānam.\footnoteB{
	°opajīvyamānam] \MS\ \EDD; °opabhujyamānam \possibleemd\ (\emph{cf.} \TVA\ and \TVB: nye bar longs spyod par gyur pa'o)
}
kaiḥ?
daśabhūmīśvaraiḥ, pariśiṣṭabhūmisthitānām\footnoteB{
	pariṣiṣṭabhūmi°] \corr; pariṣiṣṭa bhumi° \EDD
} agocaratvāt.
daśabhūmiprāptair avalokiteśvaramañjuśrīprabhṛtibhir upabhujyamānam iti yāvat.
etad uktaṃ bhavati\emdash śuddhāvāsopari ghanavyūhasaṃjñake\footnoteB{
	°saṃjñake] \emd; °saṃjñako \MS; °saṃjñakaḥ \EDD\ (\emd)
} samyaksaṃbuddhabhuvane yathā bhagavān ānandarūpaḥ sambhogakāyātmā nirmāṇadvāreṇa sakalajagadarthasampādakaḥ śrāvakapratyekabuddhanavabhūmīśvarair apy adṛśyaśarīro daśabhūmīśvarair eva paramabodhisattvair\footnoteB{
	paramabodhisattvair] \conj\ (\emph{cf.}\ \TVA\ and \TVB: mchog tu gyur pa'i byang chub sems dpa'); paraṃ bodhisatvair \MS\ \EDD\ (°sattvair)
} dharmaśravaṇadvāreṇopabhujyamānam āsaṃsāraṃ cakāsti tathaiva tat sādhyam iti tṛtīyam.\\

\textbf{\TVA}\\
dga' bcas zhes bya ba la sogs pa la/ dga' ba dang lhan cig 'jug pa'i phyir de dag dga' ba dang bcas pa yin la/ mdzes pa'i mtshan rnams kyis brgyan pa'i sku yang yin pas so/ /yang dag longs spyod pa [\TVA\ fol.\ 206v]ni/ chos bstan pa'i sgo nas nye bar longs spyod par gyur pa'o/ /su zhig gis she na/ sa bcu'i dbang phyug rnams kyis te/ sa lhag ma rnams na gnas pa'i spyod yul ma yin pa nyid kyi phyir/ sa bcu brnyes pa spyan ras gzigs dang 'jam dpal la sogs pa rnams kyis longs spyod par bya ba ma yin no zhes bya ba'i bar du'o/ /lhag ma ni go sla'o/ /'di skad du bshad par 'gyur te/ gnas gtsang ma'i steng na yang dag par rdzogs pa'i sangs rgyas kyi pho brang stug po bkod pa zhes bya ba ni ji ltar bcom ldan 'das dga' ba'i ngo bo longs skyod rdzogs pa'i sku'i bdag nyid can/ sprul pa'i sku'i sgo nas 'gro ba ma lus pa'i don yang dag par sgrub par mdzad pa nyan thos dang rang sangs rgyas dang/ sa dgu'i dbang phyug rnams kyis kyang mi mthong ba'i sku can/ sa bcu'i dbang phyug mchog tu gyur pa'i byang chub sems dpa' kho nas chos gsan pa'i sgo nas nye bar longs spyad par bya ba 'khor ba ji srid nas bzhugs te/ de lta bu'i bsgrub par bya ba gsum pa yin no/ /\\

\textbf{\TVB}\\
dga' bcas <rnam bcas kyi gzhung> zhes bya ba la sogs pa la/ dga' ba dang lhan cig 'jig pa'i phyir de dga' ba dang bcas pa yang yin la/ mdzes pa'i mtshan gyis brgyan pa'i sku yang yin pas so/ /yang dag longs spyod pa ni chos ston pa'i sgo nas nye bar longs spyod par gyur pa'o/ /su zhig gis zhe na/ /sa bcu'i dbang phyug rnams kyis te/ sa lhag ma rnams na gnas pa'i spyod yul ma yin pa nyid kyi phyir/ sa bcu pa brnyes pa spyan ras gzigs dbang phyug dang/ 'jam dpal la sogs pa rnams kyis longs spyod par bya ba yin no/ /zhes bya ba'i bar du'o/ /lhag ma ni go sla'o/ /'di skad du bshad par 'gyur te/ gnas [\TVB\ fol.\ 74r] gtsang ma'i steng na yang dag par rdzogs pa'i sangs rgyas kyi pho brang stug po bkod pa zhes bya ba na/ ji ltar bcom ldan 'das dga' ba'i ngo bo longs spyod rdzogs pa'i sku'i bdag nyid can/ sprul pa'i sku'i sgo nas 'gro ba ma lus pa'i don yang dag par sgrub par mdzad pa/ nyan thos dang rang sangs rgyas dang sa dgu'i dbang phyug rnams kyis kyang ma mthong ba'i sku can/ sa bcu'i dbang phyug mchog tu gyur pa'i byang chub sems dpa' kho nas chos gsan pa'i sgo nas nye bar longs spyod par bya ba 'khor ba ji srid du bzhugs te/ de lta bu ni bsgrub par bya ba gsum pa yin no/ /

\subsection{Verse 7}
\subsubsection{Root Text}
\begin{quote}
	saṃpūrya dānādiguṇān aśeṣān \\
	saṃbuddhakṛtyaṃ\footnoteB{
		saṃbuddhakṛtyaṃ] \emd\ (\emph{cf.} TaRaA-V: saṃbuddhānāṃ \ldots\ avaśyakartavyaṃ kṛtsnaṃ); saṃbuddhya kṛtyaṃ \MS\ \EDD
	} sakalaṃ ca kṛtvā |\\
	yad bhūtakoṭeḥ karaṇaṃ ca sākṣāt \\
	sādhyaṃ tad apy asti nirodharūpam || 7 ||
	% GGLGGLLGLGG, Indravajra

	After making replete all the qualities of giving and so forth, and after completing all the tasks of a perfect buddha, there still exists, taking the form of cessation, the goal that is the direct perception of the limit of reality.

	sbyin sogs yon tan ma lus yongs rdzogs pa/ /\\
	rdzogs sangs mdzad pa ma lus mdzad nas ni/ /\\
	gang zhig yang dag mtha' ni mngon mdzad pa/ /\\
	'gog pa'i ngo bo bsgrub bya yang dag yod/ /
\end{quote}

\subsubsection{Commentary}
saṃpūryetyādi.
dānādipāramitā eva guṇā guṇyante 'bhyasyanta iti kṛtvā.
tān saṃpūrya paripūrṇaṃ kṛtvā, yat saṃbuddhānāṃ kṛtyaṃ sakalam\footnoteB{
	kṛtyaṃ sakalam] \conj ; sakalam \MS\ \EDD	
}\footnoteA{
	The manuscript reading of simply \emph{sakalaṃ} instead of \emph{kṛtyaṃ sakalam} is asymmetrical given the following gloss, \emph{avaśyakartavyaṃ kṛtsnaṃ}. Here Tib.\ reads simply \emph{nges par mdzad par bya ba ma lus pa}, reflecting only the gloss and neither \emph{sakalam} of the Sanskrit nor the conjecture \emph{kṛtyaṃ sakalam}. It is also possible that \emph{sakalam} is a mistaken scribal addition, but it's also possible that even if the Tibetan translators saw \emph{kṛtyaṃ sakalam}, they did chose not to render this because of the superfluous sounding results in Tibetan.
} avaśyakartavyaṃ kṛtsnaṃ tad api kṛtvā, bhūtakoṭeḥ śūnyatālakṣaṇāyāś cittacaitta\footnoteB{
	cittacaitta°] \EDD\ (\emd); cittacaitya° \MS
}\hspace{0em}nirodhātmikāyā yat sākṣātkaraṇaṃ tad api sādhyam astīti pāramitānayasthā evaṃ bruvate caturthaṃ sādhyam iti.\\

\textbf{\TVA}\\
sbyin sogs zhes bya ba la sogs pa la yon tan 'du byed goms par byed pa'i phyir/ spyan pa la sogs pa'i pha rol tu phyin pa nyid yon tan yin la de dag yang dag par rdzogs pa ste/ yongs su rdzogs par byas nas/ gang rdzogs pa'i sangs rgyas rnams kyis nges par mdzad par bya ba ma lus pa de yang mdzad nas/ yang dag pa'i mtha' ste gang sems dang sems las byung ba 'gog ba'i bdag nyid stong pa'i mtshan nyid mngon sum du mdzad pa'o/ /de yang bsgrub byar yod pa yin no zhes pha rol tu phyin pa'i tshul la gnas pa dag de skad smra ba bsgrub par bya ba bzhi pa yin no/ /\\

\textbf{\TVB}\\
<dbu ma pa'i gzhung> sbyin sogs zhes bya ba la sogs pa la/ yongs su yon tan du byed goms par byed pa'i phyir sbyin pa la sogs pa'i pha rol tu phyin pa nyid yon tan yin la/ de dag yang dag par rdzogs pa ste/ rdzogs par byas nas/ gang rdzogs pa'i sangs rgyas rnams kyis nges par mdzad par bya ba ma lus pa de yang mdzad nas/ yang dag mtha' ste/ gang sems dang sems las byung ba 'gog pa'i bdag nyid kyi stong pa'i mtshan nyid mngon sum du mdzad pa'o/ /de yang bsgrub byar yod pa yin no zhes pha rol tu phyin pa'i tshul la gnas pa de dag skad du smra ba bsgrub par bya ba bzhi pa'o/ /

\subsection{Verse 8}
\subsubsection{Root Text}
\begin{quote}
	svābhāṅganāśleṣi\footnoteB{
		svābhāṅganāśleṣi \EDD\ (\corr); svābhāṅgaṇāśleṣi \MS
	} janārthakāri\footnoteB{
		janārthakāri] \conj\ (Tib: 'gro ba yi don mdzad; TaRaA-V: jagadarthakāri); ta..rthakāri \MS\ (\emph{akṣara uncertain, perhaps} gna \emph{or} mva); tadarthakāri \EDD
	} \\
	duḥkhaiḥ sukhaiś caiva vimuktirūpam |\\
	aśītyanuvyañjanabhūṣitāṅgam \\
	apetakalpaṃ pravadanti sādhyam || 8 ||
	% GGLGGLLGLGL X 2
	% LGLGGLLGLGL X 2
	
	They declare that the goal is the body adorned with the eighty minor marks that, while embracing the woman who is one's personal consort and acting for the sake of people, is by nature free of pleasure and pain and devoid of conceptualisation.

	nyid mtshungs lha mos 'khyud cing 'gro ba yi/ /\\
	don mdzad bde sdug nyid bral ngo bo nyid/ /\\
	sku ni dpe byang bzang po brgyad cus brgyan/ /\\
	rtog pa dang bral bsgrub byar rab tu 'chad/ /
\end{quote}

\subsubsection{Commentary}
idānīṃ mantranayopadiṣṭaṃ saptavidhaṃ\footnoteB{
	saptavidhaṃ] \EDD\ (Tib: rnam pa bdun); caturthaṃ \MS
} sādhyaṃ kathayitum āha\emdash svābhāṅganetyādi.
svābhāṅganām\footnoteB{
	svābhāṅganām] \EDD\ (\corr); svābhāṅganām \MS
} āśleṣituṃ śīlaṃ svabhāvo yasya tat svābhāṅganāśleṣi.\footnoteB{
	svābhāṅganāśleṣi] \corr ; svābhāṅgaṇāśleṣi \MS\ \EDD
}
[\EDD\ p.\ 136] apetakalpaṃ vyapagatakalpaṃ kalpanārahitam iti yāvat.
anyat subodham.
ayam arthaḥ\emdash samāliṅgitasvābhāṅganāśleṣi jagadarthakāri\footnoteB{
	°svābhāṅganāśleṣi jagadarthakāri] \conj\ (\TVB : nyid dang mtshungs pa'i lha mos 'khyud pa can 'gro ba'i don mdzad pa); °svābhāṅganāśleṣajagadarthakāri \MS\ \EDD; nyid dang mtshungs pa'i lha mos 'khyud pa can/ 'gro ba ma lus pa'i don mdzad pa \TVA\ (°svābhāṅganāśleṣy aśeṣajagadarthakāri)
}\footnoteA{
	The compound \emph{°svābhāṅganāśleṣajagadarthakāri} is strinckly speaking not impossible, and could be read as a kind of instrumental \emph{tatpuruṣa}, for example; however, given that this is a prose explanation of the verse, there is no need for the author to use such a compound and it seems mostly likely that the scribe left off the \emph{ikāra}.
} dvātriṃśallakṣaṇavibhūṣitaśarīram\footnoteB{
	śarīram] \EDD ; śarīra \MS
} upekṣārūpaṃ\footnoteB{
	upekṣārūpaṃ] \MS\ \EDD ; btang snyoms kyi ngo bo du 'khor ba ji srid du bzhugs pa (ji srid bzhugs pa] \TVA ; ju bzhugs pa \TVB) mngon du bya ba yin no zhe bya ba TIB (upekṣārūpaṃ āsaṃsārasthāyi sākṣātkriyata iti)
} prathamaṃ sādhyam.\\

\textbf{\TVA}\\
da ni sngags kyi tshul du bshad pa'i bsgrub par bya ba rnam pa bdun bstan pa'i phyir/ nyid mtshungs lha mos zhes bya ba la sogs pa smos te/ nyid dang mtshungs pa'i lha mos 'khyud pa'i ngang tshul can gyi rang bzhin gang la mnga' ba de ni/ nyid mtshungs lha mos 'khyud pa can no/ /rtog pa dang bral ba ni rtog pa med cing spangs pa ste/ rnam par mi rtog pa zhes bya ba'i bar du'o/ /gzhan ni go sla'o/ /'di'i don ni 'di yin te/ yang dag par 'khyud pa ste/ nyid dang mtshungs pa'i lha mos 'khyud pa can/ 'gro ba ma lus pa'i [\TVA\ fol.\ 207r]/ /don mdzad pa'i sku mtshan sum cu rtsa gnyis la sogs pas rnam par brgyan pa btang snyoms kyi ngo bo nyid du 'khor ba ji srid du bzhugs pa mngon du bya ba yin no zhes bya ba bsgrub par bya ba dang po'o/ /\\

\textbf{\TVB}\\
da ni sngags kyi tshul du bshad pa'i bsgrub par bya ba rnam pa bdun bstan par bya ba'i phyir/ nyid mtshungs <a na ga badz+ra> lha mo zhes bya ba la sogs pa smos te/ nyid [\TVB\ fol.\ 74v] dang mtshungs pa'i lha mos 'khyud pa'i ngang tshul can gyi rang bzhin gang la mnga' ba de ni nyid mtshungs lha mos 'khyud pa can no/ /rtog pa dang bral ba ni rnam par rtog pa med cing rnam par rtog pa spangs pa zhes bya ba'i bar du'o/ /gzhan ni go sla'o/ /don ni 'di yin te/ yang dag par 'khyud pa ste/ nyid dang mtshungs pa'i lha <kha> mos 'khyud pa can 'gro <ga> ba'i don mdzad pa'i sku <k> mtshan sum cu rtsa gnyis la sogs pas rnam par brgyan pa btang snyoms kyi ngo bo nyid du 'khor ba ji bzhugs pa mngon sum du bya ba yin zhes bya ba ni bsgrub bya dang po'o/ /

\subsection{Verse 9}
\subsubsection{Root Text}
\begin{quote}
	svadevatākāraviśeṣaśūnyaṃ \\
	prāg eva sambhāvya sukhaṃ sphuṭaṃ sat |\\
	mahāsukhākhyaṃ jagadarthakāri \\
	cintāmaṇiprakhyam uvāca kaścit || 9 ||
	% Metre is viparītākhyānikī:
	% LGLGGLLGLGG
	% GGLGGLLGLGG

	Some say [the goal is] bliss devoid of the qualifier of the form of a person deity. It is meditated on from the very beginning, and when it becomes vivid, it is called Great Bliss, which accomplishes the aim of beings like a wish-fulfilling jewel.

	dang po nyid nas lha yi rnam pa dag /\\
	khyad par gyis stong bde ba yang dag bsgom/ /\\
	gsal 'gyur bde chen 'gro ba'i don mdzad pa/ /\\
	yid bzhin nor bur grags pa kha cig smra/ /
\end{quote}

\subsubsection{Commentary}
svadevatetyādi.
svadevatākāraviśeṣeṇa\footnoteB{
	svadevatā°] \sigmareading{\TVA}; lha \TVA\ (devatā°)
} sveṣṭadevatākāreṇa śūnyaṃ nirākāram iti yāvat.
prāg eva prathamataram\footnoteB{
	prathamataram] \MS ; prathamataro° \EDD
} upadeśānantaram\footnoteB{
	upadeśānantaram] \EDD\ (\emd); upadeśāntaram \MS
} eva devatākāranirapekṣaṃ sukhaṃ saṃbhāvya bhāvanayā sākṣātkṛtvā sphuṭaṃ\footnoteB{
	sphuṭaṃ] \MS ; \emph{deest in} \EDD ; ma gsal ba TIB 
}\footnoteA{
	The understanding reflected in \TIB , namely \emph{aphuṭaṃ} instead of \emph{sphuṭaṃ}, is an alternative word division and also yields sense.
	It seems more likely, however, that the author is glossing \emph{sphuṭaṃ}.
} sphu[\MS\ fol.\ 4v]\hspace{0em}ṭīkṛtaṃ san mahāsukhasaṃjñakaṃ bhavati.
tac ca jagadarthakāri cintāmaṇisamānarūpam.
etad uktaṃ bhavati\emdash upadeśānantaram eva mantramudrādevatākārarahitaṃ\footnoteB{
	°rahitaṃ] \sigmareading{\TVA}; spangs te/ bde ba 'ba' zhig tsam \TVA\ (°rahitaṃ sukhamātraṃ)
} bhāvanayā sphuṭīkṛtaṃ mahāsukhasaṃjñakaṃ cintāmaṇivad jagadarthakāri māyopamam āsaṃsārasthāyi dvitīyaṃ sādhyam.\\

\textbf{\TVA}\\
dang po nyid nas zhes bya ba la sogs pa la lha'i rnam pa'i khyad par te/ bdag nyid kyis 'dod pa'i lha'i rnam pas stong pa dang/ rnam pa med pa zhes bya ba'i bar du'o/ /dang po nyid nas zhes bya ba ni/ thog ma nyid nas bshad ma thag pa'i lha'i rnam pa nyid la ltos pa med par bde ba yang dag par bsgom par bya ba rnam par bsgoms pas mngon sum du bya ba gsal ba ma yin pa gsal bar byas par gyur ba'i bde ba chen po zhes bya bar 'gyur ro/ /de yang 'gro don mdzad pa ni yid bzhin gyi nor bu dang mtshungs pa'i tshul du'o/ /'di skad bshad par 'gyur te/ bshad ma thag pa nyid kyi sngags dang phyag rgya dang lha'i rnam pa nyid spangs te/ bde ba 'ba' zhig tsam bsgoms pas gsal bar byas la bde ba chen po zhes bya/ yid bzhin gyi nor bu bzhin du 'gro ba ma lus pa'i don mdzad pa sgyu ma lta bu'i 'khor ba ji srid par bzhugs pa ni bsgrub par bya ba gnyis pa'o/ /\\

\textbf{\TVB}\\
dang po nyid nas zhes bya ba la sogs pa la/ rang lha'i rnam pa'i khyad par te/ bdag nyid kyi 'dod pa'i lha rnams pas stong pa dang/ rnam pa med pa zhes bya ba'i bar du'o/ /dang po nyid nas zhes bya ba ni thog ma nyid nas bshad ma thag pa'i lha'i rnam pa nyid la bltos pa med <shar ba> par bde ba yang dag par bsgom par bya ba bsgom pas/ mngon du bya ba gsal ba ma yin pa gsal bar byas par gyur pa ni bde ba chen po zhes bya bar 'gyur ro/ /de yang 'gro don mdzad pa ni yid bzhin gyi nor bu dang mtshungs pa'i tshul du'o/ /'di skad du bshad par 'gyur te/ bshad ma thag pa'i nyid kyi sngags <sa bon las/ lha'i sku> dang phyag rgya dang lha'i rnam pa nyid spangs ste/ bsgoms pas gsal por [\TVB\ fol.\ 75r] byas pa bde <ga> ba chen po zhes bya ba yid bzhin gyi nor <ga> bu bzhin du 'gro ba'i don mdzad pa/ sgyu ma lta bu 'khor ba ji srid <ga> par bzhugs pa ni bsgrub par bya ba gnyis pa'o/ /

\subsection{Verse 10}
\subsubsection{Root Text}
\begin{quote}
	kṛtvā sākṣāt svādhipaṃ [\MS\ fol.\ 1v] sātarūpaṃ \\
	paścāt tyaktvā sātamātraṃ phalaṃ syāt |\\
	śuddhaṃ sākṣāc chakyate naiva kartuṃ \\
	tenākāro bhāvitaḥ svādhipasya || 10 ||
	% Śālinī
	% GGGGGLGGLGG
	% GGGGGLGGLGG
	% GGGGGLGGLGG
	% GGGGGLGGLGL

	After directly perceiving one's lord with the nature of bliss, one then abandons it for a fruit that is mere bliss.
	The pure cannot be directly perceived; therefore, the form of one's lord is to be meditated on.

	rang bdag bde ba'i ngo bo mngon byas nas/ /\\
	phyes btang bde btsam zhig 'bras 'gyur ba'o/ /\\
	dag pa mngon sum byed par nus med nas/ /\\
	de phyir bdag po'i rnam pa bsgom pa yin/ /
\end{quote}

\subsubsection{Commentary}
kṛtvetyādi.
svādhipaṃ sveṣṭadaivataṃ sākṣātkṛtvāmukhīkṛtya sātarūpaṃ sukhaikasvabhāvam, paścād devatākāraṃ parityajya sukhamātraphalaṃ sādhyaṃ vyavasthitaṃ syāt.
nanu\footnoteB{
	nanu] \MS\ \EDD\ ; gal te \TVA\ ([nanu] yadi) ; \TVB : \emph{not clearly rendered}
} sākṣātkṛtvāpi devatākāras tyaktavyaḥ.
tarhi prathamam eva kasmād [\EDD\ p.\ 137] vibhāvitaḥ.
sukhamātram eva dvitīyasādhyavat kiṃ na vibhāvitaḥ?\footnoteB{
	vibhāvitaḥ] \EDD\ (\emd); vibhāgato \MS
}
kiṃ vṛthāprayāsenety\footnoteB{
	vṛthāprayāsenety] \EDD ; vyathāprayāsenety \MS
} āha\emdash śuddham ityādi.
śuddhaṃ kevalaṃ devatākāravirahitaṃ sukhamātraṃ naiva sākṣātkartuṃ śakyate, ākārarahitasya sukhasyānupalambhāt.
tasmāt tena kāraṇenākāro bhāvitaḥ svādhipasyeti tṛtīyam.\footnoteB{
	tṛtīyam] \emd\ \TVB\ (gsum pa yin no) ; tṛtīyaḥ \MS\ \EDD ; bsgrub par bya ba gsum pa yin no \TVA\ (tṛtīyaṃ sādhyam)
}
ayam arthaḥ\footnoteB{
	arthaḥ] \EDD ; artha \MS
}\emdash devatākārasaṃvalitam eva sukhaṃ vibhāvya sākṣādbhūte devatākāraṃ tyaktvā sukhamātram eva sādhyam uktaguṇam.\\

\textbf{\TVA}\\
rang bdag ces bya ba la sogs pa smos pa la rang gi bdag po ni rang gi 'dod pa'i lha nyid do/ /bde ba'i ngo bo nyid ni bde ba gcig pu'i rang bzhin no/ /mngon byas pa ni kun nas mngon sum du byas pa yin no/ /phyis lha'i rnam pa nyid yongs su btang nas sgrub pa'i 'bras bu bde ba tsam la gnas par gyur pa'o/ /gal te mngon sum du byas nas kyang lha'i rnam pa'i ngo bo nyid gtang bar bya ba yin pa de lta na ni/ dang po nyid ci'i phyir rnam par sgom par byed de/ bsgrub par bya ba gnyis pa bzhin du bde ba tsam nyid ci ste bsgom par bya ba ma yin nam ngal ba don med pas ci zhig bya zhe na/ dag pa zhes bya ba la sogs pa smos te/ dag pa 'ba' zhig lha'i rnam pa dang/ rnam par bral ba'i bde ba tsam kho na mngon sum du byed mi nus te/ rnam pa dang bral ba'i bde ba 'ba' zhig mi dmigs pa'i phyir ro/ /rgyu de'i phyir rang gi bdag po'i rnam pa sgom par byed pa ste/ bsgrub par bya ba gsum pa yin no/ /de'i don ni 'di yin te/ lha nyid kyi rnam pa dang ldan pa'i bde ba kho na rnam par bsgom par bya ba mngon sum du byas nas lha'i rnam pa nyid [\TVA\ fol.\ 207v] btang ste/ bde ba tsam kho na bsgrub par bya ba yon tan du 'chad do/ /\\

\textbf{\TVB}\\
rang bdag <rad na a kar shan ti> ces bya ba la sogs pa smos pa la/ rang gi bdag po ste/ rang gi 'dod pa'i lha nyid dang/ bde ba'i ngo bo ni bde ba gcig pu'i rang bzhin no/ /mngon byas pa ni kun nas mngon sum du byas pa yin no/ /phyis lha'i rnam pa nyid yongs su btang nas bsgrub par bya ba'i 'bras bu bde ba tsam la rnam par gnas par 'gyur ro/ /mngon sum du byas nas kyang lha'i rnam pa ngo bo nyid btang bar bya ba yin pa de lta na ni/ dang po nyid ci'i phyir rnam pa bsgom par byed de/ bsgrub par bya ba gnyis pa bzhin du bde ba tsam nyid ci ste bsgom par mi byed ma yin nam/ ngal ba don med pas ci zhig bya zhe na/ dag pa zhes bya ba la sogs pa smos te/ dag pa <bde ba> 'ba' shig lha'i rnam pa dang rnam par bra'i ba'i bde ba tsam kho nas mngon sum du byed mi nus te/ rnam pa dang bral ba'i bde ba mi dmigs <bde ba'i ngo bo> pa'i phyir ro/ /rgyu de'i phyir rang gi bdag po rnam par bsgom par byed pa zhes bya ba gsum pa yin no/ /don ni 'di yin te/ lha nyid kyi rnam par dang ldan pa'i bde ba kho na rnam par bsgom par bya ba mngon sum du byas nas lha'i rnam pa nyid btang ste/ bde ba <kha> tsam kho na sbgrub par bya ba [\TVB\ fol.\ 75v] yon tan can du 'chad do/ 

\subsection{Verse 11}
\subsubsection{Root Text}
\begin{quote}
	gagaṇasamaśarīraṃ lakṣaṇair bhūṣitāṅgaṃ \\
	nirupamasukhapūrṇaṃ\footnoteB{
		nirupama°] \EDD ; nirupama° \MS
	} svābhayā saṃgataṃ ca |\\
	sphuradamitamunīndraiḥ\footnoteB{
		munīndraiḥ] \emd ; munīndraḥ \MS\ \EDD
	} sarvasattvārthakāri \\
	pravadati punar anyaḥ sādhyam ucchedaśūnyam || 11 ||
	% Mālinī, LLLLLLGGGLGGLGG

	Another says that the goal has a body equal to space, a body adorned with the marks, filled with incomparable bliss, joined with his person consort, accomplishing the aims of all beings with limitless Buddha-sages who shoot forth, free of eradication.

	sku ni nam mkha' dang mtshungs yan lag mtshan gyis brgyan/ /\\
	dpe med bde bas gang ba nyid dang mtshungs par 'dus pa dang/ \\
	thub pa'i dbang po 'phro bas bsams can rnams kyi don mdzad pa/ /\\
	chad pas stong pa bsgrub bya nyid du gzhan dag rab tu smra/ /
\end{quote}

\subsubsection{Commentary}
gagaṇetyādi.
gagaṇasamaṃ māyopamaṃ vicārāsahaṃ\footnoteB{
	māyopamaṃ vicārāsahaṃ] \MS\ (\emph{reading slightly unclrear}) ; māyopamavicārasaha \EDD
} śarīraṃ yasya.
lakṣaṇair dvātriṃśadbhir aśītibhiś cānuvyañjanair maṇḍitāny aṅgāni yasya.
nirupamaiḥ sthaulya\footnoteB{
	sthaulya°] \MS\ \EDD ; rgya nam pa nyid dang/ rgya che ba nyid dang \TVA\ (praṇītatvasthaulya°); lhun che ba nyid dang/ \TVB\ (sthaulya ?)
}nairantaryā\footnoteB{
	°nairantaryā°] \EDD\ (\emd); °nairuttaryā° \MS
}saṃsāra\footnoteB{
	°āsaṃsāra°] \emd ; °āsaṃsāraṃ \EDD
}pravāhitvanirāsravatvādibhir upamābhāvād upamātikrāntaiḥ sukhaiḥ pūrṇaṃ romāgraparyantaṃ\footnoteB{
	\conj\ (\TIB : gang ba ni/ ba spu rtse mo'i mthar thug pa) ; pūrṇṇaṃ masimāgrapayantaṃ \MS ; pūrṇatāṃ samāśrayantaṃ \EDD ;  \TVA\ (pūrṇaṃ romāgraparyantaṃ)
} saṃpūrṇam.
svābhayā ca tathābhūtayā saṃgataṃ samāliṅgitam.
sphuradbhir anantanirmitair munīndrais tathābhūtair eva sarvasattvārthakāri.\footnoteB{
	sarvasattvārtha°] \MS\ \EDD\ (\TVB : sems can thams cad kyi don); sems can gyi don \TVA\ (sattvārtha°)
}
ucchedeneti nirodhena śūnyam tucchaṃ riktam.\footnoteB{
	tucchaṃ riktaṃ \MS ; bhūsthaṃ riktam \EDD ; spangs pa’o \TIB\ (tucchaṃ / riktaṃ)
}\\

\textbf{\TVA}\\
sku ni bya ba la sogs pa la gang gi sku nam mkha' dang mtshungs pa'i sgyu ma dang 'dra bar dpyad mi bzod pa'o/ /gang gis yan lag mtshan gyi ste/ mtshan sum cu rtsa gnyis dang/ dpe byang chub bzang po brgyad cus spras par gyur pa'o/ /dpe med ces bya ba ni gya nom pa nyid dang/ rgya che ba nyid dang/ bar med pa nyid dang/ 'khor ba ji srid par rab tu 'jug pa nyid dang zag pa med pa nyid la sogs pa rnams kyis dpe med pa'i phyir ro/ /dpe las 'das pa'i bde bas gang ba ni/ ba spu'i rtse mo'i mthar thug par gyur pa'o/ / nyid dang mtshungs pa yang de lta bur gyur pas 'dus pa ste/ kun nas 'khyud par gyur pa'o/ /thub pa'i dbang po'i dpag tu med pa 'phro bar gyur pa de yang de lta bu kho nas sems can gyi don mdzad pa'o/ /chad pas zhes bya ba ni/ 'gog pa stong pa ste spangs pa'o/ /\\

\textbf{\TVB}\\
sku ni zhes bya ba la sogs pa la/ <rang gi 'dod pa> gang gi sku nam mkha' dang mtshungs pa ni <'ja' tshon lta bur> sgyu ma dang 'dra bar dpyad mi bzod pa'o/ /gang gi yan lan mtshan gyis te/ mtshan sum cu rtsa gnyid dang dpe byad bzang po brgyad cus spras par gyur pa'o/ dpe med ces bya ba ni lhun che ba nyid dang/ bar med pa nyid dang/ 'khor ba ci srid par rab tu 'jug pa nyid dang/ zag pa med pa nyid la sogs pa rnams kyis dpe med pa'i phyir/ dpe las 'das pa'i bde bas gang ba ni/ ba spu'i rtse mo'i mthar thug par yang dag par gang bar gyur pa'o/ /nyid dang mtshungs pa yang de lta bur gyur pas 'dus pa ste kun nas 'khyud par 'gyur pa'o/ /thub pa'i dbang po dpag tu med pa 'phro ba yang de lta bur gyur pa kho nas sems can thams cad kyi don mdzad pa'o/ /chad pas zhes bya ba ni 'gog pas sgong pa ste spangs pa'o/ /\\

etad uktaṃ bhavati\emdash gaganamāyāmarīci\footnoteB{
	māyāmarīci] \MS\ \EDD\ (\TVB : sgyu ma dang/ smig rgyu dang/) ; sgyu ma dang/ smig rgyu dang/ smig rgyu dang/ \TVA\ (māyāmarīcīndrajāla / māyendrajālamarīci)
}\hspace{0em}gandharvanagarodakacandrapratibimbasvapnopamam\footnoteB{
	°svapnopayam] \EDD ; svapnāpayaṃ \MS
} [\MS\ fol.\ 5r] ekānekabhāvābhāvagrāhyagrāhakasvabhāvarahitam anādyantam aśeṣavastusaṃdohasvabhāvam\footnoteB{
	anādyantam aśeṣavastusaṃdohasvabhāvam] \MS\ \EDD; thog ma dang tha ma med pa’i dngos po ma lus pa’i rang bzhin \TVA\ \TVB\ (anādyantāśeṣavastusvabhāvam)
} anābhāsaṃ nirañjanaṃ sarvopamātikrāntaṃ paramasūkṣmātigambhīraprajñārūpatayā dharmakāyasvabhāvam, dvātriṃśallakṣaṇavibhūṣitaśarīram aśītyanuvyañjanavirājitagātraṃ\footnoteB{
	°gātraṃ] \MS\ \EDD ; \emph{deest} in \TVA\ and \TVB
} paramaśṛṅgārayauvanādyupetaṃ svābhāṅganāliṅgitāṅgaṃ rūpavajrāditārāparyantadevīgaṇair anantaprabhedānimittarati\footnoteB{
	°ānimittarati°] \conj\ (\TVA : mtshan ma med pa'i dga' ba'i); °ānimittārati° \MS \EDD ; mtshan ma med pa'i \TVB
}svarūpaparamānandopabhogadvāreṇa pratibimbavat [\EDD\ p.\ 138] sambhujyamānaṃ karuṇāsaṃvalitodārarūpatayā sambhogakāyarūpam, nānādhimuktivineyajanaparipācanārtham % EDD misreports MS as reading paripāvanārtha
anekavidhaprātihāryadvāreṇa\footnoteB{
	anekavidhaprātihārya°] \MS\ \EDD ; rdzu 'phrul dang cho 'phrul rnam pa du ma \TVA\ \TVB\ (anekaṛddhiprātihārya°)
} nirmitānantakulāntarbhūtasaṃbuddhabodhispharaṇasaṃhārakāritvena nirmāṇakāyātmakam, śūnyatākaruṇābhinnabodhicitta\footnoteB{
	°bodhicitta°] \EDD; °bodhicittā° \MS
}\hspace{0em}svabhāvāmalaprajñopāyasamādhisambhūtasatsukhāpūrṇam\footnoteA{
	See Sahajavilāsa, \emph{Svādhiṣṭhānakurukullāsādhana} (SāMā no.\ 183, p.\ 383): \emph{tataḥ prajñopāyāmalasamādhisambhūtasatsukhāpūrṇam iva svadehaṃ trailokya ca paśyet}.
} āsaṃsārasthitidharmaṃ\footnoteB{
	\conj\ (cf.\ Tib: chos can) ; dharmāṇāṃ \MS\ \EDD	
} apratiṣṭhitanirvāṇarūpaṃ nirmalanivātaniścalapradīpaśikhāprabandhanityatayā nirodhaśūnyaṃ caturthaṃ\footnoteB{
	caturthaṃ] \EDD ; caturtha \MS
} sādhyam.\\

\textbf{\TVA}\\
'di skad bshad par 'gyur te/ nam mkha' dang/ sgyu ma dang/ mig 'phrul dang/ smig rgyu dang/ dri za'i grong khyer dang/ chu zla dang/ gzugs brnyan dang/ rmi lam dang 'dra bas gcig dang/ du ma dang/ dngos po dang/ dngos po med pa dang/ gzung ba dang 'dzin pa'i rang bzhin dang bral la/ thog ma dang tha ma med pa'i dngos po ma lus pa'i rang bzhin snang ba med pas/ dri ma med pa dpe thams cad las shin tu 'das pa/ mchog tu phra zhing shin tu zab pa/ shes rab kyi ngo bo nyid kyis chos kyi sku'i rang bzhin yin pa dang/ sku mtshan sum cu rtsa gnyis kyis brgyan zhing/ dpe byad bzang po brgyad cus rnam par sbas pa/ shin tu sgeg cing na tshod la sogs pa dang ldan pa nyid dang/ nyid dang mtshungs pa'i lha mos sku la 'khyud cing/ gzugs rdo rje la sogs pa nas/ sgrol ma'i mthar thug pa'i lha mo'i tshogs kyis mtshan ma med pa'i dga' ba'i rang gi ngo bo'i rab tu dbye ba dpag tu med pas mchog tu dga' ba la nye bar longs skyod pa'i sgo nas/ gzugs brnyan dang 'dra bas yang dag par longs spyod pa ni snying rje'i rang bzhin rgya che ba nyid kyis longs spyod rdzogs pa'i sku'i ngo bo nyid dang/ gdul bya'i skye bo mos pa sna tshogs pa yongs su smin par bya ba'i don du/ rdzu 'phrul dang cho 'phrul rnam [\TVA\ fol.\ 208r]/ /pa du ma'i sgo nas rigs kyi nang du chud pa'i rdzogs pa'i sangs rgyas dang/ byang chub sems dpa' la sogs pa'i sprul pa dpag tu med pa spro ba dang/ bsdu ba mdzad pa nyid kyis ni sprul pa'i sku'i mtshan nyid dang/ stong pa nyid dang/ snying rje tha mi dad pa'i byang chub sems kyi rang bzhin dri ma med pa shes rab dang thabs kyi ting nge 'dzin las yang dag par 'byung ba'i bde ba dam pas gang ba/ 'khor ba ji srid par bzhugs pa'i chos can mi gnas pa'i mya ngan las 'das pa'i ngo bo mar me rlung gis ma bskyod pa'i rtse mo bzhin du dri ma med pa'i rgyun gyi rtag pa nyid kyis 'gog pas stong pa ni bsgrub par bya ba bzhin pa'o/ /\\

\textbf{\TVB}\\
'di skad du bshad par 'gyur te/ nam mkha' dang sgyu ma dang/ smig rgyu dang/ dri za'i khrong khyer dang/ chu zla dang gzugs brnyad dang/ rmi lam dang 'dra bar gcig dang/ du ma dang/ dngos po dang/ dngos po med pa dang/ gzung ba dang/ 'dzin pa'i rang bzhin bral ba/ thog ma dang tha ma med pa'i dngos po ma lus pa'i rang bzhin snang ba med pa dri ma med pa dpe thams cad las shin tu 'das pa/ mchog tu phra zhing shin [\TVB\ fol.\ 76r] tu zab pa shes rab kyi ngo bo nyid kyis chos kyi sku'i rang bzhin yin pa dang/ sku <ka> mtshan sum cu rtsa gnyis kyis brgyan cing dpe byad bzang po brgyad cus rnam par spras pa mchog tu sgeg cing na tshod la sogs pa dang ldan pa dang/ sku nyid dang mtshungs pa'i lha <kha> mos 'khyud cing/ gzugs rdo rje la sogs pa nas sgrol ma'i mthar thug pa'i lha mo'i tshogs kyis mtshan ma med pa'i rang gi ngo bo'i rab tu dpag tu med pas mchog tu dga' ba la nye bar longs spyod pa'i sgo nas gzugs brnyan dang 'dra bas yang dag par longs spyod pa ni snying rje'i rang bzhin rgya che ba nyid kyis longs spyod rdzogs pa'i ngo bo nyid dang/ gdul bya'i skye bo mos pa sna tshogs pa yongs su smin par bya ba'i don du rdzu 'phrul dang cho 'phrul rnam pa du ma'i sgo nas rigs kyi nang du chud pa'i rdzogs pa'i sangs rgyas dang byang chub sems dpa'i sprul pa dpag tu med pa spro ba dang/ bsdu ba mdzad pa nyid kyis na sprul pa'i sku'i bdag nyid dang/ stong pa nyid dang snying rje dang tha mi dad pa'i byang chub kyi sems kyi rang bzhin dri ma med pa shes rab dang thabs kyi ting nge 'dzin las yang dag par byung ba'i bde <ga> ba dam pas gang 'khor ba ji srid <ja> par bzhugs pa'i chos can/ mi gnas pa'i mya ngan las 'das pa'i ngo bo dang/ mar me rlung gis ma bskyod pa'i rtse mo bzhin du dri ma [N-D fol.\ 76v] med pa'i rgyun gyis rtag pa nyid kyi <cha> 'gog pas stong pa ni bsgrub par bya ba bzhi pa'o/ /

\subsection{Verse 12}
\subsubsection{Root Text}
\begin{quote}
	kṛtvā sākṣāt svādhipaṃ sātarūpaṃ \\
	tyaktvopekṣājñānamātraṃ\footnoteB{
		tyaktvopekṣā°] \MS\ (\emph{\EDD\ reports as \emph{tyajyo°}, but it cannot be; see commentary}); bhāvopekṣā° \EDD\ (\emd); not reflected in \TM
	} phalaṃ syāt |\\
	āsaṃsārasthāyi sattvārthakāri \\
	cintā\footnoteB{
		cintā°] \MS\PCreading\ \EDD ; cittā° \MS\ACreading
	}ratnaprakhyam\footnoteB{
		°prakhyam] \EDD ; °prakhyaṃm \MS
	} ekāntaśāntam || 12 ||
	
	After one directly perceives and then relinquishes one's lord with his blissful nature, there comes as fruit mere indifferent awanreness.
	It remains as long as there is \emph{saṃsāra} and accomplishes the aims of beings.
	Known as the wish-fulfilling jewel, it is entirely at peace.

	rang bdag bde ba'i ngo bo mngon byas nas/ /\\
	btang snyoms ngo bo ye shes tsam 'bras bu/ /\\
	'khor ba srid zhugs sems can don mdzad pa/ /\\
	yid bzhin nor grags gcig tu zhi gyur pa'o/ /
\end{quote}

\subsubsection{Commentary}
kṛtvetyādi.
sākṣāt svādhipaṃ kṛtvā, paścāt\footnoteB{
	paścāt] \EDD ; paścāta \MS
} tyaktvā, upekṣārūpaṃ yaj jñānaṃ tanmātraṃ sādhyaṃ syāt.
anyat sugamam.\footnoteB{
	sugamaṃ] \EDD ; sūgamaṃ \MS
}
etad uktaṃ bhavati\emdash maṇḍalacakrarūpaṃ sākṣātkṛtvā, paścāt tan nirodhya, upekṣājñānamātraṃ sādhyaṃ syāt pañcamam.\\

\textbf{\TVA}\\
rang bdag ces bya ba la sogs pa la rang gi bdag po mngon sum du byas nas phyis bstan ste/ btang snyoms kyi ngo bo gang yin pa'i ye shes de tsam zhig bsgrub byar 'gyur ro/ /gzhan ni rtogs par sla'o/ /'di skad bshad par 'gyur te/ dkyil 'khor gyi 'khor lo'i ngo bo mngon sum du byas nas/ phyis de bkag cing btang snyoms kyi ye shes tsam du gyur pa ni bsgrub par bya ba lnga pa'o/ /\\

\textbf{\TVB}\\
rang bdag ces bya ba la <zla grags> kyi gzhung sogs pa la/ rang gi bdag po mngon sum du byas nas phyis btang ste/ btang snyoms kyi ngo bo gang yin pa'i ye shes de tsam zhig bsgrub byar 'gyur ro/ /gzhan ni rtogs par sla'o/ /'di skad du bshad par 'gyur te/ dkyil 'khor gyi 'khor lo'i ngo bo mngon sum du byas nas phyis de bkag cing btang snyoms kyi ye shes tsam du 'gyur ba ni bsgrub par bya ba lnga pa'o/ /

\subsection{Verse 13}
\subsubsection{Root Text}
\begin{quote}
	kṛtvā sākṣān maṇḍalaṃ sātarūpaṃ \\
	paścāt tasya svecchayā nirvṛtiś\footnoteB{
		nirvṛtiś] \MS ; nirvṛtiṃ] \EDD 
	} ca|\\
	sattvārthasyāpy asty abhāvo na vāsmin \\
	prādurbhāvo nirvṛtād\footnoteB{
		nirvṛtād] \EDD ; nivṛtād \MS
	} asti yasmāt || 13 ||
	% Śālinī, GGGGGLGGLGG

	After directly perceiving the \emph{maṇḍala} whose nature of pleasure, there is later its cessation based on one's desire.
	Neither is there here an absence of [accomplishing] the aims of beings, since from cessation there is manifestation.

	dkyil 'khor bde gang mngon sum byas nas ni/ /\\
	phyis ni de rang 'dod pas nges ldog pa/ /\\
	sems can don yang yod de 'dir med min/ /\\
	gang phyir nges par ldog pas rab tu byung/ /
\end{quote}

\subsubsection{Commentary}
kṛtvetyādi.
kṛtvā sākṣān maṇḍalaṃ sātasaṃvalitam.\footnoteB{
	sātasaṃvalitam] \emd\ (\TIB : bde ba'i rang bzhin can); sātaṃ saṃvalitaṃ \MS\ \EDD
}
tasya svecchayā nirvṛtir nirodhaḥ.
nanu yadi sākṣātkṛtvāpi paścāt svecchayā nirodhayita[\MS\ fol.\ 5v]vyam,\footnoteB{
	nirodhayitavyam] \conj ; nirodhayitavyaḥ \MS\ \EDD
}\footnoteA{
	The word \emph{nirodhayitavya} seems to only make sense if it qualifies \emph{maṇḍala}, sharing the same agent as \emph{kṛtvā}. The manuscript reading \emph{nirodhayitavyaḥ}, therefore, is probably to be discarded.
} tadā karuṇāyā anekakālābhyastāyā abhāvaḥ syāt.
tasyāś cābhāvāt sattvārthābhāvaḥ [\EDD\ p.\ 139] syād ity āśaṅkyāha\emdash sattvārthasyāpy asty abhāvo na vetyādi.
asmin pakṣe sattvārthābhāvo nāsti, yasmān nirvṛtāc cakrāt karuṇāsaṃvalitāt sattvārthasya prādurbhāvo 'sti.\footnoteA{
	\TIB\ suggests reading \emph{karuṇāsaṃvalitasya}: ’gags pa’i ’khor lo las snying rje’i rang bzhin can sems can gyi don (’gags pa’i] \TVB ; ’gog pa’i \TVA)
}\\

\textbf{\TVA}\\
dkyil 'khor zhes bya ba la sogs pa la/ dkyil 'khor bde ba'i rang bzhin can mngon du byas nas/ de'i rang gi 'dod pa nges par ldog pa ni 'gog pa'o/ /gal te mngon sum du byas nas kyang/ phyis rang gi 'dod pas 'gog par bya ba yin na ni de'i tshe yun ring por goms par byas pa'i snying rje med par 'gyur ba ma yin nam/ de yang med pa'i phyir sems can gyi don med par 'gyur ro zhes dogs pa la/ sems can don yang yod de 'dir med min/ /zhes bya ba smos te/ phyogs 'di la sems can gyi don byed pa med pa ma yin te/ gang gi phyir 'gog pa'i 'khor lo las snying rje'i rang bzhin can sems can gyi don rab tu 'byung ba yod pa yin no/ /\\

\textbf{\TVB}\\
dkyil 'khor zhes bya ba la sogs pa la/ dkyil 'khor bde ba'i rang bzhin can mngon sum du byas nas/ de rang gi 'dod pa las nges par ldog pa ni 'gog pa'o/ gal te mgun sum du byas nas kyang phyis rang gi 'dod pas 'gog par bya ba yin na ni/ de'i tshe yun ring por goms par byas pa'i snying rje med par 'gyur ma yin nam/ de yang med pa'i phyir sems can gyi don med par 'gyur zhes dogs pa la/ sems can don yang yod de 'dir med min zhes bya ba smos te/ phyogs 'di la sems can gyi don byed pa med pa ma yin te/ gang gi phyir 'gags pa'i 'khor lo las snying rje'i rang bzhin can sems can gyi don tu rab tu 'byung ba yod pa yin no/ /\\

etenaitad evāha\emdash sātasaṃpūrṇacakraṃ sākṣātkṛtvā, yāvadiṣṭaṃ kālaṃ vyavasthāpya, paścāt tasya sarvathaiva pradīpavan nirodhaṃ kṛtvā sthātavyam.
yadā punaḥ sattvārthābhilāṣo bhavati, tadā niruddhād eva cakrāntaram utpādya sattvārthaḥ kartavyaḥ.
cakrāntarotpāde\footnoteB{
	cakrāntarotpāde] \EDD ; cakrāntaropāde \MS
} 'pi ciraniruddhād\footnoteB{
	ciraniruddhād] \emd (\TIB : rin du 'gags pa'i); citaniruddhād \MS ; cittaniruddhād \EDD
} eva cakrād yathābhavyatayā\footnoteB{
	yathābhavyatayā] \emph{variant word division in} \EDD : yathā bhavyatayā
} vineyānāṃ yathābhilaṣitaprāptir bhavatīti ṣaṣṭham.\\

\textbf{\TVA}\\
de ni 'di skad du bstan te/ bde bas yang dag par gang ba'i 'khor lo mngon sum du byas nas/ dus ji srid du 'dod par rnam par gnas pa ste/ phyis de rnams thams cad du mar me bzhin du 'gog pa yin no/ /yang gang gi tshe sems can gyi don mngon par 'dod par gyur na/ de'i 'gags pa nyid las 'khor lo'i khyad par 'byung zhing sems can [\TVA\ fol.\ 208v] gyi don mdzad do/ /'khor lo'i khyad par 'byung ba yang ring du 'gags pa'i 'khor lo'i kho na las skal ba ji lta ba bzhin du gdul bya rnams kyis mngon par 'dod ba rnyed par 'gyur ro zhes bya ba ni bsgrub par bya ba drug pa'o/ /\\

\textbf{\TVB}\\
de ni 'di skad du bstan te/ bde bas yang dag par gang ba'i 'khor lo mngon sum du byas nas dus ji srid 'dod [\TVB\ fol.\ 77r] par rnam par gnas te phyis de rnam pa thams cad du mar me bzhin du 'gog pa yin no/ /yang gang gi tshe sems can gyi don mngon par 'dod par 'gyur na/ de tshe 'gag pa nyid las 'khor lo'i khyad par 'byung zhing sems can gyi don mdzad pa'am/ 'khor lo'i khyad par 'byung ba yang/ ring du 'gags pa'i 'khor lo kho na las skal ba ji lta ba bzhin du gdul bya rnams kyis mngon par 'dod pa rnyed par 'gyur ro zhes bya ba ni drug pa'o/ /

\subsection{Verse 14}
\subsubsection{Root Text}
\begin{quote}
	kṛtvā sphuṭaṃ rūpam abhīṣṭam eṣāṃ \\
	paścān nirodhaṃ phalam āha kaścit |\\
	abhinnarūpaś ca yato nirodho \\
	na pakṣabhede 'pi tato 'sti bhedaḥ || 14 ||\\

	de rnams las 'dod ngo bo gsal byas la/ /\\
	phyir ni 'gog pas 'bras bu kha cig smra/ /\\
	gang phyir 'gog pa'i ngo bo tha dad min/ /\\
	de phyir tha dad phyogs kyi'am tha dad med/ / 
\end{quote}

\subsubsection{Commentary}
kṛtvetyādi | ṣaṇṇāṃ pakṣāṇām arthaphalasya sādhyatvād yad yad evābhīṣṭaṃ tad eva sākṣātkṛtvā paścāt sarvathaiva pradīpavan nirodhaḥ | uttarakālaṃ sattvārthādiśūnyaḥ sākṣātkartavyaḥ | nanu ṣaṭpakṣabhedena ṣaḍ eva nirodhāḥ syuḥ? tat katham eka eva nirodha ity āśaṅkayāha\emdash abhinnetyādi | abhinnaṃ rūpaṃ yasya tat tathā | na hi nirodhānāṃ ṣaṭpakṣalakṣaṇabhede 'pi bhedo 'sti, abhāvaikarūpatayā nirodhasya samānatvāt | ayam arthaḥ\emdash anyatamapakṣaṃ sākṣātkṛtvā paścāt tasya santānocchedarūpo nirodha iti saptamaṃ sādhyam || 14 ||\\

\textbf{\TVA}\\
de rnams las zhes bya ba la sogs pa smos pa yin te/ phyogs drug po rnams kyi nang nas 'bras bu bsgrub bya gang kho na mngon sum du byas shing de'i 'og tu rnam pa thams cad du mar me bzhin du 'gog pa dus gzhan na sems can gyi don la sogs pas stong pa mngon sum du byas pa yin no/ /phyogs drug gi dbye bas 'gog pa yang drug kho nar 'gyur ba ma yin nam 'gog pa de ji ltar na gcig kho nar 'gyur zhes dogs pa la/ gang phyir zhes bya ba la sogs pa smos te/ tha mi dad pa'i ngo bo gang yin pa de la de skad bya ste/ phyogs drug gi mtshan nyid kyi dbye bas kyang 'gog pa rnams la tha dad pa yod pa ma yin te/ 'gog pa ni dngos po med pa'i ngo bo nyid du gcig cing mtshungs pa'i phyir ro/ /don ni 'di yin te/ phyogs gzhan zhig mngon sum du byas nas/ de'i 'og tu de'i rgyud rgyun chad pa'i ngo bo 'gog pa ni bsgrub par bya ba bdun pa'o/ /\\

\textbf{\TVB}\\
de rnams las zhes bya ba la sogs pa smos pa la/ phyogs drug po rnams kyi nang nas 'bras bu bsgrub bya gang kho na mngon par 'dod pa de kho na nyid mngon sum du byas shing/ de'i 'og tu rnam pa thams cad du mar me bzhin du 'gog pa dus gzhan na sems can gyi don la sogs pas stong pa mngon sum du bya ba yin no/ /phyogs drug gi dbye bas 'gog pa yang drug kho nar 'gyur ba ma yin nam/ 'gog pa de ji ltar na gcig kho nar 'gyur zhes dogs pa la/ gang phyir zhes bya ba la sogs pa smos te/ tha mi dad pa'i ngo bo gang yin pa de la de skad ces bya ste/ phyogs drug gis mtshan nyid kyi dbye bas kyang 'gog pa rnams la tha dad pa yod pa ma yin te/ 'gog pa ni dngos po med pa'i ngo bo nyid du gcig cing mtshungs pa nyid kyi phyir ro/ /don ni 'di yin te/ phyogs gzhan zhig mngon sum du byas nas/ de'i [\TVB\ fol.\ 77v] 'og tu de'i rgyun chad pa'i ngo bo 'gog pa ni bsgrub par bya ba bdun pa'o/ /

\subsection{Verse 15}
\subsubsection{Root Text}
\begin{quote}
	prajñājñānād uttaraṃ bodhicittā-\\
	svādas turyaṃ sekam\footnoteB{
		sekam] \EDD ; seṣam \MS
	} āhāvaraṃ tat |\\
	yasmāt\footnoteB{
		yasmāt] \EDD ; paścāt \MS
	} sarvo bhāvanāsu prayāso \\
	vyarthaḥ prāptas tatphalasya prasiddheḥ || 15 ||\\

	shes rab ye shes phyis ni byang chub sems/ /\\
	ro myang dbang bzhir 'chad pa de mchog min/ /\\
	gang phyir bsgom sogs kun stsol don med pa/ /\\
	thob 'gyur de 'bras rab tu mi rung phyir/ /
\end{quote}

\subsubsection{Commentary}
[\EDD\ p.\ 140] prajñājñānetyādi | 'prajñājñānopadeśād uttarakālaṃ yat saṃ bodhicittasyāmṛtarūpasya rasanayā grahaṇam, tat turyaṃ caturthaṃ [\MS\ fol.\ 6r] sekam āha kaścit, tac cāvaraṃ hīnaṃ vinikṛṣṭam iti yāvat | kasmād avaram, yasmāt sarvaprayāso mantramudrādevatādyākāra bhāvanāsu punaḥ punar anuṣṭhānalakṣaṇas tathāgatokto vyarthaḥ prāptaḥ | kutas tatphalasya bhāvanāsādhyasya phalasya bodhicittāsvādakāla eva prasiddhatvāt prāptatvāt | anyasya viśiṣṭasya phalasyābhāvād iti yāvat || 15 ||\\

\textbf{\TVA}\\
shes rab ye shes zhes bya ba la sogs pa la shes rab dang ye shes ni shes rab ye shes te/ dbang bskur ba'i bye brag go/ /phyis ni 'das pa'i 'og tu'o/ /gang zhe na/ byang chub kyi sems te bdud rtsi'i ngo bo lces bkug nas blangs pa gang yin pa de ni bzhi pa ste/ dbang bskur bzhi pa yin par smra ba de ni mchog ma yin pa dang/ dman pa dang/ shin tu smad pa zhes bya ba'i bar du'o/ /ci'i phyir mchog min zhe na/ gang gi phyir sngags dang phyag rgya dang/ lha nying la sogs pa'i rnam pa bsgom pa la yang dang yang du 'bad pa dang/ gzhan yang de bzhin gshegs pas gsungs pa'i sgrub pa'i mtshan nyid don med pa thob par 'gyur ro/ /gang las she na/ de 'bras te bsgoms pas bsgrub par bya ba'i 'bras bu byang chub kyi sems kyi ro myang ba'i dus nyid na rab tu grub pa nyid dang/ 'bras bu khyad par can gzhan yang med pa'i phyir zhes bya ba'i bar du'o/ /\\

\textbf{\TVB}\\
shes rab ye shes zhes bya ba la sogs pa la/ shes rab dang ye shes te/ dbang bskur ba'i bye brag go/ /phyis te rdzogs pa'i dus kyi byang chub gang zhe na/ sems te bdud rtsi'i ngo bo lces kun nas blang ba gang yin pa de ni bzhi pa ste/ dbang bskur ba bzhi par 'ga' zhig smra ba de ni mchog ma yin pa dang/ dman pa dang shin tu smad pa zhes bya ba'i bar du'o/ /ci'i phyir mchog min zhe na/ gang phyir sngags dang phyag rgya dang lha nyid la sogs pa'i rnam pa bsgom pa la 'bad pa dang/ gzhan yand de bzhin gshegs pas gsungs pa'i bsgrub pa'i mtshan nyid don med pa thob par 'gyur ro/ /gang las zhe na/ de 'bras bu ste/ bsgom pas bsgrub par bya ba'i 'bras bu <thob pa yin na> byang chub kyi sems kyi ro myang ba'i dus nyid na rab tu grub pa nyid dang/ thob pa nyid dang/ 'bras bu khyad par can med pa'i phyir zhes bya ba'i bar du'o/ /

\subsection{Verse 16}
\subsubsection{Root Text}
\begin{quote}
	prajñājñānād uttaraṃ prāptarāmā-\\
	svādas turyaṃ sekam āhādhamaṃ tat |\\
	yasmāt sarvo bhāvanādau prayatno \\
	buddhoddiṣṭo niṣphalaḥ saṃprasaktaḥ || 16 ||

	shes rab ye shes phyir ni dga' ba'i ro/ /\\
	myong rnyed dbang bskur bzhir smra de dman te/ /\\
	gang phyir [\TM\ fol.\ 204r]/ /bsgom sogs kun la rab tu stsol/ /\\
	sangs rgyas kyis gsungs 'bras med rab thal 'gyur/ /
\end{quote}

\subsubsection{Commentary}
prajñetyādi.
prajñājñānād uttarakālaṃ yāḥ prāptā yathāmilitā rāmāḥ striyas tāsāṃ samāpattidvāreṇa ya āsvādaḥ, tat turyaṃ sekam.
tad apy adhamam.
śeṣaṃ gatārtham.\\

\textbf{\TVA}\\
shes rab ces bya ba la sogs pa la/ shes rab ye shes la dus phyis ji lta ba bzhin du rnyed pa'i dga' ba ste/ bud med [\TVA\ fol.\ 209r]/ /gang yin pa de dang/ rig pa'i sgo nas ro myang ba'i dbang bskur ba gang yin pa de ni dman pa'o/ /lhag ma'i don ni rtogs par zad do/ /\\

\textbf{\TVB}\\
shes rab ces bya ba la sogs pa la/ shes rab ye shes las dus phyis <phyis gzhan las> ji lta ba bzhin du rnyed pa'i dga' ba ste/ bud med gang yin pa de dang reg pa'i sgo nas ro myong ba'i dbang bskur ba gang yin pa de ni dman pa'o/ /lhag ma'i don ni rtogs par zad do/ /\\

atha "caturthaṃ tat punaḥ [tathā]" (gu ta 18.112) iti vyākhyāyate | caturtham iti prajñājñānaṃ tṛtīyam apekṣya caturtham ity ucyate | tad iti tacchabdena tad eva prajñājñānaṃ tadrūpaṃ parāmṛśyate | punar iti punaḥ śabdena tasmād viśeṣaḥ | viśeṣaś cātra nirāsravanirantarātyantasphītāvicchinnaprabandhapravāhitvalakṣaṇaḥ | tatheti tathāśabdena tādṛśatvam abhidhīyate | tādṛśatvaṃ ca yādṛśyā prajñādiyuktyā sāmagyrā yādṛśaṃ prajñājñānam utpannaṃ paścād api tādṛśyaiva sāmagyrā tathaiva cotpadyate, nānyatheti tathāśabdārthaḥ | atra ca lakṣyalakṣaṇabhāvenārtho boddhavyaḥ | lakṣyate 'neneti lakṣaṇam anubhūyamānaṃ prajñājñānam, apratīyamānasya lakṣaṇatvāyogāt, nāgṛhītaviśeṣaṇā [\EDD\ p.\ 141] viśeṣyabuddhir iti nyāyāt | lakṣyate jñāyate pratipādyate 'neneti lakṣyaṃ sākṣātkariṣyamāṇaṃ caturtham |\\

\textbf{\TVA}\\
de ltar de bzhin bzhi pa yang/ /zhes bya ba bshad par bya ba la bzhi pa ni gsum pa shes rab ye shes las ltos nas bzhi pa zhes bya'o/ /de ltar zhes bya ba la de zhes bya ba'i sgra shes rab ye shes kyi ngo bo de mtshon par byed pa yin no/ /yang zhes bya ba la/ yang gi sgras ni de las khyad par du gyur pa yin la/ khyad par yang zag pa med pa shin tu rgyas pa nyid dang/ bar chad med pa nyid dang/ rgyun mi 'chad par skye ba nyid kyi mtshan nyid do/ /de bzhin zhes bya ba ni de bzhin gyi sgras de dang 'dra ba nyid du bstan pa yin la/ de dang 'dra ba nyid kyang ci 'dra ba'i shes rab dang ldan pa'i tshogs pa las/ shes rab ye shes ci 'dra ba'i skye ba bzhin du phyag kyang de dang 'dra ba'i tshogs pa las kyang de dang 'dra ba bskyed kyi gzhan du ni ma yin no zhes bya ba ni/ de bzhin gyi sgra'i don yin no/ /'dir yang mtshon par bya ba'i don mtshan par byed pa'i dngos po khong du chud par bya ba yin te/ 'dir go bar byed pas na/ mtshon byed de nyams su myong bar gyur pa'i shes rab ye shes so/ /khyad par ma gzung pa'i khyad par can rtogs par mi 'gyur ba zhes bya ba'i rig pa 'dis nyams su ma myong bar mtshon byed nyid du mi 'thad do/ /mtshon par bya zhing shes par bya la go bar bya zhing bsgrub par bya bas na mtshon bya ste mngon du gyur pa'i bzhi pa'o/ /\\

\textbf{\TVB}\\
de la de ltar de bzhin bzhi pa yang zhes bya ba bshad pa la/ gsum pa shes rab ye shes las ltos [\TVB\ fol.\ 78r] nas bzhi pa zhes bya'o/ /de ltar zhes bya ba ni de'i sgras shes rab ye shes de nyid kyi ngo bo de mtshon par byed pa yin no/ /yang zhes bya ba ni yang gi sgras de la khyad par du gyur pa yin la/ khyad par yang zag pa med pa shin tu rgyas pa nyid rgyun mi chad par skye ba nyid kyi mtshan nyid do/ /de bzhin zhes bya ba ni de bzhin gyi sgras de dang 'dra ba nyid du bstan pa yin la/ de dang 'dra ba nyid kyang ci 'dra ba'i shes rab dang ldan pa'i tshogs pa las/ shes rab ye shes ci 'dra ba skye ba bzhin du/ phyis kyang de dang 'dra ba'i tshogs pa las kyang de dang 'dra ba <brtson 'grus bar ma chad rgyun ma chad> bskyed kyi gzhan du ma yin no zhes bya ba ni/ de bzhin gyi sgra'i don yin no/ /'dir mtshon par bya ba dang mtshon par byed pa'i dngos pos don khong du chud par bya ba yin te/ 'dis mtshon par byed pas na mtshon byed ste/ nyams su myong bar gyur pa'i shes rab ye shes so/ /rtogs par ma gyur pa ni mtshon byed nyid du mi 'thad pa'i phyir dang/ khyad par <dpe> ma bzung bar khyad par <don> gyi gzhi rtogs par mi 'gyur ba'o/ /zhes bya ba'i rigs pas 'dis mtshon par byed shes par byed/ khong du chud par byed bsgrub par byed pas na <mtshon par> bya ba bsgom pas mngon sum du gyur pa ni bzhi pa'o/ /\\

atra caturthaṃ nāstīty eke |\\

\textbf{\TVA}\\
'dir 'ga' zhig /dbang ni rnam pa gsum dag tu/ rgyud 'di las ni rab tu grags/ /zhes gsungs pas na/ bzhi pa ni yang dag pa ma yin no zhe na/\\

\textbf{\TVB}\\
'dir 'ga' zhig bzhi pa ni <gzhan dag na re> yang dag par ma yin no zhe na/\\

nanu caturtham ity etad asti tat padaṃ tat kathaṃ nāstīty ucyate ? satyam, upadeśasaṃrakṣārthaṃ sattvavyāmohanāya ca tṛtīyam eva caturthaśabde[\MS\ fol.\ 6v]noktaṃ bhagavatā | anyathā tatpunar iti noktaṃ syāt | tad atyantāsaṃgatam, caturthasya pramāṇasiddhasya pratipāditatvāt, pratipādayiṣyamāṇatvāc ceti | atra lakṣaṇaṃ prajñājñānaṃ pratītam eva sarvaiḥ, lakṣyā (kṣye) paraṃ vyāmohaḥ | tad vicāryate | lakṣyaṃ hi bhagavadartharūpaṃ vā syāt, jñānarūpaṃ vā? na tāvad artharūpam, arthasyaikasyābhāvāt, ekānekaviyogitvena pramāṇena tasya nirākṛtatvāt | mantranaye ca vijñānavādamadhyamakamatayor eva pradhānatvād jñānarūpaṃ vā syāt | jñānaṃ ca sākāraṃ vā nirākāraṃ vā, sākāram api citrādvaitarūpaṃ vā syād anekarūpaṃ vā syād iti vikalpāḥ | tatra sākāravijñānaṃ sarvathaiva gagaṇakamalavan nāstīti nirākāravādino bruvate |\\

\textbf{\TVA}\\
de ltar de bzhin bzhi pa yang/ /zhes bya ba'i tshig bcom ldan 'das kyis gsungs pa yod pa ma yin nam/ ji ltar med ce na/ bden te/ bcom ldan 'das kyis man ngag bsrung bar bya ba'i phyir dang/ bstan par bya ba nyid yin pa'i phyir ro/ /gzhan dag sgrub pa chen po bzhi pa ste/ /zhes bya ba la sogs pa la bzhi pa zhes bya ba'i tshig yod pa yin pa de nyid 'dir rnam par 'brel pa'i phyir bzhi pa'i sgra gsungs pa yin gyi/ bzhi pa zhes bya ba'i dbang [\TVA\ fol.\ 209v] bskur ba'i khyad par yang yod pa ma yin no zhe na/ de'i 'dod pa nyid ltar na de la mngon du bya ba 'bad pa'i phyir de bsgom pa don med pa'i nyes par 'gyur ro/ /'bras bu dang bcas par khas len pa yang bzhi pa grub pa yin no/ /de la mtshon par byed pa shes rab ye shes nyid ni khong du chud la mtshon par bya ba la thams cad rmongs par gyur pas de rnam par dpyad par bya ste mtshon par bya ba yang srid na don gyi ngo bo zhig gam/ shes pa'i ngo bo zhig tu 'gyur grang na/ de la don gyi ngo bo nyid ni ma yin te/ don nyid med pa'i phyir dang/ gcig dang du ma'i rang bzhin dang bral ba'i tshad mas de bsal ba nyid kyi phyir ro/ /sngags kyi tshul la yang rnam par shes par smra ba dang/ dbu ma pa'i 'od gzhung gtso bo nyid yin pa'i phyir ro/ /shes pa'i ngo bo zhig tu 'gyur grang na/ shes pa yang rnam pa dang bcas pa zhig gam/ rnam pa med pa zhig tu 'gyur grang/ rnam pa dang bcas pa la yang shes pa gnyis med pa'i ngo bo'am/ du ma'i ngo bo zhig tu 'gyur zhes rnam par brtag par bya ba yin no/ /de la rnam pa med par smra ba dag ni rnam par shes pa rnam pa dang bcas pa ni rnam pa thams cad du yod pa ma yin te/ nam mkha'i me tog dang 'dra'o/ / zhes smra'o/ /\\

\textbf{\TVB}\\
<lan> de lta na de ma yin pa gzhan de ltar de bzhin [\TVB\ 77v] bzhi pa yang zhes bya ba der bzhi pa zhes bya ba'i tshig <rgyud phyi ma> bcom ldan 'das kyis gsungs pa <lung gis bsal ba> yod pa ma yin nam/ de ci ltar med ces brjod/ bden te bcom ldan 'das kyis man ngag bsrung ba'i phyir dang/ sems can mgo rmongs par bya ba'i phyir gsum pa kho na la bzhi pa'i sgras gsungs pa yin gyi/ gzhan du na de ltar de bzhin bzhi pa yang zhes gsungs par mi <rgol> 'gyur ro/ /de ni shin tu <lan> 'brel pa med pa yin te/ <'dun dang de ltar gsum> tshad mas grub pa bzhi pa gong du bstan pa'i phyir dang/ bstan par bya ba bzhi pa nyid yin pa'i phyir ro/ /gzhan dag bsgrub pa chen po <bsnyen bsgrub yan lag bzhi las> bzhi pa yin zhes bya ba la sogs pa'i bzhi zhes pa'i tshig yod pa gang yin pa de nyid 'dir rnam par bkrol ba'i phyir gsungs pa yin gyi bzhi pa zhes bya ba'i dbang bskur gyi khyad par yang yod pa ni ma yin no zhe'o/ /de'i 'dod pa <ma hA mu tra> de nyid ltar na 'di la 'bad pa'i phyir bsgom pa don med pa'i nyes par 'gyur ro/ /'bras bu dang <lha'i sku dang bde ba dang> bcas pa khas len na yang bzhi pa <tshig la mi mthun don la mthun> grub pa yin no/ /de la mtshon par byed pa shes rab ye shes nyid ni khong du chud la <dpe> mtshon par bya ba <don> la thams cad rmongs par 'gyur bas de rnam par dpyad [\TVB\ fol.\ 78r] par bya ste/ mtshon par bya ba <don> yang srid na/ don gyi ngo bo zhig gam <yul la> shes pa'i ngo bo zhig tu 'gyur grang na/ de la don gyi ngo bo nyid ni ma yin te/ <gcig dang du ma dang bral bas> don nyid med pa'i phyir dang/ gcig dang du ma'i rang bzhin dang bral ba'i tshad mas de bsal ba nyid kyi phyir ro/ sngags kyi tshul la yang rnam par shes par <sems tsam du> smra ba dang/ dbu ma pa'i 'dod gzhung gtso bo nyid yin pa'i phyir/ shes pa'i ngo bo nyid du 'gyur grang na/ shes pa yang rnam pa dang bcas pa zhig gam/ rnam pa med pa zhig tu 'gyur/ rnam pa dang bcas pa la yang shes pa gnyis med pa'i ngo bo'am/ du ma'i ngo bo zhig tu 'gyur zhes rnam par brtag par bya ba yin no/ /de la rnam med par smra ba dag ni rnam par shes pa rnam pa dang bcas pa ni thams cad du yod pa ma yin te/ nam mkha'i me tog dang 'dra'o zhes smra'o/ /\\

nanu nīlapītaśuklādighaṭapaṭapra(śa)kaṭādirūpeṇa vā''kārāḥ pratibhāsante pratyakṣataḥ | te cārthasyābhāvād jñānarūpā eva | tat kathaṃ sākāraṃ nāstīti ? satyam, pratibhāsanta evākārāḥ, param alīkarūpeṇa | alīkarūpatā ca ekānekaviyogitvapramāṇalakṣaṇena prasiddhā | tasya ca pramāṇasvarūpasyānyatra kathitatvān neha pratanyate | alīkatvaprasiddhā ca māyāmayā ivākārā bhrāntirūpāḥ prakāśyante | bhrāntinivṛttau ca nirākāram eva śuddhasphaṭikasaṃkāśaṃ pāra[mā]rthikaṃ siddhaṃ bhavatīti | ataś citrādvaitarūpam anekarūpaṃ ca sākāraṃ vijñānam astīti vikalpadvayaṃ nirastaṃ bhavatīti |\\

\textbf{\TVA}\\
mngon sum la sngon po dang/ ser po dang/ dkar po la sogs pa dang/ bum pa dang/ snam bu dang/ shing rta la sogs pa'i ngo bor rnam pa rnams snang ba ma yin nam/ don de dag la med pa'i phyir shes pa'i ngo bo nyid yin na/ de ji ltar rnam pa dang bcas pa ma yin zhe na/ bden te/ rnam pa rnams snang ba nyid ni yin mod kyi 'on kyang brdzun pa yin no/ /brdzun pa nyid kyang gcig dan du ma dang bral ba'i mtshan nyid kyis rab tu grub pa'o/ /tshad ma'i ngo bo de yang gzhan dag tu bshad pa nyid kyis 'dir ma spros so/ /brdzu na pa nyid du grub pas kyang sgyu ma'i rang bzhin dang 'dra bar rnam par 'khrul pa'i rang bzhin rab tu ston la/ 'khrul pa ldog pas kyang shel sgong dag pa lta bu rnam pa med pa de kho na don dam par grub par 'gyur ro/ /de bas na rnam pa dang bcas pa'i rnam par shes pa ni/ shes pa gnyis med pa'i ngo bo'am/ du ma'i ngo bor yod do zhes pa'i rtog pa [\TVA\ fol.\ 210r]/ /gnyis ka bsal ba yin no/ /\\

\textbf{\TVB}\\
mngon sum la sngon po dang/ ser po dang/ dkar po la sogs pa dang/ bum pa dang/ snam bu dang/ shing rta la sogs pa'i ngo bor rnam pa rnams snang ba ma yin nam/ /don <rnam pa> de dag kyang med pa'i phyir shes pa'i ngo bo nyid kyang med yin na/ de ji ltar rnam pa dang bcas pa ma yin zhe na/ bden te rnam pa rnams snang ba nyid ni yin mod kyi 'on kyang brdzun pa'i ngo bor yang brdzun pa'i ngo bo nyid du gcig dang du ma dang bral ba'i tshad ma'i mtshan nyid kyis rab tu 'grub po/ /tshad ma'i ngo bo de yang <dbu ma de kho na nyid la 'jug pa> [\TVB\ fol.\ 78v] gzhan dag tu bshad pa nyid kyis 'dir ma spros so/ brdzun pa nyid du grub pas kyang/ sgyu ma'i rang bzhin dang 'dra bar rnam par 'khrul pa'i rang bzhin du rab tu ston la/ 'khrul pa ldog pas kyang shel sgong dag pa lta bu rnam pa med pa kho na don dam par grub par 'gyur ro/ /de bas na rnam pa dang bcas pa'i rnam par shes pa ni shes pa gnyis med pa'i ngo bo'am/ du ma'i ngo bor yod do zhes pa'i rtog pa gnyis bsal ba yin no/ /\\

nanu nirākāram api vijñānam upalabdhilakṣaṇaprāptaṃ svapne 'pi nopalabhyate, tat kathaṃ tad asti paramārtham i[\MS\ fol.\ 7r]ty ucyate? ucyate, sukhākāraṃ vijñānam antaḥparisphuradrūpaṃ nirākāraṃ saṃvedyata eva, nīlādyākārāḥ punar alīkāḥ pratibhāsante | anyathā teṣāṃ satyatve sarva evākārāḥ satyāḥ syuḥ | tathā hi grāhyagrāhakabhāvādikam api satyaṃ [\EDD\ p.\ 142] syāt | tataś ca sarveṣām eva satyapratibhāsatvena yu[mu]ktiprasaṅgāt, keṣāñcid api mithyāpratibhāsasya bhrāntirūpasyāpratibhāsanāt | tathā coktam "draṣṭavyaṃ bhūtato bhūtaṃ bhūtadarśī vimucyate" (abhi a 5.21) | tasmād akāmakenāpi nīlādyākārāṇām alīkatvam evaiṣṭavyam | sukhādikaṃ nirākāraṃ satyam upalabhyate | tat kathaṃ nopalabhyata iti |\\

\textbf{\TVA}\\
rnam pa med pa'i shes pa yang dmigs pa'i rig byar gyur pa rmi lam na yang ma dmigs pa ma yin nam/ de ji ltar na don dam par grub par yod pa zhes bya zhe na/ shes pa rnams ni bde ba la sogs pa'i rnam pa yongs su gsal ba'i ngo bo rnam pa med par rig pa kho na yin la/ sngon po la sogs pa'i rnam pa yang brdzun par snang ba yin no/ /gzhan du na de dag bden pa nyid yin na ni rnam pa thams cad bden par 'gyur la/ de lta na yang gzung ba dang 'dzin pa'i dngos po bden par 'gyur ro/ /snang ba bden pa nyid du gyur pa de bas na/ de dag thams cad kyang grol ba nyid du thal bar 'gyur te/ cung zhig kyang log pa'i rnam par ngo bo ni snang ba'i phyir ro/ /de skad du yang/ yang dag nyid la yang dag lta/ /yang dag mthong na rnam par grol/ /zhes gsungs pa'i phyir/ de'i phyir mi 'dod du zin kyang sngon po la sogs pa'i rnam pa brdzun pa nyid du blta bar ba'o/ /\\

\textbf{\TVB}\\
rnam pa med pa'i shes pa yang dmigs pa'i rig byar gyur pa'i rmi lam na yang mi dmigs pa ma yin nam/ de ji ltar na don dam par yod par grub pa zhes bya/ 'on te shes pa'i nang na bde ba la sogs pa'i rnam pa yongs su bsal ba'i ngo bo rnam pa med par rig pa kho na yin la/ sngon po la sogs pa'i rnam pa yang brdzun par snang ba yin no/ /gzhan du de dag <sems> bden pa nyid yin na ni/ rnam pa thams cad <rnam pa yang> bden par gyur la/ de ltar yang bzung ba dang 'dzin pa'i dngos po la sogs pa'i yang bden par 'gyur ro/ /snang ba bden pa nyid du 'gyur ba de bas na de dag thams cad kyang grol ba nyid du thal bar 'gyur te/ cung zhig dang log pa'i rnam par 'khrul pa'i ngo bo mi snang ba'i phyir ro/ /de skad du yang dang yang dag nyid la yang dag lta/ yang dag mthong na rnam par grol zhes gsungs so/ /de'i phyir <sems can> mi 'dod du zin kyang sngon po la sogs pa'i rnam la brdzun [\TVB\ fol.\ 79r] pa nyid du blta bar bya'o/ /\\

nanu sukhādyākāram eva vijñānam upalabhyate, sukhāder ākārasvabhāvatvāt | na ca sukhādyākāraśūnyaṃ jñānaṃ svapne 'pi saṃvedyate | sakalabhrāntivigamādaṣṭamyāṃ bhūmāvupalabdhilakṣaṇaprāptir bhavatīti | atrāpi śapathollaṅghanaṃ vinā anyan na pramāṇam asti prasādhakam iti, tadabhiprāyāparijñānāt, sukhādyākārasyaiva nīlādyākārarahitasya vijñānasya nirākāratveneṣṭatvāt | tac cedānīm eva svasaṃvedanapramāṇasiddhaṃ sakalaprāṇabhṛtāmastīti kathaṃ nopalabdhiḥ |\\

\textbf{\TVA}\\
bde ba la sogs pa rnam pa bden par dmigs pa yin na de ji ltar na dmigs pa ma yin zhe na/ bde ba la sogs pa'i rnam pa'i shes pa yang rnam pa dang bcas pa kho na la dmigs pa kho na yin te/ bde ba la sogs pa ni rnam pa'i rang bzhin nyid yin pa'i phyir ro/ /bde ba la sogs pa'i rnam pas stong pa'i rnam par shes pa ni/ rmi lam na yang rig pa med pa ma yin nam/ 'khrul pa ma lus pa dang bral ba'i sa brgyad pa la dmigs pa'i rig byar 'gyur ro/ /'di la yang mna' dor bam yin pa sgrub par byed pa'i chad ma yod pa ma yin no zhe na/ de ni bden pa ma yin te/ bsam pa yongs su ma shes pa'i phyir dang/ bde ba la sogs pa nyid sngon po la sogs pa'i rnam pa dang bral ba'i rnam par shes pa med pa nyid du 'dod pa'i phyir te/ de yang da lta nyid na/ rang rig pa'i tshad mas grub pa srog chags thams cad la yod pa yin na/yin no/ /ji ltar na dmigs par ma gyur/ \\

\textbf{\TVB}\\
bde ba la sogs pa'i rnam pa brdzun pa bden par dmigs pa yin na/ de ci ltar na dmigs pa ma yin zhe na/ bde ba la sogs pa'i rnam pa yang rnam pa dang bcas pa'i shes pa kho na la dmigs pa yin te/ bde ba la sogs pa'i rnam pa'i rang bzhin nyid yin pa'i phyir ro/ /bde ba la sogs pa'i rnam pas stong pa'i rnam par shes pa ni rmi lam na yang rig pa med pa [N-T fol.\ 79v] ma yin nam/ 'on te 'khrul pa ma lus pa dang bral ba'i sa brgyad pa la dmigs pa'i rig byar 'gyur ro/ /zhe na/ 'di la yang mna' dor ba ma yin pa'i bsgrub par byed pa'i tshad ma ni yod pa ma yin pas/ de ni bden pa ma yin te/ bsam pa yongs su mi shes pa'i phyir dang/ bde la sogs pa nyid sngon po la sogs pa'i rnam pa dang bral ba'i rnam par shes pa la rnam pa med pa nyid du 'dod pa nyid kyi phyir/ 'on te de yang da ltar nyid na rang rig pa'i <gcig du ma'i> tshad mas grub pa srog chags thams cad la yod pa yin na/ ji ltar dmigs par ma gyur zhe na/\\ 

nanu tad apy ekānekasvabhāvaviyogād alīkam eva bhrāntimātram, ekānekasvabhāvarahitasya sākāranirākāravijñānavyāpitvāt |\\

\textbf{\TVA}\\
de yang gcig dang du ma dang bral ba'i phyir brdzun pa nyid dang 'khrul pa tsam yin te/ gcig dang du ma'i rang bzhin dang bral ba nyid kyis/ rnam pa dang bcas pa'i shes pa dang/ rnam pa med pa'i shes pa la khyab pa nyid kyi [\TVA\ fol.\ 210v] phyir ro/ /\\

\textbf{\TVB}\\
bde ba la sogs pa'i rnam pa brdzun pa bden par dmigs pa yin na/ de ci ltar na dmigs pa ma yin zhe na/ bde ba la sogs pa'i rnam pa yang rnam pa dang bcas pa'i shes pa kho na la dmigs pa yin te/ bde ba la sogs pa'i rnam pa'i rang bzhin nyid yin pa'i phyir ro/ /\\

nanv anena nyāyena sakalasākāranirākāravijñānasyālīkatvaprasādhanān na kiñcid api pāramārthikaṃ vastutattvam astīti? tat kathaṃ lakṣyasya svarūpaṃ pramāṇata upalakṣayitavyam | naiṣa doṣaḥ, madhyamakamate pramāṇato 'līkatāsiddhāv api māyopamapratibhāsamātrasyaikānekasvabhāvarahitasya dharmirūpasyāpratiṣedhāt | tatraiva cālīke pratibhāsamātre lakṣyalakṣaṇasaṃsāranirvāṇa[\MS\ fol.\ 7v]maṇḍalacakrādibhāvanā sakalajagadarthakriyādīnām avyāhatā vyavasthā [ca] sidhyatīti | tathā coktam\emdash \\

\textbf{\TVA}\\
rig pa 'dis rnam pa dang bcas pa dang/ rnam pa med pa'i shes pa ma lus par brdzun par rab tu bsgrubs pas cung zhig kyang don dam pa'i dngos po de kho na nyid du med pa ma yin nam/ ji ltar na mtshon par bya ba'i rang bzhin tshad mas nye bar mtshon par bya zhe na/ nyes pa de ni med de/ dbu ma pa 'dod pa'i tshad mas brdzun pa nyid du grub pa yang/ gcig dang du ma'i rang bzhin dang bral ba/ chos can gyi ngo bo snang ba tsam dang sgyu ma lta bu ma bkag pa'i phyir ro/ /brdzun du snang bde nyid la yang mtshon par bya ba dang/ mtshon par byed pa dang/ 'khor ba dang/ mya ngan las 'das pa dang/ dkyil 'khor gyi 'khor lo la sogs pa bsgoms pas 'gro ba ma lus pa'i don mdzad pa la sogs pa gnod pa med par rnam bar gzhag pa grub pa yin no/ /de skad du yang/ \\

\textbf{\TVB}\\
gal te rigs pa 'dis rnam pa dang bcas pa dang rnam pa med pa'i shes pa ma lus pa brdzun par rab tu bsgrub pas cung zhig kyang don dam pa'i dngos po de kho na nyid du yang med pa ma yin nam <yin>/ de ji ltar na mtshon par bya ba'i rang bzhin du tshad mas nye bar mtshon par bya zhe na/ nyes pa de ni med dbu ma pa 'dod pa'i tshad mas brdzun pa nyid du grub kyang/ gcig dang du ma'i rang bzhin dang bral ba chos can gyi ngo bo snang ba sgyu ma lta bu ma bkag pa'i phyir ro/ /brdzun du snang ba de nyid la mtshon par bya ba dang/ mtshon par byed pa dang/ 'khor ba dang mya ngan las 'das pa dang/ dkyil 'khor gyi 'khor lo la sogs pa bsgoms pas 'gro ba ma lus pa'i don byed pa la sogs pa/ gnod pa med pa rnam par gzhag pa grub pa yin no/ /de skad du yang 

\begin{quote}
	buddhatvaṃ vajrasattvatvaṃ saṃvṛtyaiva prasādhayet |

	\textbf{\TVA}\\
	sangs rgyas rdo rje sems dpa' nyid/ /kun rdzob nyid du rab tu grub/ /

	\textbf{\TVB}\\
	sangs rgyas dang rdo rje sems dpa' nyid kyang/ kun rdzob nyid du rab tu bsgrub
\end{quote}

iti | nanu sarvam eva vastujātamalīkarūpatayā niḥsāram, tadā kimarthaṃ maṇḍalacakrādibhāvanāprayāsaḥ kriyate? asad etat | mithyādhyāropaṇārthaṃ yatno 'satyopi [\EDD\ p.\ 143] muktaye iti vacanāt | yady api vicāryamāṇaṃ pāramārthikaṃ vasturūpaṃ nāsti, tathāpy ahaṃ sukhī bhaveyaṃ [mā] duḥkhyabhūvam iti tṛṣṇā sakalaprāṇabhṛtām asti | yathā tulye 'pi mithyātve śubhāśubhasvapnayoḥ śubhasvapnadarśanāt saumanasyam, aśubhasvapnadarśanāc ca daurmanasyam, tadapanayanāya ca saddharmapāṭhamantrajāpādau pravṛttir bhavati, tathā mithyātvāviśeṣe 'pi duḥkhādiprākṛtavikalpahānāya samyaksaṃbodhilakṣaṇaprāptaye ca prekṣāvatām arthināṃ pravṛttir bhaviṣyatīti |\\

\textbf{\TVA}\\
ces gsungs so/ /dngos po skyes pa thams cad brdzun pa'i ngo bo nyid kyis snying po med ba ma yin nam/ de'i tshe ci'i phyir bri ba'i dkyil 'khor la sogs pa bsgom pa'i 'bad pa byed ce na/ de ni bden pa ma yin te/ log par sgro 'dogs gcod pa'i phyir ro/ / grol byed brdzun yang 'bad par bya/ /zhes bya ba'i tshig gis ji ltar yang rnam par dpyad na/ yang dag pa'i dngos po'i ngo bo nyid med pa de ltar na yang bdag bde bar gyur cig /bdag sdug bsngal bar ma gyur cig ces bya ba ni/ sred pa srog chags ma lus pa la yod do/ /'di ltar dge ba dang mi dge ba'i rmi lam log pa nyid du mtshungs kyang dge ba'i rmi lam mthong pas ni/ yid bde bar gyur la/ mi dge ba'i rmi lam gyis ni yid mi bde bar 'gyur te/ de bsal ba'i phyir dam pa'i chos klog pa dang/ sngags zlos pa la sogs pa la rab tu 'jug par 'gyur ro/ /de bzhin du brdzun pa bzhin du khyad med kyang sdug bsngal la sogs pa rang bzhin gyi rnam par rtog pa spangs pa'i phyir dang/ yang dag par rdzogs pa'i byang chub kyi mtshan nyid kyi 'bras bu thob par bya ba'i phyir rtog pa dang ldan pa don du gnyer ba rnams rab tu 'jug par 'gyur ro/ /\\

\textbf{\TVB}\\
ces gsungs pas so/ /'on te dngos po skyes pa thams cad brdzun pa'i ngo bo nyid kyis snying po med pa ma yin nam/ de'i tshe ci'i phyir dkyil 'khor gyi 'khor lo la sogs pa bsgom pa'i 'bad pa byed ce na/ de ni bden pa ma yin te/ /log par sgro 'dogs spangs pa'i phyir/ /grol ba po ni med dang brtson/ /zhes bya ba'i tshig gis/ ji ltar yang rnam par dpyad na yang dag pa'i dngos po'i ngo bo nyid med pa/ de lta na yang bdag bde bar gyur cig sdug bsngal bar ma gyur cig ces pa'i sred pa srog chags ma lus pa la yod do/ /'di ltar dge ba dang/ mi dge pa'i rmi lam log pa nyid du mtshungs kyang/ dge ba'i rmi lam mthong bas yid bde bar 'gyur la/ mi dge ba'i rmi lam mthong ba las [\TVB\ fol.\ 80r] ni yid mi bde bar 'gyur te/ de bsal ba'i phyir dam pa'i chos klog pa dang/ sngags zlos pa la sogs pa la rab tu 'jug par 'gyur ro/ /\\

nanu yadarthatvād ayam ārambhaḥ so 'rthaḥ pralayaṃ gataḥ | tathā hi lakṣyalakṣaṇacintātra prastutā sā ca vismṛtā, kva gateti na jñāyate |\\

\textbf{\TVA}\\
rtsom pa 'di'i [\TVA\ fol.\ 211r]/ /don gang yin pa de'i don nyams par 'gyur ba ma yin nam/ 'di ltar mtshan gzhi dang/ mtshan nyid dpyad pa na skabs la bab pa de yang gtam gzhan du thal bas brjod pa'i phyir/ gang du song ba mi shes so zhe na/ \\

\textbf{TVB}
gal te gang gi don du <bzhi pa bshad pa'i bshad pa'i dus> 'di brtsams pa'i/ /de'i don <bzhi pa> ni nyams par 'gyur/ /'di ltar mtshan gzhi dang mtshan nyid <bzhi pa> dpyad pa'i skabs la bab na/ de yang gtam gzhan du thal bas brjed pa'i phyir/ gang du song ba mi shes so zhe na/\\

nanu kṛtaiva sā saptabhir bhedaiḥ | satyam, kintu guḍagorasanyāyena | tathā hi na jñāyate, kiṃ tatsāram asāraṃ veti ? ucyate, mantragnayavihitakramābhāvāt, samāpattibhāvanāvaiyarthyād yuktā (ktya) bhāvāc ca prathamasya niḥsāratā | tathā hi\emdash samagrasāmagrīkaṃ yat tad avaśyam eva bhavati | anyathā samagrasāmagrīkam eva tan na bhavet | sākṣātkaraṇāvasthāyāṃ samagrasāmagrīkaṃ tad vartate, tad avaśyaṃ tena bhavitavyam | sati ca bhavane na pratham asya hānir iti | śarīrādyākāraśūnyasya kevalasātarūpasyānupa[la]bdher na dvitīyasya sāratā | tathā hi\emdash pramāṇaniścitaṃ prekṣāvatā bhāvanīyaṃ na yathākathañcit | pramā[\MS\ fol.\ 8r]ṇena saṃvalitarūpam eva sarvadopalabhyate | tad eva sarvajanānāṃ kamanīyatayā pratibhāsate | tasmāt kevalasya rucyabhāvāc cakrākārasaṃvalitasyopalabdheḥ sākṣāt kartum aśakyatā(tvā)c ca dvitīyasya kalpanā[mā]trateti, nirupadravabhūtārthasvabhāvatvena sātmībhūtasya tyaktum aśakyatvāt | saṃvalitarūpasya [\EDD\ p.\ 144] bhedābhāvāt prayojanābhāvāc ca na tṛtīyaḥ kalpanābhāvaḥ | tathā hi\emdash sahopalambhena tādātmyasiddhāv ekasya parityāge 'parasyāvaśyaṃ parityāgo na vā kasyacid iti | prapañcatvena bahuprayāsatvād vicārāsahatvena bhrāntirūpatayāparamārtharūpatayā ca na tṛtīyapakṣasya kalyāna(ṇa)teti |  atra kecid yuktiṃ varṇayanti\emdash prapañcarūpatvābhāve 'pi sūkṣmasya bindvādeḥ punaḥ punar bhāvanayā sākṣātkaraṇaṃ yāvat prayāsas tāvat sarvatraiva bhāvyavastuni saṃbhavati | tad atra yadi prayāsabhayam, na kiñcid api bhāvanīyaṃ prapañcarūpatvād iti cet, prapañcāprapañcayor bhāvanāvasthāyāṃ ko viśeṣa iti cet, aprapañcaṃ śīghram eva sthirībhavatīty ayaṃ viśeṣaḥ |\\

\textbf{\TVA}\\
de rnam pa bdun du dbye ba byas pa nyid ma yin nam zhe na/ bden te bu ram dang mngar ba nyid kyi tshul gyis de ltar ni mi shes te/ de las snying po dang snying po ma yin pa gang zhe na/ smras pa/ sngags kyi tshul la brjod pa'i rim pa med pa'i phyir dang/ lha'i rnal 'byor gyi snyoms par 'jug pa'i sgom pa don med pa'i phyir dang/ rigs pa med pa'i phyir na/ dang po snying po med pa nyid de/ 'di ltar tshogs pa can rnams tshogs par gyur na/ 'bras bu gang yin pa de gdon mi za ba nyid yin no/ /gzhan du na de'i tshogs pa ma yin par 'gyur te/ de bas mngon du byas pa'i gnas skabs na tshogs pa can rnams tshogs pas na/ de'i 'bras bu gdon mi za bar yod par 'gyur ro/ /de ltar gyur pas dang po nyams pa yin no/ /sku la sogs pa'i rnam pas stong pa'i bde ba'i ngo bo 'ba' zhig mi dmigs pa'i gnyis pa bden pa nyid ma yin te/ 'di ltar mtshan mas rnam par nges pa nyid rtog pa dang ldan pas bsgom par bya ba nyid yin gyi/ /gang yang rung ba ni ma yin no/ /tshad mas grub pa kho na dmigs pa nyid yin la/ de nyid skye bo thams cad kyis 'dod par bya ba nyid du snang ngo/ /de bas na gcig pu mi dmigs pa'i phyir dang/ 'dod par bya ba ma yin pa'i phyir dang/ 'khor lo'i rang bzhin mi dmigs pa'i phyir dang/ mngon sum du byed par mi nus ba'i phyir dang/ 'bad pa nyid mtshungs pa'i phyir gnyis pa rnam par rtog pa tsam yin no/ /gnod par byed pa med pa'i yang dag ba'i don gyi rang bzhin nyid kyis rang dbang du gyur pa dor bar mi nus pa nyid kyi phyir dang ldan pa'i ngo bo la bye brag med pa'i phyir dang/ dgos pa med pa'i phyir gsum pa dge ba ma yin te/ 'di ltar lhan cig dmigs nas/ de'i bdag nyid grub pa las gcig yongs su dor na gzhan yang gdon mi za bar yongs su spangs pa'am/ yang na 'ga' la yang ma [\TVA\ fol.\ 211v] yin no/ /spros pa nyid kyis 'bad pa nyid mngas pa'i phyir dang/ rnam par dpyad mi bzod pa nyid kyis 'khrul pa'i ngo bo yin pas don dam pa'i ngo bo nyid ma yin pa'i phyir/ gsum pa'i tha ma'i phyogs kyang legs pa ma yin no zhe na/ de la 'ga' zhig las rigs pa cung zhig cig brjod par mi bya ste/ spros pa'i ngo bo nyid du gyur kyang/ pra mo dang/ thig le la sogs pa yang dang yang du bsgoms pa ni ji srid du mngon sum du gyur pa de srid du 'bad pas yang dang yang du bsgoms pa'i phyir dang/ thams cad du bsgom par bya ba dngos po nyid du yod la/ de la 'dir gal te 'bad pas 'jig na ni/ cung zhig kyang bsgom par mi bya bar 'gyur ro/ /spros pa rnams kyi ngo bo nyid ni de lta yin no zhe na/ spros pa dang spros pa med pa dag bsgom pa'i gnas skabs na khyad par ci zhig yod/\\

\textbf{\TVB}\\
de bdun du dbye bas byas pa nyid ma yin nam/ bden te bu ram dang dar ba'i tshul gyis de ltar ni mi shes so/ /de las snying po dang snying po ma yin pa gang yin zhe na/ /smras pa/ sngags kyi tshul la brjod pa'i <longs spyod rdzogs pa'i> rim pa med pa'i phyir <bdun mi tshad> dang/ snyoms par 'jug pa sgom pa don med pa'i phyir <bde ba med pas gsum dang bdun> rig pa med pa'i phyir dang snying po med pa'i <bde ba> nyid de/ 'di ltar tshogs [\TVB\ fol.\ 80v] pa dang tshogs pa can la 'bras bu gang yin pa de gdon mi za ba nyid yin no/ /gzhan du na <rdo rje dang pad+ma dang lha'i nga rgyal tshogs pa las> tshogs pa dang tshogs can nyid du de <lus bde ba> mi 'gyur ro/ /<bzhi pa> mngon sum du byas pa'i gnas skabs na tshogs pa dang tshogs pa can de yod pa/ des na de gdon mi za bar 'gyur ro/ /de ltar gyur pa dang po nyams pa yin no/ /sku la sogs pa'i rnam pas stong <longs spyod rdzogs snying rje rgyun bde ba chen po zag pa med> pa'i bde ba'i ngo bo 'ba' shig mi dmigs pas/ gnyis pa bden <pa> nyid ma yin te/ 'di ltar mtshan mas <de ltar de bzhi pa yang> rnam par <bdun> nges pa nyid rtog pa dang ldan pas/ bsgom par bya ba yin gyi/ gang yang rung ba ni <drug> ma yin no/ /tshad mas grub pa'i ngo bo thams cad du dmigs la de nyid skye bo thams cad kyis 'dod par bya ba nyid du snang ngo/ /de bas na gcig pu <bzhi pa> ma dmigs pa'i phyir dang/ 'dod pa med pa'i phyir dang/ 'khor lo'i rnam pa'i rang bzhin ma dmigs pa'i phyir dang/ mngon du byed pa mi nus pa'i phyir dang/ 'bad pa <lha'i sku bde ba> nyid mtshungs pa'i phyir gnyis pa rnam par rtog pa tsam yin no/ /gnod par byed pa med pa'i <bzhi pa la> yang dag pa'i don gyi rang bzhin nyid kyis rang dbang du gyur pa dor bar mi nus pa nyid kyi phyir dang/ ldan pa'i ngo bo la <gnyis pa dang> bye brag med pa'i phyir dang/ dgos pa med pa'i phyir gsum pa dge ba ma yin te/ 'di ltar lhan <lha'i rnam pa> cig dmigs nas de'i bdag nyid grub pa las/ gcig yongs su dor na <lha'i ngo bo> gzhan yang gdon mi za bar yongs su <bde ba> spang ba'am| yang na 'ga' zhig kyang ma yin no/ /<bzhi pa> spros pa nyid kyis 'bad pa mangs pa nyid kyi phyir dang/ rnam par dpyad [\TVB\ fol.\ 81r] mi bzod pa nyid kyis 'khrul pa'i ngo bo nyid yin pas don dam pa'i ngo bo nyid ma yin pa'i phyir/ gsum pa'i mtha' ma'i phyogs kyang legs pa ma yin no zhe na/ de la 'ga' zhig rigs pa brjod par byed de/ spros pa'i <bskyed pa'i rim pa'i spro bsdu> ngo bo nyid du gyur kyang phra mo dang thig le la sogs pa yang dang yang du bsgom pa na/ yang ji srid mngon sum du <'khor lo'i rnam pa> gyur pa de srid du 'bad pas yang dang yang du bsgom pa'i phyir thams cad du bsgom par bya ba'i dngos po yod do/ /de la 'dir gal te 'bad pas 'jigs na ni/ /cung zhig dang <thig le> bsgom par mi bya'o/ /spros pa nyid kyis ngo bo nyid yin pa'i phyir ro zhe na/ spros pa dang <'khor lo'i rnam pa> spros pa med pa <thig le dang phra mo> dag thig le dang phra mo bsgom pa'i gnas skabs na khyad par ci zhig yod/\\

nanu yatraivālambane cittaṃ punaḥ punaḥ preryate, niruttaraṃ (nirantaraṃ) dīrghakālaṃ ca tatraiva sthirībhavatīty āgamaḥ, yuktiś cātrāsti | tathā coktam-\\

\textbf{\TVA}\\
'on te spros pa med par myur du brtan por gyur ba 'di khyad par yin te/ dmigs pa gang kho na la sems yun ring por rgyun mi 'chad par yang dang yang du gtad pas de nyid la brtan por 'gyur ro zhes bya ba ni rigs pa yin no/ /'di la lung yang yod de/ de skad du yang/\\

\textbf{\TVB}\\
'on te spros pa med pa myur du brtan por gyur pa <khyad par> 'di yin no/ /zhe na/ dmigs pa gang kho na la sems yun ring por rgyun mi chad par yang dang yang du gtad pas de nyid brtan por 'gyur zhes bya ba ni lung yin no/ /'di la rigs pa yang yod de/ 'di skad du 

\begin{quote}
	tasmād bhūtam abhūtaṃ vā yad yad evābhibhāvyate | \\
	bhāvanābalaniṣpattau tat sphuṭākalpadhīḥ phalam ||\footnoteA{
		Dharmakīrti, \emph{Pramāṇavārttika} 2.285
	}

	\textbf{\TVA}\\
	de'i phyir yang dag yang dag min/ /\\
	gang gang shin tu goms pa las/ /\\
	bsgom pa'i stobs ni rdzogs pa na/ /\\
	de gsal mi rtog blo 'bras can/ /

	\textbf{\TVB}\\
	de phyir de nyid yang dag min <bskyed>/ /\\
	gang gang shin tu bsgoms pa las/ \\
	bsgom pa'i stobs ni rdzogs pa na/ /\\
	de gsal mi rtog blo'i 'bras bu/ /
\end{quote}

punaś coktam—\\

\textbf{\TVA}\\
zhes gsungs pa dang/\\

\textbf{\TVB}\\
zhes gsungs so/ /\\

\begin{quote}
	aho kusīdatvam aho vimūḍhatā\\
	aho janasyāsya sadarthavakratā |\\
	svacittamātrapratibuddhabuddhatā\\
	adūravartiny api yan na sevyate || iti |

	\textbf{\TVA}\\
	'dir thag mi ring ba na/ gnas pa yin par yang gsungs te/\\
	ae ma'o rmongs nyid ae ma'o snyom las can/ /\\
	ae ma'o don dam rgyab kyis phyogs 'gro 'di/ /\\
	rang sems tsam dang 'brel pa'i sangs rgyas nyid/ /\\
	ring mi gnas kyang mi 'bad ngo mtshar che/ /

	\textbf{\TVB}\\
	'dir thag mi ring ba na <rang sems> gnas pa na yang/\\
	e ma'o rmongs nyid e ma'o snyoms las can/\\
	e ma'o mdzes pa'i don la rgyab kyis [\TVB\ fol.\ 81v] phyogs/ /\\
	rang sems tsam dang 'brel pa'i <rang sems rtogs phan> sangs rgyas nyid/ /\\
	rang <don> mi gnas kyang gang zhig <sems la> rten mi byed/ /
\end{quote}

tasmān nāyaṃ viśeṣaḥ | bhrāntirūpatvenāparamārthatvam api sarvatraiva bhāvanā\emdash viśeṣe vastuni saṃbhavatīti na kiñcidapi bhāvanīyaṃ syāt | [\MS\ fol.\ 8v] tataś ca sarvatraiva mokṣamārge bhāvanāyā vaiyarthyaṃ syāt | māyopamākārānupraveśena bhrāntirūpam apy aprapañcād [\EDD\ p.\ 145] bhāvyamānam aduṣṭaṃ bhavatīti cet, na tv ayaṃ māyākārānupraveśaḥ prapañce 'pi samāna iti | tatrāpi ko doṣasyāvakāśaḥ | tasmāt prapañcam aprapañcaṃ vā yad eva rocate pramāṇasaṃgatam itarad vā tad evālasyaṃ vihāya mahāpuruṣārthibhir bhāvayitavyam ity alam atiprasaṅgeneti | atra ca sāretaravibhāgaḥ paryupāsitagurubhir eva jñātavyaḥ |\\

\textbf{\TVA}\\
zhes pa ma yin nam/ de bas 'di la khyad par med do/ /spros pa la dmigs pa ni 'khrul pa'i ngo bo nyid kyis don dam pa ma yin pa nyid do zhe na/ thams cad du bsgom pa'i yul gyis dngos po mi 'khrul pas cung zhig kyang bsgom par bya ba med par 'gyur la/ de bas na rnam pa thams cad du thar pa'i lam bsgom pa don med par 'gyur ro/ /de ltar 'khrul yang sgyu ma lta bu'i rnam par zhugs pas spros pa med par bsgom par 'gyur bskyon yod pa ma yin no zhe na/ spras [\TVA\ fol.\ 212r]/ /pa de yang sgyu ma'i rnam par zhugs pas de la yang skyon gyi skabs ci zhig yod/ de bas na don du gnyer bas spros pa'am/ spros pa med pa yang rung/ gang kho na 'dod pa'i tshad ma dang ldan pa'am/ cig shos de nyid le lo spangs nas phyag rgya chen po don du gnyer bas bsgom par bya ste shin tu thal ches pas chog go/ /de la snying po dang cig shos kyi rnam par dbye ba yang bla ma mnyes par byed pa rnams kyis shes par byed pa yin no/ /\\

\textbf{\TVB}\\
zhe na/ thams cad du bsgom pa'i yul gyi dngos po ni/ 'khrul pa yin pas cung zhig kyang bsgom par bya ba med par 'gyur la/ de bas na rnam pa thams cad du thar pa'i lam bsgom pa la don med par 'gyur ro/ /'khrul pa'i ngo bo la yang sgyu ma lta bu'i rnam par zhugs pas/ spros pa med pa'i sgom par 'gyur ba la skyon yod pa ma yin no zhe na/ spros pa de la sgyu ma'i rnam par zhugs pa mtshungs pas de la yang de skyon gyi skabs ci zhig yod/ de bas na don du gnyer bas spros pa'am/ spros pa med pa yang rung/ gang kho na 'dod pa'i tshad ma dang ldan pa 'am/ gcig shos de nyid la le lo spangs nas phyag rgya chen po don du gnyer bas sgom par bya ste/ shin tu thal bas chog go/ /de la snying po dang gcig shos kyi rnam par dbye ba yang/ <yan lag bdun pa nas> bla ma mnyes par byed pa rnams kyis shes pa yin no/ /\\

tṛtīyapakṣe kuto doṣā nīrasatvena te prayojanābhāvān mantranayakramābhāvāc ca na pañcamaḥ parikṣīṇadoṣaḥ |\\

\textbf{\TVA}\\
dgos pa la sogs pa gsum pa'i phyogs la bshad pa'i nyes pa dang/ gsang sngags kyi tshul gyi rim pa med pa'i phyir/ lnga pa skyon dang bral ba ma yin no/ /\\

\textbf{\TVB}\\
gsum pa'i phyogs la bshad pa'i nyes pa yod pa dang/ dgos pa med pa'i phyir dang/ gsang sngags kyi tshul gyi rim pa med pa'i phyir lnga pa skyon dang bral ba ma yin no/ /\\

nanu sākṣātkaraṇāt pūrvaṃ mantranayaprayogo 'sti, tatkathaṃ tasyābhāvaḥ? satyam, sākṣātphalāvasthā sādhyā | tasyāṃ ca nāsty asau kramaḥ | sākṣātparityāge ca na prayojanam utpaśyāma iti | svecchayā nirvāpayitum aśakyatvāt prayojanābhāvāt sattvārthābhāvāc ca na prapañcāntaraprabhedakalpanā kalaṅkāśūnyā | tathā hi kasyacid nivṛttiḥ kāraṇanivṛttyā vyāpakanivṛttyā vā bhavati | na cātra sākṣātkṛtamaṇḍalacakrasya nivartakaṃ kāraṇaṃ vyāpakaṃ vā icchākāle dṛśyate |\\

\textbf{\TVA}\\
sngar sngon du byas pa'i phyir/ sngags kyi tshul la rab tu sbyor ba yod pa ma yin nam/ ci'i phyir de la med ce na/ bden te/ bsgrub par bya ba 'bras bu mngon du gyur pa'i gnas skabs de yang rim pa 'di la med pa dang/ mngon du gyur pa yongs su btang ba dang/ dgos pa ma mthong ba'i phyir ro/ /rang gi 'dod pas bzlog mi nus pa'i phyir dang/ dgos pa med pa'i phyir dang/ sems can gyi don med pa'i phyir/ lnga pa'i mtha'i rab tu dbye ba rtog pa'i dri mas stong pa ma yin no/ /de la ji ltar bzlog mi nus she na/ ji ltar 'ga' zhig ldog pa ni rgyu ldog pa'am/ khyab par byed pa bzlog pa las 'gyur na/ de la mngon sum du byas pa'i dkyil 'khor gyi 'khor lo yang ldog par byed pa'i rgyu'am/ khyab par byed pa ldog pa yong pa ma yin no/ /rang gi 'dod pas 'gog pa yang mi nus te mi mthun pa med pa'i phyir/ sdug bsngal la sogs pa 'gog pa 'dod kyang sdug bsngal la sogs pa la 'jug pa mthong ba'i phyir ro/ /\\

\textbf{\TVB}\\
sngar mngon sum du byas pa'i phyir sngags kyi tshul la rab tu sbyor ba yod pa [\TVB\ fol.\ 82r] ma yin nam/ de ci'i phyir de la med ce na bden te/ bsgrub par bya ba 'bras bu mngon du gyur pa'i gnas skabs na de yang rim pa 'di la med pa dang/ mngon sum du gyur pa yongs su btang ba la dgos pa yang ma mthong ba'i phyir ro/ /rang gi 'dod pa bzlog mi nus pa'i phyir dang/ dgos pa med pa'i phyir dang/ sems can gyi don med pa'i phyir lnga pa'i mtha'i rab tu dbye ba rtog pa'i dri mas stong pa ma yin te/ 'di ltar 'ga' zhig ldog pa ni rgyu ldog pa'am/ khyab par byed pa ldog par 'gyur na/ de la mngon sum du byas pa'i dkyil 'khor gyi 'khor lo ldog par byed pa ni/ rgyu'am khyab par byed pa log pas yang ma yin no/ rang gi 'gog par yang mi nus pa ni mi mthun pa med pa'i phyir te/ sdug bsngal la sogs pa 'gog par 'dod pa na yang sdug bsngal la sogs pa la 'jug pa mthong ba'i phyir ro/ /\\

nanu śūnyataiva nivartikāsti | yathā dārusaṃghāte prajvalito vahnir niḥśeṣamindhanaṃ bhasmīkṛtya paścāt svarasata eva nivartate, tathā maṇḍalacakraprajvalitaḥ śūnyatājñānāgniḥ sākṣān maṇḍalacakraṃ nivartayiṣyatīti cet, tad asat, viṣamatvād dṛṣṭāntasya | tathā hi tatrendhanaṃ na kāraṇaṃ vahneḥ | kāryam indhanalakṣaṇanivṛttau yuktaiva vahnilakṣaṇasya kāryasya nivṛttiḥ | iha tu na śūnyatā kāraṇaṃ maṇḍalacakrasya tat ka[\MS\ fol.\ 9r]thaṃ tannivṛttau nivṛttiḥ | athavā na ca śūnyatāyā nivṛttir asti |\\

\textbf{\TVA}\\
stong pa nyid kyis ldog par byed pa yod pa ma yin nam/ 'di ltar shing gi phung po la me rab tu 'bar bas shing ma lus pa thal bar byas nas de'i 'og tu bdag nyid kyang rang gi dang kho nas ldog par 'gyur ba de dkyil 'khor gyi 'khor lo stong pa nyid kyi ye shes kyi me rab tu 'bar bas mngon sum du byas nas kyang/ dkyil 'khor gyi 'khor lo ma lus par ldog par byed la/ de yang rang gi ldog par 'gyur ro zhe na/ de ni bden pa ma yin te/ dpe dang mthun [\TVA\ fol.\ 212v] pa ma yin pa'i phyir ro/ /de la 'di ltar shing 'gyur bar byed pa'i rgyur gyur pa'i me ni shing gi 'gyur ba yang dag par skyed pa nyid kyis shing 'gyur bar byed pa ni me yin no zhes 'jig rten pa rnams sems la 'dir na stong pa nyid kyi dkyil 'khor los sgyur bar byed pa'i rgyu ma yin na de'i phyir ldog par byed pa yin no/ /de ci'i phyir zhe na/ stong pa nyi da la rang gi ngang gis ldog go zhes kyang smra bar bya ba ma yin no/ //\\

\textbf{\TVB}\\
stong pa nyid kyis ldog par byed pa yod ma yin nam/ 'di ltar shing gi phung po la me rab tu 'bar bas shing ma lus pa thal bar byas nas/ de'i 'og tu bdag nyid kyang <me rang> rang gi ngang kho nas ldog par 'gyur ba de ltar dkyil 'khor gyi 'khor lo stong pa nyid kyi ye shes kyi me rab tu 'bar bas mngon sum du byas nas kyang/ dkyil 'khor gyi 'khor lo ma lus par ldog par byed la/ bdag nyid kyang rang gi ngang gis ldog par 'gyur ro zhe na/ de ni bden pa ma yin te/ dpe la skyon yod pa'i phyir ro/ /[\TVB\ fol.\ 82v] 'di ltar de la shing 'gyur bar byed pa'i rgyur gyur pa'i me ni shing 'gyur ba yang dag par skyed pa nyid kyis/ shing 'gyur bar byed pa ni me yin no zhes 'jig rten pa rnams sems la/ 'dir ni stong pa nyid ni dkyil 'khor gyi 'khor lo 'gyur bar byed pa'i rgyu ma yin na/ de'i phyir de ldog par byed pa yin/ stong pa nyid la rang gi ngang gis ldog go zhes kyang smra bar bya ba ma yin no/ /\\

nanu sā [na] bhavatu kāraṇaṃ śūnyatā vyāpakaṃ tu bhaviṣyati | vyāpakasya vṛkṣasya nivṛttau śiṃśapātvasya vyāpyasya nivṛttivan nivṛttir bhaviṣyatīti cet, etad apy asāram | tathāhi śūnyatā sarvadā sarvajñeyamaṇḍalavyāpikā tattvarūpā | na ca tasyā nivṛttiḥ kadācid apy asti | yadi syāt samyaksaṃbodhisākṣātkaraṇāt [\EDD\ p.\ 146] pūrvam anantaram eva vā nivṛttiḥ syāt | na ca bhavati | samyaksaṃbuddhībhūyāpi katipayakālāvasthānasya svayam eva svīkṛtatvāt | kintu śūnyatāpi jñānarūpā, cakram api jñānarūpam | śūnyatājñānotpattyā cakrajñānasyānivṛttau śūnyatājñānaṃ kena nivartanīyam | tena nivṛttiś ca virodhino 'bhāvāt kāraṇavyāpakayoś cābhāvān nāsti tasmāc chūnyatājñānasya nivṛttiḥ | nāpi maṇḍalacakrasya śūnyatāto nivṛttiriti | śūnyatā na nivartikā | ko brūte śūnyatā nivartikā? kiṃ tarhi yan nivartakās tad gurūpadeśato jñeyam ity apy asāram | gurūpadeśato 'pi na śūnyatāvyatiriktaḥ pramāṇato 'stīti | yat kiñcid etat | pratikṣaṇanivṛttiś ca kṣaṇabhaṅgarūpā sarvapadārthavyāpinī | na sā santānanivartikā | tasmān na svecchayā nivṛttiḥ | na ca nivartyā (vṛttyā) nīrasarūpayā prayojanam asti prekṣāvatām | tathā coktam\emdash \\

\textbf{\TVA}\\
rgyur mi 'gyur du chug na yang stong pa nyid khyab par byed pa ni 'gyur ba ma yin nam/ khyab par byed pa shing log pas khyab par shing sha pa ldog pa bzhin du ldog pa 'gyur ro zhe na/ de yang snying po ma yin te/ 'di ltar thams cad du shes bya'i dkyil 'khor thams cad la khyab par byed pa de kho na nyid kyi ngo bo stong pa nyid de la ldog pa nam yang yod pa ma yin no/ /gal te ldog par 'gyur na yang dag par rdzogs pa'i byang chub mngon sum du byed pa las dus snga ma'am/ de ma thag tu sangs rgyas bcom ldan 'das ldog par 'gyur na ma yin te/ yang dag par rdzogs pa'i sangs rgyas su gyur nas kyang thugs rjes dus ji srid cig gi bar du gnas pa nyid kyis zhal gyis bzhes pa'i phyir 'gyur ba yang ma yin no/ /'on kyang stong pa nyid kyang ye shes kyi ngo bo nyid la/ 'khor lo yang ye shes kyi ngo bo nyid yin no/ /ci ste stong pa nyid kyi ye shes skyes pas 'khor lo'i ye shes lo ga na stong pa nyid kyi ye shes las gang gis ldog par byed/ de nyid kyis ldog pa ni 'gal ba'i phyir la/ rgyu'am khyad par byed pa med pa de bas na stong pa nyid kyi ye shes la ldog pa med do/ /de bas na dkyil 'khor gyi 'khor gyi 'khor lo yang stod pa nyid kyis ldog go zhes bya ba yang ma yin te/ stong pa nyid la ldog par byed pa yin pa/ log pa dang stong pa nyid kyis ldog par byed pa yin no zhes su zhig smra/ 'on kyang ldog par byed pa de ni bla ma'i man ngag las shes so zhes bya ba 'di yang snying po ma yin te/ bla ma'i man ngag las kyang stong pa nyid dang/ de ldog pa las ma gtogs pa'i ldog par byed pa'i tshad ma gzhan cung zad yod pa ma yin no/ /so sor skad [\TVA\ fol.\ 213r]/ /cig ma'i 'jig pa dang/ skad cig gis nyams pa'i ngo bos dngos po thams cad la khyab mod kyi/ /rgyud kyi rgyun ldog pa ni ma yin no/ /de'i phyir rang gi bdog pa brtag par mi bya ste ldog pa'i snying po med pa la rtog pa dang ldan pa rnams kyis brtags pa dgos pa yod pa ma yin no/ /de skad du yang/ \\

\textbf{\TVB}\\
rgyur mi 'gyur du chug na stong pa nyid khyab par byed pa ni 'gyur ba ma yin nam/ khyab par byed pa shing ldog pas khyab par bya ba shing sha pa ldog pa bzhin du ldog par 'gyur ro zhe na/ de yang snying po ma yin te 'di ltar shes bya'i dkyil 'khor thams cad la khyab par byed pa de kho na nyid kyi stong pa nyid de la ldog pa yang yod pa ma yin no/ /gal te 'gyur na yang dag par rdzogs pa'i byang chub mngon sum du byed pa las dus snga ma'am/ de ma thag tu sangs rgyas bcom ldan 'das ldog par 'gyur na/ de lta ni ma yin te/ yang dag par rdzogs pa'i sangs rgyas su gyur nas kyang thugs rjes dus ji srid cig gi gnas skabs nyid kyis zhal gyis bzhes pa'i phyir 'gyur ba yang ma yin no/ /'on kyang stong pa nyid kyang ye shes kyi ngo bo yin la/ 'khor lo yang ye shes kyi ngo bo nyid yin no/ /stong pa nyid kyi ye shes skyes pas 'khor lo'i ye shes ni ldog na/ stong pa nyid kyi ye shes gang [\TVB\ fol.\ 83r] gis ldog par byed/ de nyid kyis <rang gi ngang> ldog pa yang 'gal ba med pa'i phyir dang/ rgyu 'am khyab par byed pa med pa las <ngang gis> med pa bas na/ stong pa nyid kyi ye shes la ldog pa med do/ /dkyil 'khor gyi 'khor lo yang stong pa nyid kyis ldog go zhes bya ba yang ma yin te/ stong pa nyid kyis ldog par byed pa ma yin pa la/ stong pa nyid kyis ldog par byed pa yin zhes su zhig smra/ 'on kyang ldog par byed pa gang yin pa de ni bla ma'i man ngag las shes so zhes zer ba 'di yang snying po ma yin te/ bla ma'i man ngag las kyang stong pa nyid kyis ldog par byed pa ma yin ldog pa'i tshad ma cung zhig kyang yod pa ma yin pas so sor skad cig ma'i 'jigs pa dang/ skad cig gis nyams pa'i ngo bos dngos po thams cad la khyab pa'i tshe rgyud kyi rgyun ldog par byed pa ni ma yin no/ /de'i phyir rang gi 'dod pas mi 'gyur te/ ldog pa'i snying po med pa yang rtog pa dang ldan pa rnams la dgos pa yod pa ma yin no/ /de skad du yang/

\begin{quote}
	mucyamāneṣu sattveṣu ye te prāmodyasāgarāḥ | \\
	tair eva nanu paryāptaṃ mokṣeṇārasikena kim ||\footnoteA{
		Śātideva, \emph{Bodhicaryāvatāra} 8.108
	}

	\textbf{\TVA}\\
	sems can rnams ni sgrol gyur pa/ /\\
	rab dga' rgya mtsho gang yin de/ /\\
	de nyid kyis rdzogs ma yin nam/ /\\
	thar pa ro bral ci zhig dgos/ /

	\textbf{\TVB}\\
	sems can rnams ni grol gyur <don bya> la/ /\\
	rab dga' rgya mtsho <'khor ba mya ngan 'das pa> gang yin de/ /\\
	de nyid kyis rdzogs <mi gnas pa'i mya nga ngan 'das> ma yin nam/ /\\
	thar pa ro bral <lhag med> ci zhig dgos/ /
\end{quote}

sattvārtho 'pi nivṛttau nāsti | na hi gagane gaganakamale vā kācidarthakriyā saṃbhavati | ciraniruddhād apy atītādavasturūpāc cakrāt sattvārtho bhaviṣyatīty apy asāram, virutasyāpi kukku[\MS\ fol.\ 9v]ṭasya kaṇṭhadhvaniprasaṅgāt |\\

\textbf{\TVA}\\
zhes gsungs so/ /sems can gyi don yang ldog pa la med de/ nam mkha'i me tog la nam yang don gyi bya ba cung zhig kyang mi srid de/ yun ring por 'gags kyang dngos po'i ngo bo 'das pa'i 'khor lo las sems can gyi don 'byung bar 'gyur ro zhes bya ba yang snying po ma yin te/ yun ring por khyim bya shi ba yang sgra sgrogs par thal bar 'gyur ba'i phyir ro/ /\\

\textbf{\TVB}\\
zhes gsungs so/ /sems can don yang ldog pa la med do/ /nam ka dang nam ka'i me tog la nam yang don gyi bya ba cung zhig [\TVB\ fol.\ 83v] kyang mi srid do/ /yun ring por 'gags kyang dngos po med pa'i ngo bo 'das pa'i 'khor lo las sems can gyi don 'byung bar 'gyur ro/ /zhes bya ba yang snying po ma yin te/ yun ring por lon pa'i khyim bya shi ba'i sgra sgrogs par thal ba'i phyir ro/ /\\

nanu yogyadhiṣṭhānād gaganād apy arthakriyāḥ saṃbhavanti? na saṃbhavanti, yogyadhiṣṭhānād eva cittarūpād arthakriyā na gaganāt, nīrūpatvāt tasya |\\

\textbf{\TVA}\\
rnal 'byor pa'i byin gyi rlabs kyis nam mkha' las kyang dngos po'i ngo bo 'das pa'i 'khor lo las sems can gyi don byed pa yin la/ nam mkha' ni ma yin te/ de'i ngo bo nyid ma yin pa'i phyir ro/ /\\

\textbf{\TVB}\\
'on te rnal 'byor pa'i byin gyis brlabs kyis nam mkha' las kyang don byed pa srid pa ma yin nam zhe na/ rnal 'byor pa'i byin gyi brlabs nyid sems kyi ngo bo nyid kyis don byed pa yin la/ nam mkha' ni ma yin te/ de'i ngo bo nyid ma yin pa'i phyir ro/ /\\

nanu nirodhya maṇḍalacakraṃ sattvārthakāle punar utpādyate, tato 'rthakriyā bhavati, tataḥ punar eva nirodhyate, punar evotpadyata iti cet, asadetat | yathā sattvārthakriyāyās tattvato nāsti prādurbhāvaḥ, tathā cakrasyāpi | tato nārthakriyāyāḥ sambhavaḥ | na ca nirodhya punar utpāde kiñcit prayojanam astīty alam atiprapañceneti | ṣaṣṭhapakṣoktasaṃdohasyāṣṭame 'pi bhāvān na piṣṭapeṣaṇaṃ kriyate |\\

\textbf{\TVA}\\
dkyil 'khor 'khor lo 'gags pa las sems can gyi don byed pa'i dus na yang 'byung yang de las don byed bar 'gyur ba ma yin nam/ de bas na 'gags shing yang skye bar 'gyur ro zhe na/ de ni bden pa ma yin te/ ji ltar sems can gyi don byed pa de las byung ba med pa de bzhin du don byed pa'i 'khor lo yang de la mi srid do/ /'gags pa las kyang 'byung ba la dgos pa cung zad kyang yod pa ma yin te/ shin tu spyos pas chog go/ /\\

\textbf{\TVB}\\
'on te dkyil 'khor gyi 'khor lo 'gags pas las sems can don byed pa'i dus na yang 'byung zhing/ de las don byed par 'gyur ba ma yin nam/ de bas yang 'gog cing yang skye bar 'gyur ro zhe na/ de ni bden pa ma yin te/ ji ltar sems can gyi don byed pa de las 'byung ba med pa de bzhin du 'khor lo yang de las mi srid do/ /'gags pa las yang 'byung ba la yang dgos pa cung zad yod pa ma yin te/ shin tu spros pas chog go/ /\\

nanu ṣaṣṭhena saptamasya samānatvāt kathaṃ saptamasya tato viśeṣaḥ? asti viśeṣaḥ | pūrvāvasthāyāṃ niyatacakrākāratā, punaḥ svecchayā svecchotpādanaṃ ceti | saptame punar etan nāsti | tato na samānatā bhinnaś ca nirdiṣṭa iti || 16 ||\\

\textbf{\TVA}\\
drug pa'i phyogs la bshad pa'i skyon gyis tshogs bdun pa la yang yod pas brtag pa la yang brtag par mi bya'o/ /drug pa dang bdun pa mtshungs pa nyid kyi phyir de la khyad par ci yod ce na/ khyad par yod de/ snga ma'i gnas skabs 'khor lo'i rnam pa nyid nges par byas nas yang dang yang du rang gi 'dod pas skyed par byed pa nyid yin la/ bdun pa la ni de med pa de bas na mtshungs pa dang tha mi dad pa ma yin par bstan to/ /\\

\textbf{\TVB}\\
drug pa'i phyogs la bshad pa'i skyon gyi tshogs bdun pa la yang yod pas/ btags pa la 'thag pa mi bya'o/ /drug pa dang bdun pa mtshungs pa nyid kyi phyir de las bdun pa khyad par ci yod ce na/ khyad par yod de snga ma'i gnas skabs su 'khor lo'i rnam pa nyid nges par byas nas/ yang rang gi 'dod pas 'gog cing rang [\TVB\ fol.\ 84r] gi 'dod pas skyed par byed pa'i phyir la/ bdun pa la yang med pa de bas na mtshungs pa dang tha mi dad pa ma yin par bstan to/ /

\subsection{Verse 17}
\subsubsection{Root Text}
\begin{quote}
	dambholibījasruti\footnoteB{
		°sruti°] \corr ; śruti \MS\ \EDD
	}dhautaśuddha-\\
	pāthoja\footnoteB{
		pāthoja°] \EDD\ (\emph{\EDD reports the ms.\ as reading \emph{pāthojña}, but this seems to be incorrect}); pāthauja° \MS
	}bhūtāṅkurabhūtapuṣṭi |\\
	turīyaśasyaṃ\footnoteB{
		turīyaśasyaṃ] \EDD; tutīyaśasyaṃ \MS
	} paripākam eti\footnoteB{
		eti] \EDD\ (\emd); eta \MS
	} \\
	sphuṭaṃ caturthaṃ viduṣo 'pi gūḍham || 17 ||

	rdo rje'i sa bon dri med dga' 'bab pa/ /\\
	dag pa'i chu skyes yang dag myu gu 'byung rgyas shing/ /\\
	bzhi pa'i 'bru ni yongs gsal smin 'gyur ba'i/ /\\
	bzhi pa dag ni mkhas pa rnams la sbas/ /
\end{quote}

[\EDD\ p.\ 147] dambholītyādi | etat sadgurūpadeśato jñeyam || 17 ||\\

\textbf{\TVA}\\
rdo rje'i sa bon zhes bya ba la sogs [\TVA\ fol.\ 213v] pa la/ de dag ni bla ma'i man ngag las shes par bya'o/ /\\

\textbf{\TVB}\\
rdo rje'i sa bon zhes bya ba la sogs pa la/ 

\subsection{Verse 18}
\subsubsection{Root Text}
\begin{quote}
	pañcapradīpāmṛtabinducandra-\\
	bhrūmadhyabindūdbhavamaṇḍalāni |\\
	vāyoḥ svarūpaṃ galaśuṇḍikādyam \\
	atattvarūpaṃ svayam ūhanīyam || 18 ||

	gsal byed bdud rtsi lnga dang thig le zla ba dang/ \\
	smin dbus thig le las byung dkyil 'khor dang/ /\\
	rlung gi rang bzhin lce'u chung sogs de nyid/ /\\
	ngo bo min pa bdag nyid kyis rtag tu/ /
\end{quote}

pañcapradīpetyādi | pañcapradīpaśabdena gokudahanalakṣaṇasya, amṛtaśabdena vimumāraśulakṣaṇasya satatānuṣṭhānam eva sādhyaṃ manyante | bindur iti hṛccandrasthaṃ binduṃ dedīpyamānaṃ tattvaṃ sādhyaṃ ceti kṛtvā kecid bhāvayanti | candra iti hṛdisthaṃ kalārūpam arddhacandraṃ vā hṛtkamalasthaṃ kecidb hāvayanti | bhrūmadhyabindūdbhavamaṇḍalānīti bhruvor madhye ūrṇāyāṃ binduṃ vibhāvya tadbindūdbhavāni maṇḍalāni vāyuvāruṇamāhendrāgneyalakṣaṇāni | etad uktaṃ bhavati\emdash mukhaśravaṇanāsikācakṣurghrāṇarasanāni hastāṅgulībhiḥ pidhāya bhrūmadhyabindur draṣṭavyaḥ | tasya sphuṭāvasthāyāṃ śubhāśubhanimitta[\MS\ fol.\ 10r]saṃsūcakāni māhendrādimaṇḍalāny upajāyante | taṃ ca binduṃ tattvam iti manyante | vāyoḥ svarūpam iti | pūrakakumbhakarecakapraśāntakalakṣaṇam ānāpānādilakṣaṇaṃ ceti | [e]tad uktaṃ bhavati | sai(śai)vasaṃ(sāṃ) khyādinirdiṣṭaṃ vāyusvarūpaṃ jñātvā taṃ vāyuṃ nirodhabhāvanayā sthirīkṛtyākāśenotplutya gamanaṃ parapurapraveśaṃ yāvan muktiṃ ca sākṣātkurvanti vāyuvādinaḥ | galaśuṇḍiketi | galapradeśe jihvāmūlopari hastiśuṇḍikākārā adhaḥpralambamānā upajihvāsaṃjñikā galaśuṇḍikāsti | sā ca śaktirūpā, tadadhaḥ śivarūpam asti tattvam | sā ca [\EDD\ p.\ 148] jihvāgreṇa spṛśyamānā nirantarāmṛtaṃ sravati | tena ca ghargharāmṛtavarṣaṇena santarpyamānam ātmānaṃ dhyāyād iti galaśuṇḍikātattvam | ādiśabdena hṛnmadhyaṣoḍaśanāḍikācakramadhyasthajñānasvarūpaṃ śivarūpaṃ tattvaṃ bhāvayitavyam ityādīnāṃ parigrahaḥ | tatsarvaṃ tīrthikādibhis tattvarūpeṇābhimatam | atattvam iti | svayam evohanīyaṃ vicāraṇīyam iti yāvat || 18 ||\\

\textbf{\TVA}\\
gsal byed lnga zhes bya ba la sogs pa la/ gsal byed lnga zhes bya ba'i sgra la/ ba lang dang/ khyi dang/ glang po dang/ rtag dang/ mi'i mtshan nyid do/ /bdud rtsi lnga'i sgra ni/ bi dang/ mu dang/ ma dang/ ra dang/ shu'i mtshan nyid do/ /de dag rtag tu brten pa kho nas bsgrub par bya bar sems so/ /thig le zhes bya ba ni/ snying gar zla ba la gnas pa'i thig le 'od 'phro bar gyur pa la de kho na nyid bsgrub par bya ba yin no zhes sems nas rnam par sgom par byed do/ /zla ba zhes bya ba ni snying ga'i gnas su zla ba phyed pa'am/ zla ba rgyas pas snying gar padma la gnas pa 'ga' zhig rnam par sgom par byed do/ /smin dbus thig le las byung dkyil 'khor dang/ /zhes bya ba ni smin ma'i dbus te mdzod spu'i gnas su thig le rnam par bsgom par bya'o/ /thig le de las byung ba'i dkyil 'khor ni/ rlung dang/ chu dang/ dbang chen gyi dkyil 'khor dang me'i mtshan nyid do/ /'di skad du bstan par 'gyur te/ kha dang/ rna ba dang/ sna dang/ mig lag pa'i sor mos bkab la/ smin ma'i dbus su thig le blta bar bya'o/ /de la yang gsal por gyur pa'i gnas skabs su dge ba dang mi dge ba'i ltas yang dag par ston par byed pa/ dbang chen la sogs pa'i dkyil 'khor rab tu bskyed pa'i thig le de la yang de kho na nyid du sems so/ /rlung gi rang bzhin zhes bya ba ni pu ra ka dang/ kum pa ka dang/ re tsi ka dang/ sh'anta ka ste/ dbugs 'byung rngub la sogs pa'i mtshan nyid do/ /'di skad du bshad pa yin te/ de la 'di ltar grangs can la sogs pas bstan pa'i rlung gi rang bzhin shes par byas nas/ dbugs bsgags pa las brtan por byas nas/ rnam mkha' la rgyu bas 'ong ba dang/ gzhan gyi grong du 'jug pa dang/ grol ba'i bar du mngon sum du byed par 'gyur ba'o zhes bya ba ni rlung gi de kho na nyid du smra ba rnams kyi'o/ /lce'u chung zhes bya ba ni med pa'i phyogs lce'i phugs kyi rang gi steng na glang po che'i sna'i rnam pa lta bu kha 'og tu bltas pa lce'i ming [\TVA\ fol.\ 214r]/ /can du lce'u chung yod de/ de yang nus pa'i ngo bo sdig pa'i rang bzhin du yong pa de kho na nyid yin no/ /de la lce'i rtse mos reg par gyur tsam na bdud rtsi rgyun mi 'chad par 'jig par 'gyur ro/ /bdud rtsi rgyun babs pa des kyang don dam pa'i bdag nyid tshim par 'gyur ba bsgom zhes bya ba ni lce'u chung gi de kho na nyid yin no/ /sogs pa'i sgras ni snying ga'i dbus kyi dkyil 'khor rtsibs bcu drug pa'i dbus na h'um gnas pa ye shes kyi rang bzhin bzhi pa'i ngo bo de kho na nyid bsgom par bya'o zhes bya ba la sogs pa bzung ste/ yang smras pa/ bcu las drug lhag rtsa dang ldan pa'i 'khor lo yi/ /dkyil na gnas pa'i snying gar rnam par gnas pa'i bdag /des ni de yi khyad par lta bu'i grub pa ster/ /de ni mngon par mi g-yo ba yi yid dag gis/ /rnal 'byor pa yi sems de de ltar mngon par bsam/ / nub par gyur pa'i mgon po rgyal bar gyur de ni/ /nus pa dag gis de ni yongs su bskor dang bcas/ /zhes bya ba de lta bu mu stegs byed pa la sogs pa de kho na nyid du 'dod pa de dag thams cad ni de kho na nyid ma yin no zhes bdag nyid kyis brtags par bya/ dpyad par bya'o zhes bya ba'i bar du'o/ /\\

\textbf{\TVB}\\
gsal byed lnga'i sgra ni/ ba glang dang/ khyi dang/ glang po dang/ rta dang/ mi'i mtshan nyid do/ /bdud rtsi lnga'i sgra ni/ be <'bhe sta phyi> dang/ mu <tra chu> dang/ mang <'ad sha> dang ra dang <ka ta khrag> shu'i mtshan nyid do/ /rtag tu bsten pa kho na bsgrub byar sems so/ /thig le zhes bya ba ni/ snying kar zla ba la gnas pa'i thig le 'od 'phro bar gyur pa de kho na nyid bsgrub bya yin no zhes sems nas rnam par bsgom par byed do/ /zla ba zhes bya ba ni snying ka'i gnas su zla ba'i bzhi dum bu 'am zla ba phyed pa'am/ zla ba rgyas pa snying kar pad+ma la gnas pa 'ga' zhig rnam par bsgom par byed do/ /smin dbus thig le las byung dkyil 'khor zhes bya ba ni/ smin ma'i dbus te mdzod spu'i gnas su thig le rnam par sgom par bya'o/ /thig le de las byung ba'i dkyil 'khor ni rlung <g.yon dang> dang chu <ngal ba> dang dbang chen <gnyi ga> gyi dkyil 'khor dang <kha dog> <mthun par/> me'i mtshan <g.ya> nyid do/ /'di skad du bshad par 'gyur te/ kha dang rna ba dang sna dang mig lag pa'i sor mos bkab la smin ma'i dbus su thig le blta bar bya'o/ /de yang [\TVB\ fol.\ 84v] gsal por gyur pa'i gnas skabs su dge ba dang/ mi dge ba'i ltas yang dag par ston par byed pa dbang chen gyi la sogs pa'i dkyil 'khor rab tu skyed pa'i thig le de la yang de kho na nyid du sems so/ rlung gi rang bzhin zhes bya ba ni phu ra <'gengs byed> ka dang/ kum bha <'dzin byed> ka dang/ re tsa <'byin byed> ka dang/ rab tu zhi ba'i mtshan <tha ma sangs> nyid dang/ dbugs dbyung rngub la sogs pa'i mtshan nyid yin no/ /'di skad du bshad pa yin te/ de la 'di ltar shi ba dang grangs can la sogs pas bstan pa'i rlung gi rang bzhin shes par byas nas/ dbugs 'gag pa'i sgom pa brtan por byas nas nam mkha' la rgyu bas 'ong ba dang/ gzhan gyi grong la 'jug pa dang/ grol ba'i bar mngon sum du byed par 'gyur ro/ /zhes bya ba ni rlung gi de kho na nyid du smra ba rnams kyi'o/ /lce'u chung zhes bya ba ni/ mid pa'i phyogs lce'i phugs kyi steng na glang po che'i sna'i rnam pa lta bu kha 'og tu ltas pa nye ba'i lce'i ming can lce'u chung yod do/ /de yang nus pa'i rang bzhin no de'i 'og na zhi ba'i ngo bo yod pa <zhi ba'i ngo bo> de kho na nyid yin no/ /de la yang lce'i rtse mos reg par gyur pa na bdud rtsi'i rgyun mi 'chad par 'dzag par 'gyur ro/ rtsi'i rgyun 'bab pa des don dam pa'i bdag nyid tshim par gyur par sgom mo/ /zhes bya ba ni lce'u chung gi de kho na nyid yin no/ /sogs pa'i sgras snying ka'i dbus kyi 'khor [\TVB\ fol.\ 85r] lo rtsibs bcu drug pa'i dbus na gnas pa ye shes kyi rang bzhin zhi ba'i ngo bo de kho na nyid bsgom par bya'o zhes bya ba la sogs pa bzung ngo / /de yang smras pa/ bcu las drug lhag rtsa dang ldan pa'i 'khor lo'i dkyil na gnas pa snying kar rnam par gnas pa'i bdag /des ni de'i khyad par lta bu yi grub pa ster/ de ni mngon par mi g.yo ba'i yid dag gis/ /rnal 'byor pa yis de ltar mngon par bsam par bya/ /nus par gyur pa'i mgon po rgyal bar gyur/ /de ni nus pa dag gis de ni yongs su bskyor dang bcas/ /zhes de lta bu mu stegs la sogs pa de kho na nyid du mngon par 'dod pa de dag thams cad ni de kho na nyid ma yin zhes bdag nyid kyis brtag par bya/ dpyad par bya zhes bya ba'i bar du'o/ /

\subsection{Verse 19}
\subsubsection{Root Text}
\begin{quote}
	svapnendrajālapratibimbamāyā-\\
	marīcigandharvapurāmbu{[}\MS\ fol.\ 2r{]}candraiḥ |\\
	anyaiś ca sarvair upamābhidheyair \\
	naivāsti sādhyaṃ kathitādihānyat || 19 ||

	rmi lam mig 'phrul gzugs brnyan sgyu ma dang/ /\\
	smig rgyu dri za'i grong khyer chu zla dang/ /\\
	gzhan gyi sgra yis mngon par brjod pa yis/ /\\
	'dir bshad bsgrub bya gzhan la yod ma yin/ /
\end{quote}

\subsubsection{Commentary}
svapnendrajāletyādi | svapnendrajālopamaṃ pratibimbamāyāmarīcigandharvanagarodakacandropamam iti śabdair anyaiś ca gagaṇapratiśrutkaphenopamam ityādiśabdair upamābhidheyair upamāvācakair naivāsti sādhyaṃ kathitāt sādhyād anyat | paraṃ kathita eva sādhye ete śabdāḥ pravartanta iti svayaṃ boddhavyam || 19 ||\\

\textbf{\TVA}\\
rmi lam mig 'phrul zhes bya ba la sogs pa la/ rmi lam dang/ mig 'phrul lta bu dang/ gzugs brnyan dang/ sgyu ma dang/ smig rgyu dang/ dri za'i grong khyer dang/ chu zla lta bu'o/ /gzhan gyi sgras ni/ nam mkha' dang/ brag ca dang/ dbu ba lta bu'o zhes bya ba la sogs pa'i sgras so/ /de lta bu'i mngon par brjod pa ni smra ba pos de lta bu'i bstan pa'i bsgrub par bya ba las gzhan med pa nyid de/ gzhan yang 'dir bshad pa kho na bsgrub par bya ba yin no zhes sgra de de la 'jug par rang nyid kyis shes par bya'o/ /\\

\textbf{\TVB}\\
rmi lam mig 'phrul zhes bya ba la sogs pa la/ rmi lam dang/ mig 'phrul lta bu dang/ gzugs brnyan dang/ sgyu ma dang/ smig rgyu dang/ dri za'i grong khyer dang/ chu zla lta bu'o/ /gzhan gyi sgras ni nam mkha' dang/ brag cha dang/ dbu ba lta bu'o zhes bya ba la sogs pa'i sgras so/ /de lta bur mngon par brjod pa ni smra ba pos de lta bus bstan pa'i bsgrub bya las gzhan med pa nyid de/ gzhan du yang 'dir bshad pa kho na bsgrub bya yin no zhes sgra de las rab tu 'jug par rang nyid kyis shes par bya'o/ /

\subsection{Verse 20}
\subsubsection{Root Text}
\begin{quote}
	gambhīraśūnyapratibhāsamātra\footnoteB{
		°mātra°] \EDD ; mātraṃ \MS
	}-\\
	śāntāti\footnoteB{
		śāntāti] \EDD ; sāntādi \MS
	}sūkṣmānabhilāpyaśabdaiḥ |\\
	nirlepanīrūpa\footnoteB{
		nirlepanīrūpa°] \EDD\ (\emd) ; nirlepanīpa \MS
	}nirañjanādyair \\
	bhrāntir na kāryāparasādhyasattve || 20 ||

	zab mo stong nyid so sor snang ba tsam/ /\\
	zhi ba shin tu phra ba sgras brjod min/ /\\
	gos med dngos bral dri med la sogs dang/ /\\
	gzhan sgrub yod pa 'khrul par mi bya'o/ /

	rnam pa bdun po bkod pa'i ngo bo dag las nam mkha' ma lus pa'i/ /\\
	slob dbyings khyab par byed pa'i bsod nams gang zhig bdag gis thob gyur pa/ /\\
	des ni 'jig rten 'di dag dpe med pa yi rig mas 'khyud par 'gyur ba yi/ /\\
	mdzes pa'i sku las rgyal ba'i sprul pa dag gis 'gro ba'i don byed shog /
\end{quote}

\subsubsection{Commentary}
[\EDD\ p.\ 149] gambhīraśūnyaṃ pratibhāsamātraṃ śāntātisūkṣmamanabhilāpyaṃ nirlepaṃ nirupamaṃ (nīrūpam) nirañjanam | ādiśabdāt śivaṃ nirākāraṃ niṣprapañcam anādyantanidhanam i[\MS\ fol.\ 10v]tyādiśabdair bhrāntir na kartavyā | aparasādhyasattve | aparasya sādhyasya sattve sattāyām | ebhiḥ sarvair eva param api kiñcit sādhyaṃ kathitād astīti bhrāntir na kartavyā | atha nātikathitam eva sādhyam ebhiḥ sarvair abhidhīyata iti niścayaḥ || 20 ||\\

\textbf{\TVA}\\
zab mo stong nyid so sor snang ba tsam/ zhi ba shin tu phra ba sgras brjod min/ /gos med dngos bral dri med la sogs dang/ /zhes bya ba la sogs pa'i sgras zhi ba dang/ rnam pa med pa dang/ spros pa med pa dang/ thog ma med pa dang/ 'gag pa med pa zhes bya ba la sogs pa'i sgra yis kyang 'khrul par mi bya zhing/ gzhan [\TVA\ fol.\ 214v] sgrub tu yod pa ni gzhan gyis bsgrub par bya ba yod pa ni 'dir yod pa'i sgra des bshad pa'i bsgrub par bya ba las mchog 'ga' zhig yod do zhes 'khrul par mi bya ste/ bshang pa'i bsgrub par bya ba kho na de dag thams cad kyis brjod par bya ba yin zhes nges par bya'o/ /\\

\textbf{\TVB}\\
{[}\TVB\ fol.\ 85v{]} zab mo stong nyid so so snang ba tsam/ /zhi ba shin tu sgras brjod min/ gos dngos bral dri med la sogs dang/ /zhes bya ba la sogs pa'i sgra yis/ zhi ba dang rnam pa med pa dang/ spros pa med pa dang/ thog ma med pa dang/ 'gag pa med pa zhes bya ba la sogs sgra yis kyang 'khrul par mi bya zhing/ gzhan bsgrub tu yod pa ni/ gzhan gyi bsgrub par bya ba yod pa ste/ 'dir yod pa'i sgras des bshad pa'i sgrub par bya ba las mchog 'ga' shig yod do zhes 'khrul par mi bya ste/ bshad pa bsgrub bya kho na de dag thams cad kyis brjod par bya ba yin zhes nges par bya'o/ /\\

\subsection{Conclusion}
\subsubsection{Root Text}
\begin{quote}
	akhilagagaṇagarbhavyāpisaptaprakāra\footnoteB{
		°saptaprakāra°] \EDD ; °sarvaprakāra° \MS
	}-\\
	grathitavacanarūpādyan mayāsādi puṇyam |\\
	anupamasukhavidyāsaktasaddehanirmi-\\
	jjinajanitajanārthas tena loko 'yam astu ||

	|| tattvaratnāvalokaḥ samāptaḥ | kṛtir iyaṃ paṇḍitavāgīśvarakīrtipādānām ||\\

	de kho na nyid rin po che snang ba slob dpon ngag gi dbang phyug grags pas mdzad pa rdzogs so// 
\end{quote}

\subsubsection{Commentary}
śrīsamāje parā yasya bhaktirniṣṭhā ca nirmalā\\
tasya vāgīśvarasyeyaṃ kṛtir vimatināśinī ||\\

\textbf{\TVA}\\
dpal ldan gsang ba 'dus pa las/ /\\
dri med dag mchog mthar phyin pa'i/ /\\
ngag gi dbang phyug de yis 'di/ /\\
byas pas blo ngan 'jig gyur cig /\\

\textbf{\TVB}\\
dpal ldan gsang ba 'dus pa las/ \\
dri med dad mchog mthar phyin pas/ /\\
ngag gi dbang phyug de yis 'di/ /\\
byas pas blo ngan 'jigs gyur cig /\\

vikacakumudakṣīratārakundānukāri\\
pracitam api ca puṇyaṃ yan mayā granthito 'smāt |\\
anupamasukhapūrṇaḥ svābhavidyopagūḍho\\
bhavatu nikhilalokas tena vāgīśvaraśrīḥ ||\\

\textbf{\TVA}\\
bdag gis 'di bshad byas pa las/ /\\
pad dkar kunda kha bye lta bu dang/ /\\
'o ma skar ma lta bu'i bsod nams gang bsags pa/ /\\
de yis 'gro ba bdag nyid dang mtshungs rig mas nyer 'khyud pa'i/ /\\
dpe med bde chen gang ba'i dbang phyug dpal gyur shog /\\

\textbf{\TVB}\\
bdag gis 'di bshad pa las pad kar kun dakha bye dang/ /\\
'o ma lta bur skar ma lta bu'i bsod nams dpag med gang bsags pa/ \\
'gro ba ma lus pa dag nyid mtshungs rig mas nyer 'khyud pa'i/ /\\
dpe med bde chen gang ba'i ngag gi dbang phyug dpal 'gyur shog// //\\

|| tattvaratnāvalokavivaraṇaṃ samāptam || kṛtir iyaṃ paṇḍitācāryavāgīśvarakīrtipādānām ||\\

\textbf{\TVA}\\
de kho na nyid rin po che snang ba'i rnam par bshad pa/ slob dpon ngag gi dbang phyug grags pas mdzad pa rdzogs so// //lo ts'a ba 'gos lhas btsas kyis rgya gar shar phyogs kyi dpe las zhu gtugs g-yar khral 'tshal ba lags so//\\

\textbf{\TVB}\\
de kho na nyid rin po che snang ba'i rnam par bshad pa/ slob dpon ngag gi dbang phyug grags pas mdzad pa rdzogs so// //

\end{document}
